\begin{filecontents*}{synthesis.bib}
@article{shen2025,
  title = {Tight Generic PRF Security of HMAC and NMAC},
  author = {Yaobin Shen and Xiangyang Zhang and Lei Wang and Dawu Gu},
  year = {2025},
  journal = {IACR Cryptology ePrint Archive},
  url = {https://www.semanticscholar.org/paper/0b676b3075484b7db7b6741bc589ce02e4bdb407},
}

@article{bellare2006,
  title = {New Proofs for NMAC and HMAC: Security without Collision Resistance},
  author = {M. Bellare},
  year = {2006},
  journal = {Journal of Cryptology},
  doi = {10.1007/s00145-014-9185-x},
  url = {https://www.semanticscholar.org/paper/c00e5fd533c5207ad445dbeebe0c82f223ff2665},
}

@article{yasuda2007,
  title = {Multilane HMAC - Security beyond the Birthday Limit},
  author = {K. Yasuda},
  year = {2007},
  journal = {International Conference on Cryptology in India},
  doi = {10.1007/978-3-540-77026-8_3},
  url = {https://www.semanticscholar.org/paper/e5de88cdb12481736bf56067076ed51a4669ab86},
}

@article{hosoyamada2021,
  title = {On Tight Quantum Security of HMAC and NMAC in the Quantum Random Oracle Model},
  author = {Akinori Hosoyamada and Tetsu Iwata},
  year = {2021},
  journal = {IACR Cryptology ePrint Archive},
  doi = {10.1007/978-3-030-84242-0_21},
  url = {https://www.semanticscholar.org/paper/ea19e5f9176d0de71d3ecfbb3d6b4134b54ae8b0},
}

@article{ye2017,
  title = {Verified Correctness and Security of mbedTLS HMAC-DRBG},
  author = {Katherine Q. Ye and M. Green and Naphat Sanguansin and Lennart Beringer and Adam Petcher and A. Appel},
  year = {2017},
  journal = {Conference on Computer and Communications Security},
  doi = {10.1145/3133956.3133974},
  url = {https://www.semanticscholar.org/paper/46906901407bac5ab595b55b05d661db5a798255},
}

@article{bellare2006b,
  title = {New Proofs for NMAC and HMAC: Security Without Collision-Resistance},
  author = {Mihir Bellare},
  year = {2006},
  journal = {Lecture notes in computer science},
  doi = {10.1007/11818175_36},
  url = {https://openalex.org/W2134615993},
}

@article{bellare2006c,
  title = {New Proofs for NMAC and HMAC: Security Without Collision-Resistance.},
  author = {Mihir Bellare},
  year = {2006},
  journal = {IACR Cryptol. ePrint Arch.},
  url = {https://openalex.org/W3031430546},
}

@article{beringer2015,
  title = {Verified correctness and security of OpenSSL HMAC},
  author = {Lennart Beringer and Adam Petcher and Katherine Q. Ye and Andrew W. Appel},
  year = {2015},
  url = {https://openalex.org/W2181293862},
}

@article{dodis2012,
  title = {To Hash or Not to Hash Again? (In)Differentiability Results for \$\$H^2\$\$ and HMAC},
  author = {Yevgeniy Dodis and Thomas Ristenpart and John Steinberger and Stefano Tessaro},
  year = {2012},
  journal = {Lecture notes in computer science},
  doi = {10.1007/978-3-642-32009-5_21},
  url = {https://openalex.org/W105372217},
}

@article{gaži2014,
  title = {The Exact PRF-Security of NMAC and HMAC},
  author = {Peter Gaži and Krzysztof Pietrzak and Michal Rybár},
  year = {2014},
  journal = {Lecture notes in computer science},
  doi = {10.1007/978-3-662-44371-2_7},
  url = {https://openalex.org/W2113782372},
}

@article{koblitz2013,
  title = {Another look at HMAC},
  author = {Neal Koblitz and Alfred Menezes},
  year = {2013},
  journal = {Journal of Mathematical Cryptology},
  doi = {10.1515/jmc-2013-5004},
  url = {https://openalex.org/W2064190686},
}

@article{fouque2008,
  title = {HMAC is a randomness extractor and applications to TLS},
  author = {Pierre-Alain Fouque and David Pointcheval and Sébastien Zimmer},
  year = {2008},
  doi = {10.1145/1368310.1368317},
  url = {https://openalex.org/W1995952728},
}

@article{song2017,
  title = {Quantum Security of NMAC and Related Constructions},
  author = {Fang Song and Aaram Yun},
  year = {2017},
  journal = {Lecture notes in computer science},
  doi = {10.1007/978-3-319-63715-0_10},
  url = {https://openalex.org/W2733208076},
}

@article{yasuda2009,
  title = {HMAC without the “Second” Key},
  author = {Kan Yasuda},
  year = {2009},
  journal = {Lecture notes in computer science},
  doi = {10.1007/978-3-642-04474-8_35},
  url = {https://openalex.org/W116634836},
}

@article{bhaumik2025,
  title = {Cryptographic Treatment of Key Control Security In Light of NIST SP 800-108},
  author = {Bhaumik, Ritam and Dutta, Avijit and Inoue, Akiko and Iwata, Tetsu and Jha, Ashwin and Minematsu, Kazuhiko and Nandi, Mridul and Sasaki, Yu and Turan, Meltem Sonmez and Tessaro, Stefano},
  year = {2025},
  journal = {ADVANCES IN CRYPTOLOGY-CRYPTO 2025, PT V},
  doi = {10.1007/978-3-032-01901-1_12},
  url = {https://www.webofscience.com/wos/woscc/full-record/WOS:001588004900012},
}

@article{choi2021,
  title = {Optimization of PBKDF2 Using HMAC-SHA2 and HMAC-LSH Families in CPU Environment},
  author = {Hojin Choi and Seog Chung Seo},
  year = {2021},
  journal = {IEEE Access},
  doi = {10.1109/ACCESS.2021.3065082},
  url = {https://ieeexplore.ieee.org/document/9374457/},
}

@article{gupta2016,
  title = {A replay-attack resistant message authentication scheme using time-based keying hash functions and unique message identifiers},
  author = {Boudhayan Gupta},
  year = {2016},
  journal = {cs.CR},
  doi = {10.48550/arXiv.1602.02148},
  url = {http://arxiv.org/abs/1602.02148v1},
}

@article{chen2025,
  title = {Deterministic Random Bit Generators Based on Ascon for Embedded Systems},
  author = {Abel C. H. Chen},
  year = {2025},
  journal = {cs.CR},
  doi = {10.48550/arXiv.2512.02082},
  url = {http://arxiv.org/abs/2512.02082v1},
}
\end{filecontents*}

\documentclass{article}
\usepackage[utf8]{inputenc}
\usepackage[colorlinks=true,allcolors=blue]{hyperref}
\usepackage{enumitem}
\usepackage{longtable}
\usepackage{booktabs}
\usepackage{array}
\usepackage[backend=biber,style=authoryear]{biblatex}
\addbibresource{synthesis.bib}

\title{HMAC security}
\date{2026-01-19}

\begin{document}
\maketitle

\begin{abstract}
This report synthesizes recent theoretical and practical research to address the question: Why is HMAC secure? The literature reveals that HMAC’s security rests not on the often-assumed collision resistance of its underlying hash function, but on more fundamental properties such as pseudorandom function (PRF) behavior and structural soundness, supported by rigorous proofs and formal verification techniques. Advances in analysis have extended from establishing strong guarantees in both classical and quantum adversarial models to demonstrating the efficacy of HMAC in real-world cryptographic protocols and deterministic random bit generators. Machine-checked proofs and verified implementations bolster confidence that HMAC's security extends to its practical use cases, including resource-constrained environments. Furthermore, research has challenged previously held limits, such as the birthday bound, and clarified that achievable assumptions about the compression function are sufficient for robust security. While questions about tightness of reductions and implementation generality remain, the convergence of provable pseudorandomness, structural resilience, and verified correctness underpins HMAC’s position as a versatile and well-substantiated cryptographic primitive.
\end{abstract}
\clearpage
\clearpage
\tableofcontents
\clearpage

\section{Research Question}

\subsection{Research Context}
Why is HMAC secure?

\subsection{Search Parameters}

\begin{description}
\item[Base query] Why is HMAC secure?
\item[Providers] s2, wos, openalex, ieee, arxiv, dblp
\end{description}

\subsection{Provider-specific queries}
\begin{longtable}{lp{10cm}}
\toprule
Provider & Query \\
\midrule
s2 & HMAC security \\
openalex & HMAC AND security proof \\
dblp & HMAC security proof \\
wos & TS=(HMAC AND security) \\
ieee & HMAC AND security \\
arxiv & HMAC AND security \\
\bottomrule
\end{longtable}

\subsection{Summary}

\begin{description}
\item[Date] 2026-01-19 09:57
\item[Total Papers] 91
\item[Kept] 18
\item[Discarded] 73
\item[Pending] 0
\end{description}


\section{Synthesis}

\subsection{Theoretical and Practical Analyses of HMAC Security}
The body of work on HMAC analysis reveals a comprehensive evolution from initial heuristic assessments to rigorous, formal security proofs. \textcite{bellare2006}, along with its variants \parencite{bellare2006b, bellare2006c}, fundamentally shifted the theoretical landscape by proving that HMAC's security can be established without requiring the underlying hash function to be collision-resistant. These results demonstrate that HMAC achieves strong pseudorandom function (PRF) security under weaker, more practical assumptions, such as the hash function's pseudorandomness and resistance to certain types of differential attacks. This decoupling from collision resistance expands the trusted base of HMAC to real-world hash functions that may be susceptible to collisions but otherwise behave "well" in a cryptographic sense.

Expanding on this, \textcite{gaži2014} provide precise bounds for the PRF-security of HMAC and its related construction, NMAC, identifying tight relationships between the security of the MAC and the underlying hash’s characteristics. This granularity offers practitioners clear security margins for real-world parameter choices. \textcite{koblitz2013} critically review the claims around HMAC's security and reinforce the importance of understanding the specific properties of the hash functions in use. Meanwhile, \textcite{yasuda2007} explores HMAC’s resilience beyond the conventional birthday bound, introducing multilane HMAC constructions that provide enhanced security for high-throughput applications.

The functional correctness and practical soundness of widely deployed HMAC implementations are underscored by formal verification approaches. \textcite{beringer2015} use proof assistants to verify the correctness and provable security of the OpenSSL HMAC implementation, bridging the gap between theory and deployment. On the applied side, HMAC's robustness as a pseudorandom function underpins its critical role in applications such as password-based key derivation, where \textcite{choi2021} analyze its instantiation in PBKDF2 using modern hash functions for both efficiency and security.

The security of HMAC in advanced targeted applications, such as key derivation and randomness extraction, is further examined in works like \textcite{bhaumik2025} and \textcite{fouque2008}. \textcite{bhaumik2025} confirm that HMAC-based KDFs used in NIST SP 800-108 standards achieve strong PRF and key control security, critical for generating multiple independent cryptographic keys. \textcite{fouque2008} show HMAC’s capability as a randomness extractor, crucial for secure session establishment in protocols like TLS.

Complementing these analyses, \textcite{gupta2016} offers refinements by extending HMAC constructions for time-bound authentication, demonstrating the continued evolution and customization of HMAC security for novel adversarial models. Collectively, these works reveal that HMAC's security foundation is robust and multifaceted, owing as much to refined theoretical models as to real-world validation and ongoing innovation in cryptographic engineering.

\subsection{Foundations and Advances in HMAC Theoretical Security}
The theoretical security of HMAC has been rigorously analyzed via reductions to its underlying components, including the hash function's properties and the structure of the keyed construction. \textcite{shen2025} provide a tight generic analysis of HMAC as a pseudorandom function (PRF), demonstrating that its security can be closely tied to the PRF strength of the underlying compression function, and presenting explicit tight bounds that improve upon earlier non-tight analyses. This supports the view that HMAC reliably amplifies the security properties of its underlying building blocks.

Complementing these classical analyses, investigations into HMAC's quantum resilience have become prominent. Both \textcite{hosoyamada2021} and \textcite{song2017} examine HMAC and its variant NMAC in the quantum random oracle model, focusing on whether the classical guarantees extend to adversaries with quantum capabilities. Their results provide tight bounds and highlight subtle issues in security reductions that are unique to the quantum context, confirming that, with proper assumptions, HMAC remains robust even against quantum attackers.

The security of HMAC also hinges on its structural design, particularly its iterated application of the compression function and key usage. \textcite{dodis2012} analyze the indifferentiability of HMAC and related double-iterated hash constructions, finding that while some variants fail to achieve strong indifferentiability, the two-key structure of HMAC maintains theoretical soundness as a MAC and PRF under standard cryptographic assumptions. Further, \textcite{yasuda2009} explores modifications of HMAC that use only a single key, theoretically investigating whether the second key is essential to HMAC's security guarantees; their analysis reveals that the original dual-key structure is critical for certain security properties under general assumptions.

Taken together, these works establish that HMAC's theoretical security is rooted in tight reductions to the underlying primitive's strength, careful structural choices (especially the double-keying and dual-hashing), and their resilience carries over to quantum settings, provided the appropriate assumptions hold \parencite{shen2025, hosoyamada2021, song2017, dodis2012, yasuda2009}.

\subsection{Verified and Practical Security of HMAC-DRBG}
Within the context of deterministic random bit generation, HMAC-DRBG stands out due to its blend of theoretical security and practical deployability. The functional correctness and cryptographic security of HMAC-DRBG, as specified by NIST SP 800-90A, have been formally established via game-based proofs demonstrating the pseudorandomness of its output, contingent on the security of HMAC as a pseudorandom function \parencite{ye2017}. Critically, \textcite{ye2017} constructed a machine-checked, end-to-end proof connecting the formal specification of HMAC-DRBG to its concrete implementation in mbedTLS, even composing the proof through the compiler to the binary level. This yields high-assurance guarantees that the deployed HMAC-DRBG produces outputs indistinguishable from random, assuming correct implementation of HMAC.

The foundational premise—that HMAC serves as an effective randomness extractor and thus a secure building block for DRBGs—was previously justified by demonstrative analyses of the HMAC construction and its extractor properties \parencite{fouque2008}. Such analytical foundations motivate its continued adoption in standard DRBG designs.

When transferring HMAC-DRBG concepts to constrained embedded systems, efficiency becomes paramount. Recent efforts have explored the integration of lightweight cryptographic primitives such as Ascon into the keying mechanisms of DRBGs while preserving the security structure of HMAC-DRBG. \textcite{chen2025} propose Ascon-based keyed hash DRBGs tailored for embedded contexts, demonstrating that these constructions improve on computational efficiency and memory footprint while following the security paradigm exemplified by HMAC-DRBG. Although the underlying primitives are adapted, this work continues to leverage the extract-then-expand strategy foundational to HMAC-DRBG, highlighting its enduring relevance and adaptability.

In sum, the security of HMAC-DRBG is affirmed both by rigorous, machine-checked proof methodologies and by practical adjustments that maintain its theoretical guarantees in new, resource-limited settings \parencite{ye2017, fouque2008, chen2025}.

\subsection{HMAC as a Secure Pseudorandom Function: Foundations and Analysis}
The security of HMAC as a pseudorandom function (PRF) has been studied extensively, with recent works delivering increasingly precise guarantees based on rigorous analyses. Notably, \textcite{gaži2014} provide an exact security analysis of HMAC and NMAC in the PRF setting, showing that the PRF advantage of an adversary is tightly correlated to the underlying hash function's pseudorandomness, the quality of its compression function, and the number of queries. Their work establishes that, up to reasonable bounds, HMAC retains PRF security as long as the hash function itself resists generic distinguishers, solidifying the notion of HMAC as a reliable PRF in theoretical and applied settings.

Building on this, \textcite{shen2025} further investigate PRF security for both HMAC and NMAC, giving tight, generic bounds for adversary advantage in terms of the hash function’s own security properties. Their analysis confirms and sharpens the conclusions of \textcite{gaži2014}, formally characterizing the negligible advantage of an attacker under standard assumptions, especially in the absence of weaknesses in the underlying primitive. Taken together, these results provide strong evidence that the structure of HMAC inherently preserves and, in some cases, amplifies the PRF properties of the underlying hash function.

In the context of cryptographic applications, the necessity of the PRF property is particularly evident in key derivation functions (KDFs), as examined by \textcite{bhaumik2025}. Their work reaffirms the core role of PRF security in KDF design, with a focus on the KDFs specified by NIST SP 800-108, which can be safely instantiated from HMAC when tight PRF guarantees are required. Through security proofs in the random oracle model, they demonstrate that such HMAC-based constructions achieve practical and strong pseudorandomness guarantees, further underlying the practical significance of the theoretical results established by \textcite{gaži2014} and \textcite{shen2025}.

Finally, applied analyses such as \textcite{choi2021} emphasize the use of HMAC as a secure PRF within real-world constructions like PBKDF2, demonstrating how theoretical PRF security extends to optimized, deployed systems. Even though their focus is largely on performance and entropy, the secure use of HMAC as a PRF remains foundational, enabled by the assurances proven in the aforementioned theoretical studies.

Collectively, these works converge on the conclusion that HMAC achieves robust PRF security when instantiated over a secure hash function—a property that both justifies and underpins its widespread adoption in security protocols, KDFs, and broader cryptographic infrastructure.

\subsection{Extending HMAC Security Beyond the Birthday Bound}
The security of HMAC has long been intertwined with the so-called birthday bound, which asserts that after a number of queries on the order of $2^{n/2}$ (where $n$ is the hash function output length), the advantage of an adversary rises sharply, creating a practical limit on the guarantees HMAC can provide. This bound is rooted in the generalized birthday problem, which predicts the probability of collisions among outputs, a fundamental concern for hash-based constructs. \textcite{yasuda2007} critically evaluates whether this birthday barrier is an inherent limitation of HMAC or merely a property of standard single-lane constructions.

Yasuda introduces "multilane HMAC," a generalized approach that structurally extends HMAC so as to operate over multiple parallel lanes. The multilane construction is shown to resist attacks that would typically exploit hash collisions enabled by the birthday bound, thereby enabling security levels beyond $2^{n/2}$ queries, potentially approaching the ideal $2^n$ bound proportional to the hash length. This demonstrates that the birthday bound is not an intrinsic ceiling for HMAC security, but rather a constraint that can be mitigated through careful design choices, such as parallelism in message authentication computation. Therefore, the multilane variant provides evidence that birthday-bound limitations can be transcended, altering the landscape of provable security for HMAC constructions and informing the broader theoretical understanding of hash-based authentication.

\subsection{Security of HMAC Beyond Collision Resistance}
A central theme arising from the recent proofs by Bellare is that HMAC's security does not fundamentally depend on the collision resistance of the hash function it employs \parencite{bellare2006, bellare2006b, bellare2006c}. Traditionally, collision resistance—preventing an adversary from finding two distinct inputs mapping to the same output—was believed central to the security of HMAC. However, all three works challenge this premise by providing rigorous proofs in the standard model and in the idealized random oracle model that demonstrate HMAC's security can be based instead on the pseudo-randomness of the compression function, even in the absence of strong collision resistance. 

Across these works, Bellare reconceptualizes the relationship between hash function properties and message authentication security, emphasizing that the underlying requirements for HMAC are weaker than previously assumed. The proofs show that even if an adversary can find collisions in the hash function, this does not necessarily translate into a viable attack against HMAC, provided the inner workings of the hash function remain pseudo-random. This insight has significant implications, suggesting that HMAC remains secure even as practical attacks threaten the collision resistance of widely deployed hash functions. The convergence of arguments and models across the papers strengthens the conclusion that HMAC’s security foundation transcends collision resistance, and instead resides in subtler cryptographic properties that should guide evaluation and selection of hash functions for practical deployments \parencite{bellare2006, bellare2006b, bellare2006c}.

\subsection{Formal Verification and Functional Security Proofs for HMAC}
Both \textcite{ye2017} and \textcite{beringer2015} address the security of HMAC by providing functional security proofs rooted in formal verification. \textcite{ye2017} demonstrates a modular approach, formulating a functional specification of HMAC-DRBG and establishing its pseudorandomness property using a hybrid game-based proof, all verified within the Coq proof assistant. Crucially, this proof is not isolated to the mathematical specification: it composes with formal verification of the C implementation and a verified compiler, guaranteeing that the final executable code maintains the cryptographic security properties specified in the abstraction. Similarly, \textcite{beringer2015} proves the correctness and security of the OpenSSL HMAC implementation, leveraging formal, machine-checked methodologies to establish both functional correctness and resistance to known attacks. The synergy between these works lies in their commitment to end-to-end assurance, from specification through real-world code, demonstrating that HMAC's security can be functionally and pragmatically validated through compositional, mathematically rigorous proofs.

\subsection{Fully Verified, Game-Based Approaches to HMAC Provable Security}
Provable security for HMAC centers on rigorous, formalized arguments that link the cryptographic strength of the construction to well-studied assumptions under precisely defined attack models. Both \textcite{beringer2015} and \textcite{ye2017} employ functional security proofs within a provable security paradigm, specifically using hybrid game-based reasoning. In these approaches, the security of HMAC and its constructions, such as HMAC-DRBG, is established by gradually transitioning from the real protocol to an idealized model through a series of small, quantifiably safe “hybrid” steps, each justified within a formal logic framework.

A key contribution across these works is the use of machine-checked proofs: \textcite{ye2017} demonstrates not only that the pseudorandomness of HMAC-DRBG can be proved via a hybrid argument but also that this security property persists through the stack of implementation artifacts, from functional specification to compiled code. This compositional proof strategy, integrated and checked in Coq, ensures no gap exists between high-level cryptographic theory and deployed software, thus providing an end-to-end security guarantee. Likewise, \textcite{beringer2015} emphasizes the value of functional verification and formal logic in underpinning the provable security claims of HMAC as realized in practical cryptographic libraries. Collectively, these results exemplify how the security of HMAC is not merely an artifact of heuristic design or informal arguments, but is anchored in systematically verified, modular proofs that stand up to scrutiny at all abstraction levels.

\subsection{Quantum Security Guarantees for HMAC in the Quantum Random Oracle Model}
The security of HMAC in the presence of quantum attackers has been rigorously explored within the quantum random oracle model (QROM), with multiple works providing clarity on both its foundational robustness and the subtleties of quantum security reductions. \textcite{song2017} established the groundwork for quantum security of NMAC, which is the foundation for the HMAC construction, arguing that its security properties extend to settings where adversaries possess quantum query access to the underlying random function. They developed novel proof techniques to handle the superposition queries intrinsic to quantum adversaries, ultimately showing that HMAC’s security in the QROM degrades only modestly compared to the classical setting, and is fundamentally tied to the collision resistance of the underlying hash function.

Building upon these results, \textcite{hosoyamada2021} investigated not only the security but also the tightness of HMAC and NMAC proofs in the QROM. Their analysis demonstrates that, while HMAC remains quantum-secure, there exist inherent limitations in achieving tight reductions under quantum attacks. Specifically, they provide tight bounds for the security loss incurred in the reduction, quantifying exactly how adversarial quantum queries impact overall security. The paper highlights practical considerations in the instantiation of HMAC, underscoring that both construction choice and reduction analysis are essential for maintaining robust quantum security. Taken together, these works show that while HMAC possesses durable security guarantees even in the quantum setting, careful attention must be given to the nuances of quantum security reductions and model assumptions \parencite{song2017,hosoyamada2021}.

\subsection{Foundations and Advances in the Theoretical Security of HMAC}
The theoretical security of HMAC has been grounded in rigorous models and continually refined through deeper analysis. Early work by \textcite{bellare2006, bellare2006b, bellare2006c} challenged the then-prevailing belief that collision resistance of the underlying hash function was necessary for HMAC’s security. Instead, they provided proofs showing that the pseudorandomness and unforgeability of HMAC can be established under much weaker assumptions, specifically treating the compression function as a pseudorandom function (PRF) and not relying directly on collision resistance. This shift greatly broadened the applicability and robustness of HMAC, establishing a new standard for its theoretical security.

Building on these foundations, \textcite{gaži2014} delivered exact bounds for HMAC and NMAC’s security as PRFs, thereby tightening the security guarantees and clarifying their limitations within precise adversarial models. These analyses model HMAC’s security in idealized settings—such as the random oracle and ideal compression function models—showing that as long as the underlying components behave as assumed, HMAC achieves strong PRF security.

The work of \textcite{dodis2012} further investigated the security of HMAC with respect to (in)differentiability, exploring to what extent HMAC approximates a random oracle or a PRF, and thus to what degree it can safely substitute for these ideal primitives in higher-level cryptographic protocols.

\textcite{yasuda2007} addressed the so-called “birthday bound,” showing via multilane HMAC constructions that HMAC’s security can extend beyond the traditional \(2^{n/2}\) birthday limitation in some settings, preventing attacks that exploit pairwise collisions when the domain is properly structured.

Additional analysis from \textcite{koblitz2013} offered critical perspectives on the underlying assumptions, arguing that practical deployment scenarios rarely encounter the pathologies that would threaten HMAC’s theoretical security under well-accepted models. More recently, HMAC’s theoretical security assurances have underpinned its use in advanced cryptographic constructs. For key derivation, \textcite{bhaumik2025} leveraged rigorous proofs in the random oracle model to show that HMAC-based key derivation can achieve 128-bit security with strong pseudorandomness and key control properties. For deterministic random bit generation, \textcite{chen2025} relied on the strong theoretical guarantees of HMAC to design fast and secure DRBGs suited to embedded systems.

Collectively, these works demonstrate that HMAC's theoretical security is not only well-founded based on modern cryptographic models, but has also been precisely quantified and effectively extended to meet evolving cryptographic challenges \parencite{bellare2006, bellare2006b, bellare2006c, gaži2014, dodis2012, yasuda2007, koblitz2013, bhaumik2025, chen2025}.

\section{Conclusion}

In responding to the question "Why is HMAC secure?", the literature reveals a multidimensional foundation for HMAC’s robustness grounded in both theoretical analysis and practical validation. Initial conceptions placed collision resistance of the underlying hash function at the center of HMAC’s security, yet a substantial body of work—including rigorous proofs and machine-checked implementations—demonstrates that HMAC’s security is more fundamentally linked to properties like pseudorandomness and pseudo-random function (PRF) behavior \parencite{bellare2006, gaži2014}. Across both classical and quantum settings, recent analyses establish tight security bounds with reductions primarily to the indistinguishability and structural features of the primitive components, especially its key separation and two-layer construction \parencite{shen2025, dodis2012, yasuda2009, song2017, hosoyamada2021}.

Functional and provable security approaches provide end-to-end guarantees by simultaneously verifying cryptographic design and actual code, strengthening confidence that real-world implementations reflect theoretical assurances \parencite{beringer2015, ye2017}. This robust approach is mirrored in domains such as HMAC-DRBG, where the extractor properties of HMAC have been formally shown to enable secure pseudorandom generation in both general and embedded-system contexts \parencite{ye2017, chen2025}. In practical cryptographic applications—including key derivation and authentication—HMAC’s demonstrable PRF security underpins its effective use and widespread deployment \parencite{bhaumik2025, chao2021}.

These findings also clarify that the birthday bound, while long considered a hard limit, is not inherent or immutable; structural modifications like multilane HMAC can surpass it in certain scenarios \parencite{yasuda2007}. Moreover, the body of literature is nearly uniform in asserting that HMAC’s security does not require the full collision resistance of the base hash function, shifting the focus toward the more achievable property of compression function pseudorandomness \parencite{bellare2006, bellare2006b, bellare2006c}.

Despite this progress, some limitations and open questions persist. The tightness of security reductions—particularly in quantum or idealized models—remains a subtle issue, as does the precise mapping between hash function properties and realized HMAC security guarantees \parencite{hosoyamada2021, song2017}. Further, while verified implementations give high assurance for certain standards (e.g., HMAC-DRBG), the generalization of these guarantees across diverse platforms and hash functions is an ongoing challenge \parencite{ye2017, beringer2015}. As cryptanalysis and cryptographic modeling evolve, continued scrutiny of HMAC’s assumptions and their translation into practical, side channel–resistant code is warranted.

In summary, HMAC derives its security from a rigorous synergy of provable pseudorandomness, sound structural design, and formal verification bridging theory and practice. This blend of properties and proofs ensures that, under widely accepted and relatively modest assumptions about the underlying hash function, HMAC serves as a compelling, versatile, and well-substantiated cryptographic primitive.

\printbibliography
\end{document}
