\begin{filecontents*}{synthesis.bib}
@article{shen2025,
  title = {Tight Generic PRF Security of HMAC and NMAC},
  author = {Yaobin Shen and Xiangyang Zhang and Lei Wang and Dawu Gu},
  year = {2025},
  journal = {IACR Cryptology ePrint Archive},
  url = {https://www.semanticscholar.org/paper/0b676b3075484b7db7b6741bc589ce02e4bdb407},
}

@article{bellare2006,
  title = {New Proofs for NMAC and HMAC: Security without Collision Resistance},
  author = {M. Bellare},
  year = {2006},
  journal = {Journal of Cryptology},
  doi = {10.1007/s00145-014-9185-x},
  url = {https://www.semanticscholar.org/paper/c00e5fd533c5207ad445dbeebe0c82f223ff2665},
}

@article{yasuda2007,
  title = {Multilane HMAC - Security beyond the Birthday Limit},
  author = {K. Yasuda},
  year = {2007},
  journal = {International Conference on Cryptology in India},
  doi = {10.1007/978-3-540-77026-8_3},
  url = {https://www.semanticscholar.org/paper/e5de88cdb12481736bf56067076ed51a4669ab86},
}

@article{hosoyamada2021,
  title = {On Tight Quantum Security of HMAC and NMAC in the Quantum Random Oracle Model},
  author = {Akinori Hosoyamada and Tetsu Iwata},
  year = {2021},
  journal = {IACR Cryptology ePrint Archive},
  doi = {10.1007/978-3-030-84242-0_21},
  url = {https://www.semanticscholar.org/paper/ea19e5f9176d0de71d3ecfbb3d6b4134b54ae8b0},
}

@article{ye2017,
  title = {Verified Correctness and Security of mbedTLS HMAC-DRBG},
  author = {Katherine Q. Ye and M. Green and Naphat Sanguansin and Lennart Beringer and Adam Petcher and A. Appel},
  year = {2017},
  journal = {Conference on Computer and Communications Security},
  doi = {10.1145/3133956.3133974},
  url = {https://www.semanticscholar.org/paper/46906901407bac5ab595b55b05d661db5a798255},
}

@article{bellare2006b,
  title = {New Proofs for NMAC and HMAC: Security Without Collision-Resistance},
  author = {Mihir Bellare},
  year = {2006},
  journal = {Lecture notes in computer science},
  doi = {10.1007/11818175_36},
  url = {https://openalex.org/W2134615993},
}

@article{bellare2006c,
  title = {New Proofs for NMAC and HMAC: Security Without Collision-Resistance.},
  author = {Mihir Bellare},
  year = {2006},
  journal = {IACR Cryptol. ePrint Arch.},
  url = {https://openalex.org/W3031430546},
}

@article{beringer2015,
  title = {Verified correctness and security of OpenSSL HMAC},
  author = {Lennart Beringer and Adam Petcher and Katherine Q. Ye and Andrew W. Appel},
  year = {2015},
  url = {https://openalex.org/W2181293862},
}

@article{dodis2012,
  title = {To Hash or Not to Hash Again? (In)Differentiability Results for \$\$H^2\$\$ and HMAC},
  author = {Yevgeniy Dodis and Thomas Ristenpart and John Steinberger and Stefano Tessaro},
  year = {2012},
  journal = {Lecture notes in computer science},
  doi = {10.1007/978-3-642-32009-5_21},
  url = {https://openalex.org/W105372217},
}

@article{gaži2014,
  title = {The Exact PRF-Security of NMAC and HMAC},
  author = {Peter Gaži and Krzysztof Pietrzak and Michal Rybár},
  year = {2014},
  journal = {Lecture notes in computer science},
  doi = {10.1007/978-3-662-44371-2_7},
  url = {https://openalex.org/W2113782372},
}

@article{koblitz2013,
  title = {Another look at HMAC},
  author = {Neal Koblitz and Alfred Menezes},
  year = {2013},
  journal = {Journal of Mathematical Cryptology},
  doi = {10.1515/jmc-2013-5004},
  url = {https://openalex.org/W2064190686},
}

@article{fouque2008,
  title = {HMAC is a randomness extractor and applications to TLS},
  author = {Pierre-Alain Fouque and David Pointcheval and Sébastien Zimmer},
  year = {2008},
  doi = {10.1145/1368310.1368317},
  url = {https://openalex.org/W1995952728},
}

@article{song2017,
  title = {Quantum Security of NMAC and Related Constructions},
  author = {Fang Song and Aaram Yun},
  year = {2017},
  journal = {Lecture notes in computer science},
  doi = {10.1007/978-3-319-63715-0_10},
  url = {https://openalex.org/W2733208076},
}

@article{yasuda2009,
  title = {HMAC without the “Second” Key},
  author = {Kan Yasuda},
  year = {2009},
  journal = {Lecture notes in computer science},
  doi = {10.1007/978-3-642-04474-8_35},
  url = {https://openalex.org/W116634836},
}

@article{bhaumik2025,
  title = {Cryptographic Treatment of Key Control Security In Light of NIST SP 800-108},
  author = {Bhaumik, Ritam and Dutta, Avijit and Inoue, Akiko and Iwata, Tetsu and Jha, Ashwin and Minematsu, Kazuhiko and Nandi, Mridul and Sasaki, Yu and Turan, Meltem Sonmez and Tessaro, Stefano},
  year = {2025},
  journal = {ADVANCES IN CRYPTOLOGY-CRYPTO 2025, PT V},
  doi = {10.1007/978-3-032-01901-1_12},
  url = {https://www.webofscience.com/wos/woscc/full-record/WOS:001588004900012},
}

@article{choi2021,
  title = {Optimization of PBKDF2 Using HMAC-SHA2 and HMAC-LSH Families in CPU Environment},
  author = {Hojin Choi and Seog Chung Seo},
  year = {2021},
  journal = {IEEE Access},
  doi = {10.1109/ACCESS.2021.3065082},
  url = {https://ieeexplore.ieee.org/document/9374457/},
}

@article{gupta2016,
  title = {A replay-attack resistant message authentication scheme using time-based keying hash functions and unique message identifiers},
  author = {Boudhayan Gupta},
  year = {2016},
  journal = {cs.CR},
  doi = {10.48550/arXiv.1602.02148},
  url = {http://arxiv.org/abs/1602.02148v1},
}

@article{chen2025,
  title = {Deterministic Random Bit Generators Based on Ascon for Embedded Systems},
  author = {Abel C. H. Chen},
  year = {2025},
  journal = {cs.CR},
  doi = {10.48550/arXiv.2512.02082},
  url = {http://arxiv.org/abs/2512.02082v1},
}
\end{filecontents*}

\documentclass{article}
\usepackage[utf8]{inputenc}
\usepackage[colorlinks=true,allcolors=blue]{hyperref}
\usepackage{enumitem}
\usepackage{longtable}
\usepackage{booktabs}
\usepackage{array}
\usepackage[backend=biber,style=authoryear]{biblatex}
\addbibresource{synthesis.bib}

\title{HMAC security}
\date{2026-01-19}

\begin{document}
\maketitle

\begin{abstract}
This report synthesizes foundational and modern scholarship to answer the question: why is HMAC secure? It traces HMAC’s security to rigorous theoretical reductions showing that, when built on a well-designed hash function, HMAC delivers strong pseudorandom function (PRF) guarantees and remains robust even under practical and quantum attack models. The analysis encompasses collision resistance, showing that HMAC’s design provides security benefits independent of hash collisions, and addresses real-world aspects, including provable security for deployed protocols and verified implementations. Developments such as HMAC-DRBG and multilane variants demonstrate HMAC’s adaptability and sustained security in both resource-constrained environments and beyond classical security limits like the birthday bound. Mechanized and formal verification advances further validate HMAC’s trustworthy operation at both the design and implementation levels. Although certain nuanced challenges—such as (in)differentiability and quantum adversary scenarios—invite continued research, the report concludes that HMAC’s layered assurance across proofs, practice, and evolving adversarial models firmly underpins its status as a secure and reliable cryptographic primitive.
\end{abstract}
\clearpage
\clearpage
\tableofcontents
\clearpage

\section{Research Question}

\subsection{Research Context}
Why is HMAC secure?

\subsection{Search Parameters}

\begin{description}
\item[Base query] Why is HMAC secure?
\item[Providers] s2, wos, openalex, ieee, arxiv, dblp
\end{description}

\subsection{Provider-specific queries}
\begin{longtable}{lp{10cm}}
\toprule
Provider & Query \\
\midrule
s2 & HMAC security \\
openalex & HMAC AND security proof \\
dblp & HMAC security proof \\
wos & TS=(HMAC AND security) \\
ieee & HMAC AND security \\
arxiv & HMAC AND security \\
\bottomrule
\end{longtable}

\subsection{Summary}

\begin{description}
\item[Date] 2026-01-19 09:57
\item[Total Papers] 91
\item[Kept] 18
\item[Discarded] 73
\item[Pending] 0
\end{description}


\section{Synthesis}

\subsection{Security Foundations and Theoretical Advances in HMAC Analysis}
Rigorous analysis of HMAC has anchored its reputation as a secure keyed hash-based message authentication code, with security rooted in both theoretical reductions and practical verifications. Early influential work by \textcite{bellare2006,bellare2006b,bellare2006c} demonstrated that HMAC's security does not fundamentally rely on the strong collision resistance of the underlying hash, but can instead be reduced to the hash function's resistance to specific classes of attacks, such as the inability to find second-preimage and related-key collisions. This shift enabled proofs under weaker assumptions, greatly expanding the range of suitable hash functions for HMAC.

More recently, precise security bounds for HMAC as a pseudorandom function (PRF) have been established. Notably, \textcite{gaži2014} provided a tight characterization of HMAC's PRF security in the standard model, quantifying how HMAC's security relates to that of the underlying hash. This nuanced view is further reinforced in applications such as key derivation, where HMAC is often employed to instantiate KDFs. \textcite{bhaumik2025} formalize new security properties like key control security and present provable guarantees—via random oracle proofs—for HMAC-based KDFs, showing that these constructions meet the robust 128-bit security expected in modern standards.

HMAC's resilience against classical attacks is also highlighted in analysis of its operation relative to the birthday bound: \textcite{yasuda2007} explores multilane HMAC variants designed to push security well beyond the limitations typically imposed by birthday attacks, implying that proper instantiation can mitigate otherwise limiting hash function properties.

Theoretical assurances are bolstered by practical validation. For example, \textcite{beringer2015} deliver a formal, machine-verified proof of the functional correctness and security properties of OpenSSL’s HMAC implementation, bridging the gap between cryptographic design and real-world deployment. Meanwhile, systemic analyses (\textcite{koblitz2013,gupta2016}) provide critical perspectives on HMAC’s strengths and practical limitations, including its use in advanced authentication schemes and its behavior under adversarial exploitation. Furthermore, studies such as \textcite{fouque2008} extend HMAC's utility into randomness extraction and key derivation applications, reinforcing its versatility and affirming the robustness of its core structure.

Overall, the literature reveals a consistent trend: HMAC's security is secured by a combination of strong theoretical foundations—especially reduction-based proofs and PRF analyses—and practical validation, even as it is refined through new applications and adversarial models.

\subsection{Foundations and Boundaries of HMAC Security: Classical and Quantum Analyses}
Theoretical analyses of HMAC converge on its security as a pseudorandom function (PRF), with recent work emphasizing the importance of tight security reductions and specific model assumptions. \textcite{shen2025} present a tight generic PRF security proof for HMAC, showing that its security can be tightly reduced to that of the underlying compression function. This result improves the concrete security guarantees of HMAC, reducing the security loss compared to earlier, looser analyses, and confirming the robustness of its construction in the standard model.

A complementary perspective is provided by the (in)differentiability framework, where the question of whether HMAC achieves indifferentiability from a random oracle is explored. \textcite{dodis2012} analyze HMAC as a form of nested hash, and demonstrate that HMAC's construction does not always inherit the indifferentiability guarantees of its components, particularly depending on the compression function and specific hashing strategies used. This nuanced view highlights that HMAC's security as a PRF does not necessarily translate to random oracle indifferentiability, and motivates careful consideration of underlying primitives.

Attention has also been paid to the role of the HMAC key schedule. \textcite{yasuda2009} investigate the necessity of HMAC's "second" key, showing that the use of two independent keys is not always essential for preserving security. By analyzing simplified variants, Yasuda develops theoretical grounds for reliable, potentially more efficient instantiations, further clarifying the relationship between key structure and overall security.

The emergence of quantum adversaries has driven renewed analysis of HMAC in the quantum-accessible random oracle model. Both \textcite{hosoyamada2021} and \textcite{song2017} provide rigorous quantum security analyses of HMAC and related constructions. They identify the conditions under which HMAC preserves its PRF security in the presence of quantum query attacks, and refine the concrete bounds on its advantage. These works highlight both the resilience and the subtle quantum limitations of HMAC, delineating scenarios where classical security guarantees extend (or fail to extend) to the quantum setting.

Taken together, these studies establish a nuanced theoretical foundation for HMAC's security, characterizing its strengths as a tightly secure PRF, clarifying the limitations that arise from its structure and component choices, and outlining the evolving landscape of security guarantees as the threat model shifts from classical to quantum.

\subsection{Formal Security and Efficiency of HMAC-DRBG}
Research on HMAC-DRBG has advanced in two major directions: formal security assurance and practical adaptability. \textcite{ye2017} have delivered a comprehensive formalization and machine-checked cryptographic proof for HMAC-DRBG, precisely specifying its functional requirements and demonstrating, via a hybrid game-based argument, that its outputs are pseudorandom under the standard HMAC assumption. Notably, the proof extends to verified C implementations (such as in mbedTLS) and even through to compiled machine code, establishing a robust end-to-end guarantee that practical deployments adhere to the strong pseudorandomness property. This work sets a high standard for assurance, providing a provably secure and referenceable functional specification applicable to arbitrary conforming implementations.

Complementing these formal results, research has also examined the underlying role of HMAC as a high-entropy extractor and its strengths when leveraged inside DRBG constructions. \textcite{fouque2008} analyze the extractor properties of HMAC and their direct application to DRBGs, supporting the claim that HMAC-DRBG maintains strong security guarantees as long as HMAC itself behaves as a secure pseudorandom function. These insights underpin the theoretical soundness of HMAC-DRBG's standard usage and widen its applicability to protocols like TLS.

Recent developments target efficiency in resource-constrained environments, such as embedded systems. \textcite{chen2025} extend the HMAC-DRBG paradigm by proposing Ascon-driven HMAC-DRBG variants optimized for such settings. Through empirical evaluation on platforms like Raspberry Pi, they show that these new constructions retain the security characteristics of standard HMAC-DRBG while achieving improvements in computational and memory efficiency, demonstrating the roadmap for secure and performant random bit generation in constrained contexts.

Taken together, these contributions show that HMAC-DRBG stands on a strong theoretical and practical foundation. Its security is well-understood through formal proofs, and it adapts efficiently to diverse implementation requirements while maintaining its core security properties.

\subsection{Foundations and Advances in the Pseudorandom Function Security of HMAC}
The security of HMAC as a pseudorandom function (PRF) forms one of its central cryptographic justifications. Building on earlier work, recent research has focused on formalizing and tightening PRF security proofs for HMAC, analyzing its behavior both generically and within specific application contexts. 

\textcite{gaži2014} provide exact security bounds for HMAC and NMAC, showing that when built from a pseudorandom function or a random oracle, HMAC achieves strong PRF security properties. Their analysis introduces a framework for accurately quantifying the advantage of an adversary, thus addressing the gaps between classical security proofs and practical instantiation. This provides a theoretical foundation demonstrating that HMAC behaves indistinguishably from a random function under standard assumptions.

Expanding on this, \textcite{shen2025} prove tight generic PRF security bounds for HMAC and NMAC. Their results confirm that HMAC inherits the pseudorandomness of its underlying hash function in a tight and quantifiable way, further solidifying the confidence in HMAC's security in the generic-shape proof settings. This work is significant in that it addresses the tightness of the reduction—meaning that the proven security loss in transitioning from the hash function to HMAC is minimized, leading to more faithful estimates of real-world security.

From a practical perspective, \textcite{bhaumik2025} demonstrate that the strong PRF security of HMAC underpins its safe deployment in key derivation functions, as prescribed in NIST SP 800-108. Their proofs in the random oracle model confirm that HMAC-based constructions meet the stringent pseudorandomness requirements essential for key derivation, and by extension, for a wide array of applied cryptographic protocols where derived keys must remain indistinguishable from random. This work also explores the interplay between PRF and additional properties such as key control, but fundamentally relies on the well-established PRF security of HMAC.

Operationally, deployments such as PBKDF2 make direct use of HMAC in the role of a PRF, as explored by \textcite{choi2021}. Their studies of PBKDF2 instantiated with HMAC-SHA2 and HMAC-LSH confirm the suitability of HMAC as a PRF in password-based key derivation contexts, indicating no practical deviations from the expected pseudorandomness even as optimization and implementation choices are explored.

Across these works, a clear consensus emerges: HMAC's PRF security is robustly established under conventional cryptographic assumptions, with increasingly tight bounds and supporting evidence both from theory and practice \parencite{gaži2014, shen2025, bhaumik2025, choi2021}.

\subsection{Security of HMAC Beyond the Birthday Bound}
The birthday bound is a central limitation in the security analysis of cryptographic constructions based on hash functions, such as HMAC. Traditionally, the resistance of HMAC to generic forgery and collision attacks is considered limited to around $2^{n/2}$ queries due to the birthday phenomenon, where $n$ is the output length of the underlying hash function. This boundary arises because the probability of finding a collision—or two distinct inputs producing the same tag—by random chance becomes significant after roughly $2^{n/2}$ queries.

However, \textcite{yasuda2007} critically evaluates this restriction in the context of HMAC and explores whether it is an unavoidable limitation. The paper demonstrates that the standard HMAC design indeed succumbs to the birthday bound, confirming that security degrades sharply beyond this threshold. Yet, Yasuda introduces and analyzes ``multilane HMAC,'' a modified construction enabling security guarantees beyond what is typically expected from the birthday bound. By processing multiple independent lanes in parallel, multilane HMAC increases the effective collision complexity, potentially extending HMAC's resistance to attacks even as query counts approach or surpass $2^{n/2}$. This work suggests that, despite the traditional birthday-based security ceiling, it is possible to design variants of HMAC that achieve higher practical security levels by structurally restricting adversarial collision-finding strategies.

Overall, the analysis by \textcite{yasuda2007} reveals both the inherent limitations imposed by the birthday bound in standard HMAC and innovative approaches to exceeding them, thus enriching the theoretical foundation of HMAC's security.

\subsection{HMAC Security Beyond Collision Resistance}
The link between HMAC security and the collision resistance of its underlying hash function has long been viewed as fundamental; however, the trio of works by \textcite{bellare2006}, \textcite{bellare2006b}, and \textcite{bellare2006c} collectively revise this assumption. These analyses provide new proofs indicating that HMAC's core security properties—specifically, its resistance to forgery—do not hinge on the collision resistance of the base hash function. Instead, they show HMAC maintains its security even when the hash function allows collisions, provided that certain weaker properties (such as pseudorandomness and the absence of chosen-prefix collisions) hold.

A central insight across these papers is that, whereas collision resistance is essential for applications like digital signatures, HMAC's security as a message authentication code (MAC) primarily derives from its keyed structure and the unpredictability of its internal state to an adversary without the key. This theoretical shift, substantiated by the new proofs, decouples HMAC's security from the collision resistance requirement and roots it in assumptions more directly related to MAC security. The works collectively demonstrate that, even in scenarios where attackers can efficiently find hash collisions, HMAC resists existential forgery attacks as long as other critical properties of the compression function are maintained. Thus, the necessity of collision resistance for HMAC security is debunked and replaced with a more nuanced set of assumptions focused on pseudorandomness and secret-key security \parencite{bellare2006,bellare2006b,bellare2006c}.

\subsection{Formal and Machine-Checked Proofs of HMAC Functional Security}
Functional security proofs form a pivotal foundation for establishing the security of HMAC, bridging the gap between cryptographic specifications and actual implementations. Both \textcite{beringer2015} and \textcite{ye2017} exemplify a shift toward rigorous, machine-checked approaches that not only define what HMAC should compute, but systematically prove that its operation meets the desired cryptographic guarantees when correctly implemented. In particular, \textcite{beringer2015} present a detailed formalization and proof of the functional correctness and security of OpenSSL’s HMAC implementation, grounding its security guarantees in well-defined specifications. Building on similar principles, \textcite{ye2017} undertake a comprehensive, modular proof for HMAC-DRBG, using hybrid game-based arguments and verifying that the implementation (in mbedTLS) adheres exactly to a formal functional specification. Importantly, both works employ mechanized verification—most notably in the Coq proof assistant—which ensures that the proofs are both highly reliable and reproducible. These functional security proofs guarantee that the cryptographic properties (such as pseudorandomness) are preserved from the specification level down to real-world code, establishing HMAC’s security on robust, formally verified grounds.

\subsection{Machine-Checked Proofs and Modular Reasoning in HMAC Security}
Proving HMAC's security has advanced notably through the use of formal verification and provable security techniques. \textcite{beringer2015} establish a functional security proof for OpenSSL's HMAC implementation using machine-checked reasoning, demonstrating that the cryptographic specification is faithfully realized in the actual code. This approach strengthens conventional game-based arguments by enabling end-to-end assurance from the high-level security property to the concrete implementation. Building on similar foundations, \textcite{ye2017} utilize a modular, hybrid game-based proof to guarantee the pseudorandomness property for HMAC-DRBG, formalizing the proof in Coq and composing it with verified low-level software correctness. Both works highlight modularity as a key strength—once a security property is formally proven for a specification, any conforming implementation can inherit this guarantee. Furthermore, the composition of functional correctness and cryptographic proofs yields comprehensive, end-to-end assurance that the machine-level realization of HMAC maintains its theoretical security properties. Collectively, these results affirm the power of formal, machine-checked, and modular proofs in providing high-assurance, provable security for HMAC and its applications \parencite{beringer2015,ye2017}.

\subsection{Quantum Security Foundations of HMAC}
HMAC's security in the presence of quantum adversaries fundamentally relies on the robustness of its underlying cryptographic assumptions when extended to quantum models. Both \textcite{hosoyamada2021} and \textcite{song2017} investigate HMAC and closely related constructions (such as NMAC) in the quantum random oracle model (QROM), a setting that aims to capture the capabilities of quantum adversaries who can query oracles in superposition. 

\textcite{song2017} approach the question by developing analyses that extend classical proofs to take into account quantum query techniques. Their results suggest that, under reasonable assumptions, HMAC preserves a form of security even when adversaries exploit quantum parallelism to interact with the random oracle, though some security loss may occur compared to the classical setting. 

Building on this, \textcite{hosoyamada2021} provide tighter bounds for the quantum security of both HMAC and NMAC in the QROM. Their analysis refines earlier bounds, demonstrating that the gap between classical and quantum security can be minimized, provided that the underlying compression function retains strong pseudorandomness properties against quantum queries. Both works emphasize that the security of HMAC in the quantum setting is not inherent, but rather is contingent on the ability to translate classical security reductions to the quantum domain without incurring prohibitive loss.

Together, these analyses show that a substantial foundation exists for believing in the quantum-resilience of HMAC, contingent upon the instantiation of suitably robust cryptographic primitives and the careful adaptation of proofs to the quantum model \parencite{song2017,hosoyamada2021}.

\subsection{Formal Security Foundations and Recent Advances in HMAC Theory}
The theoretical security of HMAC is anchored in the formal analysis of its construction as a keyed hash function and its interpretation as a pseudorandom function (PRF). Groundbreaking work by \textcite{bellare2006, bellare2006b, bellare2006c} established that HMAC remains provably secure even if the underlying hash function is not collision resistant, provided it behaves as a pseudorandom function or is modeled as an ideal keyed function. This "beyond collision resistance" approach represented a significant conceptual advance, decoupling HMAC’s security from the stronger properties traditionally assumed for hash functions and allowing for security proofs in settings where collision resistance might be compromised.

Extending this line of inquiry, precise bounds for HMAC's PRF security were formalized by \textcite{gaži2014}, who demonstrated exact security guarantees under standard assumptions. These results clarify how HMAC’s layered use of the hash function and key separation directly contribute to its resistance against key-recovery and forgery attacks, quantifying the advantage of an adversary as a function of the number of queries and the hash function's inherent properties.

Exploring the security behaviour of HMAC under different settings, \textcite{yasuda2007} considered the impact of the birthday bound, showing that HMAC constructions can securely authenticate messages with strengths surpassing the traditional birthday attack limits. Additionally, the (in)differentiability analysis of HMAC by \textcite{dodis2012} contextualizes its idealized security: their results probe the subtle relationships between hash-based modes like HMAC and theoretical security metrics such as random oracle and ideal cipher models.

Further, \textcite{bhaumik2025} investigated HMAC’s security as a KDF, particularly in the context of key control (KC) security and pseudorandomness, providing rigorous proofs that HMAC instantiations meet modern cryptographic standards, such as the 128-bit security margin in NIST SP 800-108. These findings underline HMAC's versatility and the theoretical assurances supporting its deployment in broader cryptographic applications.

Complementary perspectives and critiques are offered in \textcite{koblitz2013} and \textcite{gupta2016}, both underscoring the theoretical soundness of HMAC while examining variations and modifications that preserve its security properties. While \textcite{gupta2016} particularly explores extensions to strengthen replay resistance, both works reaffirm the foundational principle that HMAC’s robustness stems from provable security reductions and established cryptanalytic practice.

Finally, the deployment of HMAC in resource-constrained environments, as discussed by \textcite{chen2025}, leverages its strong theoretical security to justify its selection in lightweight cryptographic constructions, further attesting to the solid mathematical assurances underlying its use.

Collectively, this literature substantiates that HMAC's theoretical security is founded on rigorous proofs, careful abstraction of hash functions, and continuous refinement through analysis and practical deployment experience. The consensus is that, assuming reasonable properties of the underlying hash and prudent key management, HMAC achieves strong, quantifiable security guarantees against a wide array of attacks.

\section{Conclusion}

In synthesis, the security of HMAC is robustly grounded in rigorous theoretical, practical, and mechanized-analysis traditions, providing a multifaceted answer to why HMAC is secure. Across the literature, there is consensus that HMAC's security fundamentally arises from properties of pseudorandomness and keyed unpredictability provided by its underlying hash function, rather than from hash collision resistance alone \parencite{bellare2006, gaži2014, bellare2006b, bellare2006c}. Formal reductionist proofs confirm that well-constructed hash functions imbue HMAC with strong PRF guarantees, making it resilient to classical attacks and underpinning its reliability as a cryptographic primitive \parencite{gaži2014, shen2025, bhaumik2025}. 

This theoretical assurance is reinforced by practical analyses and machine-checked security proofs, which validate HMAC's end-to-end security in concrete implementations and real-world protocols \parencite{beringer2015, ye2017, bhaumik2025}. Extensions to advanced domains such as HMAC-DRBG demonstrate that these security properties carry over even to compiled and resource-constrained environments \parencite{ye2017, chen2025}. Meanwhile, the birthday bound remains a universal constraint, with standard HMAC secure up to $2^{n/2}$ queries—although research into multilane variants shows promise for surpassing traditional collision and forgery limits \parencite{yasuda2007}. Notably, detailed analyses have demonstrated that HMAC remains secure in both classical and quantum random oracle models under appropriate assumptions, with recent work establishing increasingly tight quantum reductions \parencite{hosoyamada2021, song2017}.

Though the body of work is extensive and generally convergent, some nuanced limitations and open questions remain. The precise boundaries of HMAC's (in)differentiability as a random oracle and subtle quantum adversary models invite further study \parencite{dodis2012, hosoyamada2021}. Additionally, while provable security results are strong for contemporary hash functions, the continual evolution of cryptanalytic techniques and the prospect of new quantum capabilities suggest that ongoing formal analysis remains essential \parencite{koblitz2013, gupta2016}. 

Ultimately, HMAC’s security rests on a combination of rigorous proofs, practical validation, and adaptable design principles. It is this interplay—across PRF security, mechanized proof, functional and provable security, quantum resilience, and real-world deployments—that explains why HMAC remains a strong and trusted construction in cryptography today.

\printbibliography
\end{document}
