\mode*

\section{Digital Postbox}

Traditionally we send paper mail by addressing a letter to a physical address 
(including the recipient's name).
We write the address on an envelope and send it.
We want a digital transformation.
You should design a digital postbox and delivery service.

This is a first attempt.
Do not worry if your design is underspecified or if you end up with unanswered 
questions.
After the protocol lecture series (Week 5), we will return to your designs in 
Week 6 and use the protocol tools to make the missing parts explicit.

Document your group's design in your group's tab in
          \url{https://tinyurl.com/tilkry26-postbox}
%\url{https://docs.google.com/document/d/1eRv-ekY04tdTCvwss4dK3juKlUez4CnoZihcC4J9mKc/edit}
\ltnote{%
  Pedagogical intent (try-first-tell-later + variation theory):
  Here students make a full, unconstrained design attempt.
  We deliberately avoid giving a protocol checklist, so that (i) students must 
  decide what is critical vs non-critical on their own, and (ii) the protocol 
  lecture series can later create contrast between what they wrote down and 
  what a protocol description needs to make explicit.
  In Week 6 we return to the same object of learning (designing a secure 
  postbox service), but with new tools; the redesign then fuses several aspects 
  (security properties, attacker model, message flow, key lifecycle, recovery) 
  that typically remained implicit in Week 4.
}

\mode<presentation>{%
\begin{frame}
  \begin{idea}[Digital mail]
    \begin{itemize}
      \item We send paper mail by writing an address on an envelope.
      \item We want to have a digital version.
    \end{itemize}
  \end{idea}

  \begin{remark}
    \begin{itemize}
      \item We have a few already: Kivra, eBoks, Digimail, Min myndighetspost.
      \item Managed by DIGG\footnote{%
          See \url{https://www.digg.se/digital-post}.
        }
    \end{itemize}
  \end{remark}
\end{frame}
}

\begin{frame}
\begin{exercise}[Functionality]
  What functionality do we need? (High level)
\end{exercise}

\mode<presentation>{%
  \begin{onlyenv}<2>
    \begin{solution}
      We'd need at least the following:
      \begin{itemize}
        \item Registration
        \item Sending mail
        \item Reading mail
      \end{itemize}
    \end{solution}
  \end{onlyenv}
}
\end{frame}

\mode<article>{%
\begin{solution}[Sketch]
  At a high level, a digital postbox service needs at least:
  \begin{itemize}
    \item registration (create an account tied to a real identity),
    \item sending (submit a letter to a recipient address),
    \item reading (retrieve and open received letters).
  \end{itemize}
  Typical additional functionality that quickly becomes necessary in practice:
  \begin{itemize}
    \item key management (publish/rotate/revoke keys),
    \item recovery (lost device/credentials),
    \item delegation (e.g., power of attorney),
    \item auditing and dispute handling.
  \end{itemize}
\end{solution}
}

\mode<article>{%
\begin{center}
  \rule{0.8\linewidth}{0.4pt}
\end{center}
}

\begin{frame}
\begin{exercise}[Requirements]
  What are the requirements on the service?
\end{exercise}
\end{frame}

\mode<article>{%
\begin{solution}[Sketch]
  A reasonable first list of requirements is:
  \begin{itemize}
    \item confidentiality of letter contents against intermediaries,
    \item authenticity/integrity (recipient can verify who sent what),
    \item access control (only the owner/delegate can read),
    \item availability (mail must be reachable when needed),
    \item clear trust assumptions and a recovery story.
  \end{itemize}
  Note that even if contents are protected end-to-end, metadata (who talks to
  whom, when, and roughly how much) is typically still observable.
\end{solution}
}

\mode<article>{%
\begin{center}
  \rule{0.8\linewidth}{0.4pt}
\end{center}
}

We have the following requirements on the service:
\begin{frame}
\begin{description}
  \item[Registration]
    A user can register for a post box.
    As this post service will replace physical mail, it must be tied to a real 
    identity with a physical address.

    \pause

  \item[Sending mail]
    A sender should be able to send mail to a recipient.
    This mail should not be readable by any intermediary.

  \item[Reading mail]
    The owner can read the mail.
    Only the owner should be able to do this --- no one else, not even service 
    staff!
    Recipient must be able to verify the authenticity of mail.
%
%  \item[Anonymous mail] (Extra feature)
%    A sender can send a letter to a recipient, but neither the recipient nor 
%    the service will learn the identity of the sender.
%    The service will not learn the recipient either.
\end{description}
\end{frame}

\begin{frame}
  \begin{remark}
    DIGG\footnote{Myndigheten för digital förvaltning} sets the official 
    requirements, see
    \url{https://www.digg.se/digitala-tjanster/digital-post/digital-post-for-dig-som-leverantor/allmanna-villkor/allmanna-villkor-fr.o.m.-16-januari-2023/bilaga-1-krav-pa-sakerhet-for-brevladeoperatorer}.
  \end{remark}
\end{frame}

%\begin{frame}[fragile]
%  \begin{remark}[Requirement]
%    \begin{itemize}
%      \item Must work with EIDAS\@.
%      \item Implementations from other countries might use other algorithms.
%    \end{itemize}
%  \end{remark}
%\end{frame}

Solving this assignment will touch upon almost every topic in the course.
Make sure to base your designs on the theory of the course, add references 
(that will help you).
\enquote{This feels secure} is not a convincing argument.
Likewise, \enquote{all connections should use TLS} will not cut it either; why 
do you want TLS, what properties do you need and which of those will TLS 
provide and why?

\mode<presentation>{%
  \begin{frame}
    \begin{exercise}[Design the digital mail service]
      \begin{description}
        \item[Registration]
          Alice can register for a post box.

        \item[Sending mail]
          Alice should be able to send mail to a recipient Bob.
          Mail should not be readable by any intermediary.

        \item[Reading mail]
          Bob can read all his received mail.
          Service staff should not be able to read mail!
          Must be able to verify the authenticity.
      \end{description}
    \end{exercise}
  \end{frame}

  \begin{frame}
    \begin{exercise}
      \begin{itemize}
        \item Registration
        \item Sending mail
        \item Reading mail
      \end{itemize}
    \end{exercise}

    \begin{block}{Organization}
      \footnotesize
      \begin{itemize}
        \setlength{\itemsep}{0pt}
        \setlength{\parskip}{0pt}
        \setlength{\parsep}{0pt}
        \item Groups of around three--four people.
        \item First sketch the full service together.
        \item Then pick one of the three functionalities.
        \item \alert<2>{Try to go into detail: detail the crypto protocol.}
        \item Document in your group's tab:
          \url{https://tinyurl.com/tilkry26-postbox}
          %\url{https://docs.google.com/document/d/1eRv-ekY04tdTCvwss4dK3juKlUez4CnoZihcC4J9mKc/edit}
        \item Work for 20 minutes.
        \item Group presentations and discussions.
      \end{itemize}
    \end{block}
  \end{frame}

  \begin{frame}
    \begin{solution}
      \begin{itemize}
        \item Presentation from groups.
        \item Comments from everyone else.
      \end{itemize}
    \end{solution}

    \begin{question}
      \begin{itemize}
        \item Obstacles?
        \item Limitations?
        \item Recovery from failure?
        \item In whom do we trust?
      \end{itemize}
    \end{question}
  \end{frame}
}

\mode<article>{%
\begin{solution}[Sketch]
  One common starting point is:
  \begin{itemize}
    \item a directory mapping recipient addresses to public keys (including
      status: current/\allowbreak revoked),
    \item hybrid encryption for letters (random symmetric key + public-key
      encryption of that key to the recipient),
    \item sender authentication via digital signatures on the letter.
  \end{itemize}
  The hard parts tend to be lifecycle questions (key rotation, revocation,
  recovery) and defining what the service is trusted to learn and to do.
\end{solution}
}

\mode<article>{%
\begin{center}
  \rule{0.8\linewidth}{0.4pt}
\end{center}
}

Notes from the session:
\begin{itemize}
  \item Same as Signal protocol.
  \item Just tie in BankID to get real identities.
  \item The problem of end-to-end encryption: lost keys doesn't work. But it 
    still works with the existing digital mail services.
\end{itemize}

\begin{frame}
  \begin{exercise}[Data and metadata]
    What data and metadata are revealed to whom?
    How does this compare to the physical mail service?
  \end{exercise}
\end{frame}

\mode<article>{%
\begin{solution}[Sketch]
  Separate content from metadata.
  End-to-end protection can hide letter contents from the operator, but the
  operator may still learn (depending on design):
  \begin{itemize}
    \item sender and recipient identifiers (or at least routing identifiers),
    \item timestamps, frequency, and rough size,
    \item device/IP-level metadata (unless mitigated).
  \end{itemize}
  Compare to physical mail: the postal service sees envelope address fields and
  handling times, but not the letter contents.
\end{solution}
}

\mode<article>{%
\begin{center}
  \rule{0.8\linewidth}{0.4pt}
\end{center}
}

\mode<presentation>{%
  \begin{frame}
    \begin{exercise}[Adapt for anonymity]
      \begin{description}
        \item[Weak sender anonymity]
          Alice can send a letter to Bob, but the service will not learn the 
          identity of the sender.

        \item[Sender anonymity]
          Alice can send a letter to Bob, but neither Bob nor the service will 
          learn the identity of the sender.

        \item[Sender--receiver anonymity]
          Same, but the service will not learn that Bob is the recipient 
          either.
      \end{description}
    \end{exercise}

    \pause

    \begin{block}{Organization}
      \footnotesize
      \begin{itemize}
        \setlength{\itemsep}{0pt}
        \setlength{\parskip}{0pt}
        \setlength{\parsep}{0pt}
        \item Groups of around three--four people.
        \item Work for 20 minutes, document: 
          \url{https://tinyurl.com/tilkry26-postbox}
          %\url{https://docs.google.com/document/d/1eRv-ekY04tdTCvwss4dK3juKlUez4CnoZihcC4J9mKc/edit}
        \item Group presentations and discussions.
      \end{itemize}
    \end{block}
  \end{frame}

  \begin{frame}
    \begin{solution}
      \begin{itemize}
        \item Presentation from groups.
        \item Comments from everyone else.
      \end{itemize}
    \end{solution}

    \begin{question}
      \begin{itemize}
        \item Obstacles?
        \item Limitations?
        \item Recovery from failure?
        \item In whom do we trust?
      \end{itemize}
    \end{question}
  \end{frame}
}

\mode<article>{%
\begin{solution}[Sketch]
  Start by stating precisely what should be hidden and from whom.
  \begin{itemize}
    \item Weak sender anonymity: hide the sender from the service (but not
      necessarily from the recipient).
    \item Sender anonymity: hide the sender from both service and recipient.
    \item Sender--receiver anonymity: additionally hide the recipient from the
      service.
  \end{itemize}
  Typical building blocks (kept deliberately high level): unlinkable payment or
  access tokens, relays/mix networks, and careful metadata minimization.
\end{solution}
}

\mode<article>{%
\begin{center}
  \rule{0.8\linewidth}{0.4pt}
\end{center}
}


\section{Digital identity}

\begin{frame}
  \begin{exercise}
    \begin{itemize}
      \item What requirements do we have on digital identity for our postbox 
        system?
        How can we achieve those requirements?
      \item Document in your group's tab:
          \url{https://tinyurl.com/tilkry26-postbox}
        %\url{https://docs.google.com/document/d/1eRv-ekY04tdTCvwss4dK3juKlUez4CnoZihcC4J9mKc/edit}
    \end{itemize}
  \end{exercise}
\end{frame}

\mode<article>{%
\begin{solution}[Sketch]
  Identity requirements for a postbox service often include:
  \begin{itemize}
    \item strong binding from account to a real person (and sometimes address),
    \item authentication with appropriate assurance (phishing and
      account-takeover resistance),
    \item revocation and recovery (lost credentials/devices),
    \item attribute minimization (use only what is needed for the task),
    \item delegation (guardianship/power of attorney).
  \end{itemize}
  When reusing an external eID, be explicit about what you inherit (assurance
  level, enrollment checks, and recovery policy).
\end{solution}
}
