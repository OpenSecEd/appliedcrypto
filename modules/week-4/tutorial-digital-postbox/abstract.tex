\emph{Summary:}
This tutorial uses digital postbox services (e.g., Kivra/eBoks) as a
case study for designing a security-critical system.
You sketch a design for registration, sending, and reading mail; you then
analyze which security properties you need, what metadata is exposed to whom,
and what requirements this implies on digital identity.
The aim is a first (possibly incomplete) design that we revisit after the
protocol lecture series.

\emph{Intended learning outcomes:}
After this tutorial, you should be able to:

\begin{itemize}
  \item analyze a real-world service and identify its security goals and
    assumptions;
  \item apply basic cryptographic primitives (encryption, signatures, hash
    functions, and message authentication codes) to sketch a solution to a
    design problem;
  \item reason about the difference between protecting content and protecting
    metadata, and compare digital mail to physical mail;
  \item describe key lifecycle and recovery questions (revocation, device
    loss, delegation) that become critical in end-to-end protected systems;
  \item formulate requirements on digital identity (binding, assurance,
    authentication, recovery) for a public-facing service.
\end{itemize}

\emph{Prerequisites:}
Basic familiarity with the course notions of confidentiality and authenticity,
and with encryption and digital signatures at a high level.

\emph{Reading:}
The official security requirements for Swedish digital postbox operators (DIGG):
\url{https://www.digg.se/digitala-tjanster/digital-post/digital-post-for-dig-som-leverantor/allmanna-villkor/allmanna-villkor-fr.o.m.-16-januari-2023/bilaga-1-krav-pa-sakerhet-for-brevladeoperatorer}.
