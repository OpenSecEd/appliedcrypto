\title{Ciphers and shared-key encryption}
\author{Daniel Bosk\thanks{%
    With some ideas from Douglas Wikström.
}}
\institute{KTH EECS}

\mode*

\begin{frame}
  \maketitle
\end{frame}

\begin{abstract}
  Public-key cryptography has a central role in modern cryptographic solutions, 
it ranges from HTTPS to electronic ID systems.

The \acp{ILO} are that you should be able
\begin{itemize}
  \item to \emph{apply} various public-key cryptography primitives in 
    situations based on their high-level properties.
\end{itemize}

The basics are covered by
Chapter 5 in Anderson's \citetitle{Anderson2008sea}~\cite{Anderson2008sea} and
Chapter 14 in Gollmann's \citetitle{Gollmann2011cs}~\cite{Gollmann2011cs}.
Furthermore, we refer to the \citetitle{EOCS}~\cite{EOCS}.

\end{abstract}

\clearpage

\begin{frame}
  \only<presentation>{\tableofcontents[hideallsubsections]}
  \only<article>{\tableofcontents}
\end{frame}

\clearpage

\section{Introduction}

Cryptography is one of the oldest arts, yet also one of the most modern 
sciences.
Before we dive into definitions and constructions, take a moment to reflect on 
what the word \enquote{cryptography} means to you.

\begin{frame}
  \begin{exercise}
    When you hear \enquote{cryptography}, what do you think it does?
  \end{exercise}

  \ltnote{%
    Try-first: activate prior knowledge before terminology.
    The contrast between student intuition and the definitions below makes the
    scope (contents vs existence) discernible.
  }
\end{frame}

The word \emph{cryptography} derives from Greek: \emph{kryptos} 
(\ibygr{krupto's}) meaning \enquote{hidden}, and \emph{graphos} 
(\ibygr{gra'fos}) meaning \enquote{writing}.
Literally, cryptography is the art of hidden writing.
The field has existed for millennia---from ancient Egyptian hieroglyphics to 
Roman military ciphers to modern computer algorithms.

It is important to distinguish cryptography from a related but distinct field: 
\emph{steganography}.
While both aim to protect communication, they do so in fundamentally different 
ways.
Steganography hides the \emph{existence} of a message (for example, by 
embedding secret text within an innocent-looking image), whereas cryptography 
hides the \emph{contents} of a message while leaving its existence visible.
An eavesdropper knows a cryptographic message was sent, but cannot understand 
it without the key.

\mode<presentation>{%
\begin{frame}
  \begin{itemize}
    \item The word has its origin in greek~\footfullcite{OED2013cg}:
      \begin{description}
        \item[\ibygr{krupto's}] (\emph{kryptos}) meaning 
          hidden~\footfullcite{OED2013c}.
        \item[\ibygr{gra'fos}] (\emph{graphos}) meaning 
          writing~\footfullcite{OED2013g}.
      \end{description}

      \pause{}

    \item The area has been around for ages.

      \pause{}

    \item We should not confuse it with \emph{steganography}.
    \item Steganography concerns hiding a message's \emph{existence}.
    \item Cryptography concerns hiding a message's \emph{contents}.
  \end{itemize}
\end{frame}
}

Modern cryptography goes far beyond simple secret writing.
As \textcite{GoldreichFOC-1} notes, cryptography is concerned with the 
conceptualization, definition, and construction of computing systems that 
address security concerns.
This includes not only encryption, but also digital signatures, authentication 
protocols, secure computation, and much more.
Indeed, any concept that has security properties can be considered a 
cryptographic notion.

\begin{frame}
  \blockcquote{GoldreichFOC-1}{%
    Cryptography is concerned with the conceptualization, definition,
    and construction of computing systems that address security
    concerns.%
  }

  \blockcquote{Douglas Wikström}{%
    Cryptography is not limited to the construction and analysis of
    well-known notions such as encryption or hash functions. Any
    concept which has security properties is a cryptographic notion.
  }
\end{frame}

Historically, cryptography was an art practiced by a select few---military 
officers, diplomats, and spies who devised \enquote{clever} schemes to 
obscure their messages.
The security of these schemes rested on the assumption that adversaries could 
not figure out the method.
This turned out to be a poor assumption: there are always people with 
sufficient time, motivation, and ingenuity to break such schemes.
The transition from art to science came with the realization that security 
must be provable, not merely hoped for.

\mode<presentation>{%
\begin{frame}
  \begin{itemize}
    \item Then it was an art, now it's a science.

      \pause{}

    \item People used \enquote{clever} constructions.
    \item These were thought to be secure: \enquote{How can anyone figure this 
        out?}

      \pause{}

    \item Well, it turns out that there are always a lot of people with a lot 
      of time and motivation \dots
  \end{itemize}
\end{frame}
}

\subsection{Outline}

Modern cryptography can be organized into three broad categories, each with 
its own flavor and historical development.

\begin{frame}
\begin{description}
  \item[Shared-key (symmetric) cryptography]
    \only<presentation>{Stems from the classical crypto where a key is shared 
    between two users.}
    \only<article>{The oldest form, where the same secret key is shared between 
      the communicating parties.
      This is the focus of this lecture.
      The main challenge is \emph{key distribution}: how do Alice and Bob agree 
    on a shared secret in the first place?}

  \item[Public-key (asymmetric) cryptography]
    \only<presentation>{This is more modern crypto, from 1970s.
    Each user has a public and a private key.}
    \only<article>{A revolutionary development from the 1970s.
      Each user has a pair of keys: a public key (known to everyone) and a 
      private key (known only to the owner).
      This elegantly solves the key distribution problem but introduces new 
    computational assumptions.}

  \item[Counter-intuitive cryptography]
    \only<presentation>{More modern, from 1980s and onwards.
      How to do computations on secret inputs, prove knowledge without 
    revealing of what.}
    \only<article>{Modern developments, from the 1980s onward.
      These include zero-knowledge proofs (proving you know something without 
      revealing what), secure multi-party computation (computing on secret inputs 
    without revealing them), and other seemingly impossible feats.}

  \item[Fully homomorphic encryption]
    More recent (early 2000s) allowing arbitrary computations on encrypted data 
    without decrypting it first.
    This has implications for cloud computing and data privacy.
\end{description}
\end{frame}


\section{Shared-key cryptography}

In shared-key (or symmetric) cryptography, Alice and Bob share a common 
secret---a \emph{key}---that they use to encrypt and decrypt messages.
This is the oldest and most intuitive form of cryptography: if you and a 
friend share a secret code, you can communicate privately even if others are 
listening.

But what exactly is this \enquote{shared secret}?
Before we formalize the notion of a cipher, let us explore what kinds of 
secrets people already use in everyday life.

\ltnote{%
  Variation theory (try-first): start from learners' everyday experience
  (invariant: "private communication"; varies later: "how we achieve it")
}
\begin{frame}
  \begin{question}
    Alice and Bob wish to communicate in private.
    You all know how to do this with friends! How?
  \end{question}

\end{frame}

\note{%
  Use a quick whole-class prompt: “What do you already do to keep a message
  private from others?” Collect 2--3 ideas (e.g., whispering, inside jokes,
  code words). Do not evaluate yet; we want prior knowledge on the table.
}

\ltnote{
  Variation theory (contrast): show "public transmission" vs "private 
  meaning"
  Invariant: message exchange; Varies: representation (m vs c)
}
\begin{frame}
  \begin{question}
    Is the following possible?
    \begin{enumerate}
      \item Alice transforms her message \(m\) into a garbled string \(c\).
      \item Alice sends the garbled string \(c\) to Bob.
      \item Bob transforms \(c\) back to \(m\).
    \end{enumerate}

    \medskip
    \emph{Goal:} \(c\) hides the content of \(m\).
  \end{question}

  \begin{onlyenv}<2>
  \begin{exercise}
    What do we need for this to work?
  \end{exercise}

  \end{onlyenv}
\end{frame}

\ltnote{%
  \textbf{Variation pattern}: Try-first.

  \textbf{What varies}: Student hypotheses about the missing ingredient.

  \textbf{What remains invariant}: The cipher setup (message, garbled text, 
  transformation).

  \textbf{Critical aspect}: Bob can invert the transformation \emph{only if} he 
  has extra information that Eve lacks.
  This surfaces the need for a shared secret (key).
}

\note{%
Ask students to point out what each symbol means: \(m\) (message/meaning),
\(c\) (ciphertext/garbled text). Emphasize that the channel is assumed to be
observable (others can see \(c\)). Then ask: “What could go wrong if anyone
can do step 3?” Lead into the missing ingredient.

Give 30--60 seconds of think-pair-share. Listen for phrases like:
“a password,” “a key,” “something only they know.” If someone says
“a more complicated algorithm,” steer them toward: complexity is not
enough if it is public—what matters is \emph{secret information}.
}

\begin{frame}
  \begin{solution}
    We have two options:
    \begin{enumerate}
      \item Either we keep the garbling procedure secret.
      \item Or we need a secret component in a publicly known garbling.
    \end{enumerate}
  \end{solution}
\end{frame}

\begin{frame}
  \begin{example}[Secret language]
    \foreignquote{swedish}{Rövarspråket}:
    \begin{itemize}
      \item Each consonant \(x\) is replaced by the triplet \(xox\).
      \item Each vowel is left unchanged.
    \end{itemize}
  \end{example}
\end{frame}

The missing ingredient in any cipher is a \emph{joint secret}---some piece of 
information that Alice and Bob share but that is difficult for an adversary to 
guess.
In cryptography, this joint secret is called a \emph{key}.

\begin{frame}
  \begin{definition}[Joint secret]
    A \emph{joint secret} is a piece of shared information of Alice and Bob
    that is hard to guess for others.
  \end{definition}
\end{frame}

\ltnote{%
  \textbf{Variation pattern}: Generalization.

  \textbf{What varies}: The form of the secret (memory-based, random bits, 
  passwords, inside jokes, secret handshakes).

  \textbf{What remains invariant}: The concept of a joint secret---shared 
  information that is hard for others to guess.

  \textbf{Critical aspect}: Students should discern that the \emph{principle} 
  of shared secrets is independent of the particular form the secret takes.
}

\note{%
Connect back to the first question: many “friend privacy” methods rely on
shared background knowledge. State explicitly: in cryptography this shared
secret is typically called a \emph{key}. (You can mention the word “key”
now, or save it for the next slide if you prefer pacing.)
}

Joint secrets can take many forms.
A shared memory (\enquote{Remember how many fish we caught that rainy day?}) 
is meaningful to Alice and Bob but hard for an outsider to guess.
However, such memory-based secrets have limitations: they may not be uniformly 
hard to guess, and information might leak through social media or mutual 
friends.

\begin{frame}
  \begin{example}[Joint secret: shared memory]
    \enquote{Remember how many fish we caught that rainy day?}
  \end{example}

\ltnote{%
  \textbf{Variation pattern}: Contrast.

  \textbf{What varies}: Perspective on the secret---meaningful to Alice and Bob 
  versus hard to guess for outsiders.

  \textbf{What remains invariant}: The secret itself (the shared memory).

  \textbf{Critical aspect}: A good secret must be \emph{both} meaningful to the 
  parties (so they can remember and use it) \emph{and} hard for adversaries to 
  guess.
  This foreshadows the tension between usability and security in key management.
}


  \begin{example}[Joint secret: random bits]
    Alice and Bob agree on a sequence of randomly chosen bits intended
    for use later.
  \end{example}

\ltnote{%
  \textbf{Variation pattern}: Generalization.

  \textbf{What varies}: The nature of the secret---from informal (shared 
  memories) to formal (random bit strings).

  \textbf{What remains invariant}: The requirement that the secret is shared 
  only by the communicating parties.

  \textbf{Critical aspect}: Students should discern that \emph{security should 
  not rely on obscurity of the method}, but on the secrecy of the key.
  This prepares them for Kerckhoff's principle.
}
\end{frame}

\note{%
Ask: “Why might this be hard for an outsider?” Highlight: it depends on
shared experience. Then point out a limitation: it may not be uniformly
hard to guess; it could leak via social media, friends, etc. This motivates
a more formal secret.

Stress: the bits are \emph{random} and \emph{shared only by Alice and Bob}.
Then connect to the cipher story: the missing ingredient is a joint secret
(a shared key) that enables Bob to reverse the transformation, while others
cannot. Optionally preview: next we will formalize encryption/decryption
as algorithms that take \(m\) and a key \(k\) to produce \(c\).
}

A more formal approach is to use a sequence of randomly chosen bits as the 
key.
Random bits have the advantage that their guessing probability is precisely 
quantifiable: a truly random \(n\)-bit key can be guessed with probability at 
most \(2^{-n}\).
This leads to a fundamental principle that separates modern cryptography from 
its historical predecessors.

\subsection{Kerckhoff's Principle}

In 1883, Auguste Kerckhoffs articulated a principle that remains central to 
cryptographic design: a cryptosystem should remain secure even if everything 
about it, except the key, becomes public knowledge.
This is often paraphrased as \enquote{no security by obscurity}.

\begin{frame}
  \begin{displayquote}\relax
    [A cryptosystem] should not require secrecy, and it should not be 
    a problem
    if it falls into the enemy hands;
  \end{displayquote}

  \pause{}

  \begin{block}{Kerckhoff's Principle}
    \begin{itemize}
      \item No security-by-obscurity
      \item The key should be the only secret
    \end{itemize}
  \end{block}
\end{frame}

This does not mean we must broadcast our cryptographic choices to the world.
Rather, it means that our security analysis should assume the adversary knows 
everything except the key.
If the system remains secure under this assumption, then any additional 
obscurity is a bonus rather than a requirement.
This principle has practical benefits: cryptographic algorithms can be 
publicly scrutinized by experts, implementations can be standardized, and 
security does not depend on keeping algorithms secret---something that is 
notoriously difficult in practice.

\mode<presentation>{%
\begin{frame}
  \begin{remark}
    \begin{itemize}
      \item This doesn't mean we must tell the adversary what we're using.
      \item But we shouldn't loose any security if we do.
    \end{itemize}
  \end{remark}
\end{frame}
}

\subsection{Ciphers}

With Kerckhoff's principle in mind, we can now formalize what a cipher 
(symmetric cryptosystem) is.
The basic idea is simple: Alice and Bob share a small secret key, Alice uses 
it to transform her message into an unintelligible ciphertext, and Bob uses 
the same key to recover the original message.
An eavesdropper (traditionally called Eve) who captures the ciphertext should 
learn nothing about the message.

\mode<presentation>{%
\begin{frame}
  \begin{idea}
    \begin{itemize}
      \item Alice and Bob share a (small) common secret.

        \pause{}

      \item Alice takes a message, combines it with the secret, sends it to 
        Bob.

        \pause{}

      \item If Eve captures whatever Alice sent, she shouldn't learn anything 
        about the message.

        \pause{}

      \item Bob combines what he received with the secret and gets the message.
    \end{itemize}
  \end{idea}
\end{frame}
}

\begin{frame}
  \begin{block}{Block-cipher encryption}
    \begin{description}
      \item[Input] A fixed-sized \emph{key} \(k\), a fixed-sized block of 
        \emph{plaintext} \(p\).
      \item[Output] A fixed-sized block of \emph{ciphertext} \(c\).
      \item[Notation] \(\Enc[_k][p] = c\)
    \end{description}
  \end{block}

  \pause{}

  \begin{block}{Block-cipher decryption}
    \begin{description}
      \item[Input] A fixed-sized \emph{key} \(k\), a fixed-sized block of 
        \emph{ciphertext} \(c\).
      \item[Output] A fixed-sized block of \emph{plaintext} \(p\).
      \item[Notation] \(\Dec[_k][c] = p\)
    \end{description}
  \end{block}
\end{frame}

A \emph{block cipher} operates on fixed-size blocks of data.
Given a key \(k\) and a plaintext block \(p\), the encryption function 
produces a ciphertext block \(c = \Enc_k(p)\).
Decryption reverses this: \(\Dec_k(c) = p\).
The key must be the same for both operations, which is why this is called 
\emph{symmetric} encryption.

\begin{frame}
  \begin{definition}[Crypto 
    system]\footfullcite{Stinson2006cta}\label{CryptoSystem}
    A \emph{crypto system} is a tuple \((\M, \C, \K, \E, \D)\), where:
    \begin{itemize}
      \item \(\M\) is a finite set of \emph{plaintexts} or messages,
      \item \(\C\) is a finite set of \emph{ciphertexts},
      \item \(\K\) is the \emph{keyspace}, a finite set of keys,
      \item \(\E\) and \(\D\) are the sets of encryption and decryption
        rules, respectively.
    \end{itemize}

    \pause{}

    For every \(k\in \K\) there is an \(\Enc_k\in \E\) and a \(\Dec_k\in \D\) 
    such that:
    \begin{itemize}
      \item \(\Enc_k\colon \M\to \C\) and \(\Dec_k\colon \C\to \M\) are 
        functions, and
      \item \(\Dec[_k][\Enc[_k][m]] = m\) for all plaintexts \(m\in \M\).
    \end{itemize}
  \end{definition}
\end{frame}

Formally, a cryptosystem consists of five components: a message space \(\M\), 
a ciphertext space \(\C\), a keyspace \(\K\), and families of encryption and 
decryption functions \(\E\) and \(\D\).
The fundamental requirement is \emph{correctness}: for every key \(k\) and 
message \(m\), decrypting the encryption of \(m\) must yield \(m\) again.
That is, \(\Dec_k(\Enc_k(m)) = m\).

As a simple example, consider the \emph{shift cipher} (also known as the 
Caesar cipher, after Julius Caesar who reportedly used it).
We represent letters as integers modulo some number (for example, 29 for the 
Swedish alphabet), and encryption simply adds the key to each letter.

\begin{frame}
  \begin{definition}[Shift Cipher]\label{ShiftCipher}
    We define it as follows:
    \begin{itemize}
      \item Let \(\M = \C = \K = \ZZ_{29}\).
      \item For each \(k\in \K\) we define
        \begin{align}
          \nonumber
          \Enc[_k][m] &= (m + k) \bmod 29, m\in \M, \text{\ och } \\
          \nonumber
          \Dec[_k][c] &= (c - k) \bmod 29, c\in \C.
        \end{align}
    \end{itemize}
  \end{definition}

  \pause{}
  \mode<presentation>{\only<2>{\vspace{-2em}}}

  \begin{example}
    With key \(k = 3\), we encrypt individual letters as follows:
    \begin{itemize}
      \item \(\Enc[_3][7] = 7+3 \bmod 29 = 10\)\hfill h\(\to\)J
      \item \(\Enc[_3][4] = 4+3 \bmod 29 = 7\)\hfill e\(\to\)G
      \item \(\Enc[_3][9] = 9+3 \bmod 29 = 12\)\hfill j\(\to\)L
    \end{itemize}
  \end{example}
\end{frame}

\mode<presentation>{%
\begin{frame}
  \begin{remark}
    \begin{itemize}
      \item The shift cipher is a classical cipher --- also know as the Caesar 
        Cipher.
      \item It's easily broken \emph{by hand}!
      \item It's used here for illustrative purposes.
    \end{itemize}
  \end{remark}
\end{frame}
}

The shift cipher is trivially insecure: there are only 29 possible keys (or 
26 for the English alphabet), so an attacker can simply try all of them.
We use it here purely for illustration, not as a model of secure encryption.
The question of what makes a cipher \emph{secure} is more subtle, and we turn 
to it next.

An obvious improvement to the shift cipher is to use multiple shifts in 
sequence.
The \emph{Vigénère cipher} does exactly this: instead of a single shift 
applied to every letter, a short keyword determines a sequence of shifts that 
repeat cyclically.
This was considered so strong that it was called \enquote{le chiffre 
indéchiffrable}---the unbreakable cipher---and resisted cryptanalysis for 
nearly 300 years.
We will examine the Vigénère cipher in detail when we study why classical 
ciphers fail, and we will see that its downfall comes from the key being 
\emph{shorter than the message} and reused.

\section{Security}

What does it mean for a cipher to be secure?
The strongest notion is \emph{perfect secrecy}, formalized by Claude Shannon 
in 1949.
Intuitively, a cipher has perfect secrecy if observing the ciphertext gives 
an adversary no additional information about the plaintext beyond what they 
already knew.

\begin{frame}
  \begin{definition}[Perfect secrecy]\footfullcite{ShannonSecrecy}
    Consider a cryptosystem \((\M, \C, \K, \E, \D)\) with stochastic variables 
    \(M\) (message) and \(C\) (ciphertext):
    \begin{itemize}
      \item The system has \emph{perfect secrecy} if and only if 
        \[\Pr(M = m\mid C = c) = \Pr(M = m)\] 
        for all \(m\in \M\) and \(c\in \C\).
    \end{itemize}
  \end{definition}

  \pause{}

  \begin{remark}
    Equivalent to \(H(M\mid C) = H(M)\), i.e.\ ciphertext does not reveal 
    anything about plaintext.
  \end{remark}
\end{frame}

Mathematically, perfect secrecy means that the probability distribution over 
messages is unchanged by observing the ciphertext: 
\(\Pr(M = m \mid C = c) = \Pr(M = m)\).
In information-theoretic terms, the conditional entropy of the message given 
the ciphertext equals the unconditional entropy: \(H(M \mid C) = H(M)\).
The ciphertext reveals \emph{nothing}.

Shannon's theorem characterizes exactly when perfect secrecy is achievable.

\begin{frame}
  \begin{theorem}[Shannon's theorem]
    \begin{itemize}
      \item Assume cryptosystem \((\M, \C, \K, \E, \D)\) such that \(|\K| 
          = |\C| = |\M|\).

        \pause{}

      \item This provides perfect secrecy if and only if
        \begin{enumerate}
          \item every key \(k\in \K\) is used with equal probability 
            \(1/|\K|\),
          \item for every plaintext \(m\in \M\) and \(c\in \C\) there is a 
            unique key such that \(\Enc[_k][m] = c\).
        \end{enumerate}
    \end{itemize}
  \end{theorem}
\end{frame}

The theorem states that perfect secrecy requires the keyspace to be at least 
as large as the message space, and keys must be used uniformly at random.
This has a sobering implication: for perfect secrecy, you need as much key 
material as message material.

The \ac{OTP} achieves perfect secrecy.
The key is a random bit string as long as the message, and encryption is 
simply XOR (addition modulo 2).
Decryption is the same operation, since XOR is its own inverse.

\begin{frame}
  \begin{example}[One-time Pad]
    We construct it as follows:
    \begin{itemize}
      \item Let \(n\) be a positive integer.
      \item Let \(\M = \C = \K = (\ZZ_2)^n\).

        \pause{}

      \item For each key \(k = (k_1, \ldots, k_n)\in \K\), plaintexts 
        \(m = (m_1, \ldots, m_n)\in \M\) and ciphertexts 
        \(c = (c_1, \ldots, c_n)\in \C\), we define
        \[\Enc[_k][m] = (m_1 + k_1, \ldots, m_n + k_n).\]

        \pause{}

      \item We also define \(\Dec = \Enc\).

        \pause{}

      \item The key \(k\in \K\) must be chosen uniformly at random for each 
        encryption.
    \end{itemize}
  \end{example}
\end{frame}

The \ac{OTP} can be viewed as a binary version of the Vigénère cipher, with 
two crucial differences: the key is as long as the message (so there is no 
repetition), and the key is used only once.
These differences are precisely what gives the \ac{OTP} perfect secrecy, 
whereas Vigénère's key repetition proved its undoing.

The one-time pad is provably unbreakable, but it has severe practical 
limitations.
The key must be as long as the message, must be truly random, must be shared 
securely in advance, and---crucially---must never be reused.
If the same key is used twice, an attacker can XOR the two ciphertexts to 
obtain the XOR of the two plaintexts, which often reveals both.
Key distribution and the prohibition on reuse make the \ac{OTP} impractical for 
most applications.

In practice, we settle for \emph{computational} security rather than 
information-theoretic security.
We use ciphers based on \emph{\aclp{PRP}} (\acsp{PRP}): functions that are 
indistinguishable from truly random permutations by any efficient algorithm.

\begin{frame}
  \begin{definition}[\Acl{PRP}, \acs{PRP}]\footfullcite{KatzLindell-v1}
    Let \(F\colon \{0,1\}^s\times \{0, 1\}^n\to \{0,1\}^n\).
    We say \(F\) is \iac{PRP} if:
    \begin{enumerate}
      \item for any \(k\in \{0, 1\}^s\), \(F_k\) is a bijection;

        \pause{}

      \item for any \(k\in \{0, 1\}^s\), we can \enquote{efficiently} 
        evaluate \(F_k(x)\);

        \pause{}

      \item for all \enquote{efficient} distinguishers~\(D\),
        \[\left|\Pr[D^{F_k}(1^n) = 1] - \Pr[D^{f_n}(1^n) = 1]\right| 
          < \epsilon(s)\] 
        when we choose \(k\in \{0,1\}^s\) and the random 
        permutation \(f_n\) uniformly at random.
    \end{enumerate}
  \end{definition}
\end{frame}

A \ac{PRP} is a function that, when keyed, behaves like a random permutation to 
anyone who does not know the key.
Modern block ciphers like \ac{AES} are designed to be \acsp{PRP}: given a 
128-bit key, \ac{AES} maps 128-bit blocks to 128-bit blocks in a way that is 
computationally indistinguishable from a random permutation.

\subsection{Why classical ciphers fail}

Before modern cryptography, people designed ciphers based on intuition and 
cleverness.
Let us examine why these classical ciphers fail, and what lessons we can draw 
for modern cipher design.

\mode<presentation>{%
\begin{frame}
  \begin{definition}[Shift cipher (Caesar)]
    \begin{itemize}
      \item Think of letters as integers modulo \(n\) (e.g., \(n=26\) or
        include space).
      \item Key \(k\in\mathbb{Z}_n\).
      \item \(\Enc[_k][m] = (m+k)\bmod n\), \(\Dec[_k][c] = (c-k)\bmod n\).
    \end{itemize}
  \end{definition}
\end{frame}
}

The shift cipher (Caesar cipher) is the simplest classical cipher.
Each letter is shifted by a fixed amount determined by the key.
But how would you break it if you intercepted a Caesar-encrypted message?

\begin{frame}
  \begin{exercise}
    You intercept a Caesar-encrypted message, but you do not know the key.
    How would you try to break it?
  \end{exercise}

  \ltnote{%
    Try-first: gives students a chance to propose brute force vs statistics.
    In discussion, collect both and use them to motivate the concept of
    \enquote{keyspace} and \enquote{structure leakage}.
  }
\end{frame}

There are two main approaches to breaking the Caesar cipher.
First, \emph{brute force}: since there are only \(n\) possible keys (26 for 
English, 29 for Swedish), an attacker can simply try them all.
Second, \emph{frequency analysis}: since the cipher merely rotates the 
frequency distribution of letters, the most common ciphertext letter likely 
corresponds to the most common plaintext letter (E in English, A in Swedish).
Once a single letter is guessed correctly, the key is revealed.

\mode<presentation>{%
\begin{frame}
  \begin{block}{Cryptanalysis of Caesar}
    \begin{itemize}
      \item Brute force: try all keys (tiny keyspace).
      \item Statistics: letter frequencies are preserved (just ``rotated'').
      \item If we guess one plaintext letter \(\alpha\) maps to ciphertext
        letter \(\beta\), then \(k = \beta - \alpha \bmod n\).
    \end{itemize}
  \end{block}
\end{frame}
}

\begin{frame}
\begin{figure}
  \begin{sidecaption}{%
    Polar plot of letter frequencies in plaintext (inner) and Caesar-encrypted
    ciphertext (outer). The distribution is simply rotated by the key value,
    preserving the characteristic pattern.
  }[fig:polar-caesar]
    \flushright
    \includegraphics[width=0.8\linewidth]{figs/polar-caesar.png}
  \end{sidecaption}
\end{figure}
\end{frame}

The \emph{substitution cipher} generalizes the shift cipher by allowing an 
arbitrary permutation of the alphabet.
The keyspace is now \(n!\) (approximately \(4 \times 10^{26}\) for 26 
letters), which is astronomically large.
Surely this is secure?

\begin{frame}
  \begin{definition}[Substitution cipher]
    We define it as follows:
    \begin{itemize}
      \item Key: a permutation \(\sigma\) of the alphabet.
      \item Encryption: \(c_i = \sigma(m_i)\).
      \item Decryption: \(m_i = \sigma^{-1}(c_i)\).
    \end{itemize}
  \end{definition}
\end{frame}

Unfortunately, the substitution cipher is still vulnerable to frequency 
analysis.
The letter E appears about 13\% of the time in English text, and this 
frequency is preserved in the ciphertext---just under a different letter.
Digrams (two-letter combinations like TH) and trigrams (THE) provide 
additional structure.
An experienced cryptanalyst can break a substitution cipher on a 
moderate-length text by hand.

The lesson is profound: \emph{a large keyspace is necessary but not 
sufficient for security}.
If the cipher leaks structure, it can be broken regardless of how many keys 
exist.

\mode<presentation>{%
\begin{frame}
  \begin{block}{Cryptanalysis: structure still leaks}
    \begin{itemize}
      \item Single-letter frequencies leak (A/E/T/... are uneven).
      \item Digrams/trigrams (``th'', ``the'', ...) give more clues.
      \item A large keyspace is not enough if the scheme leaks structure.
    \end{itemize}
  \end{block}
\end{frame}
}

\begin{frame}
\begin{figure}
  \begin{sidecaption}{%
    Frequency analysis of a substitution cipher.
    Even though individual letters are permuted, the overall frequency
    distribution remains, allowing an attacker to match ciphertext letters to
    their likely plaintext counterparts.
  }[fig:freq-caesar]
    \flushright
    \includegraphics[width=0.9\linewidth]{figs/freq-caesar.png}
  \end{sidecaption}
\end{figure}
\end{frame}

The \emph{Vigenère cipher} attempts to defeat frequency analysis by using a 
repeating key.
Instead of a single shift, a sequence of shifts \(k_0, k_1, \ldots, 
k_{\ell-1}\) is applied cyclically.
This was considered unbreakable for centuries and was called \enquote{le 
chiffre indéchiffrable} (the unbreakable cipher).

\begin{frame}
  \begin{definition}[Vigénère cipher]
    We define it as follows:
    \begin{itemize}
      \item Key: a short sequence \(k_0,\ldots,k_{\ell-1}\) reused periodically.
      \item Encryption: \(c_i = m_i + k_{i\bmod\ell} \bmod n\).
      \item Decryption: \(m_i = c_i - k_{i\bmod\ell} \bmod n\).
    \end{itemize}
  \end{definition}
\end{frame}

The Vigénère cipher fell to cryptanalysis in the 19th century.
The key insight is that if you can determine the key length \(\ell\), the 
problem reduces to \(\ell\) independent Caesar ciphers.
The key length can be guessed by looking for repeating patterns in the 
ciphertext (Kasiski examination) or by testing different lengths and checking 
whether the resulting columns have language-like frequency distributions 
(index of coincidence).

\mode<presentation>{%
\begin{frame}
  \begin{block}{Cryptanalysis of Vigénère (simple version)}
    \begin{itemize}
      \item If we guess the key length \(\ell\), then we split the ciphertext
        into \(\ell\) columns.
      \item Each column is a Caesar cipher with its own shift.
      \item Try plausible \(\ell\) and test for language-like statistics.
    \end{itemize}
  \end{block}
\end{frame}
}

The \emph{Hill cipher}, invented by Lester Hill in 1929, takes a different 
approach.
Instead of substituting individual letters, it operates on blocks of letters 
using matrix multiplication.
This \emph{diffuses} the statistics across the block, defeating simple 
frequency analysis.

\begin{frame}
  \begin{definition}[Hill cipher]
    We define it as follows:
    \begin{itemize}
      \item Key: an invertible \(m \times m\) matrix \(K\) over \(\ZZ_n\).
      \item Plaintext: a vector \(\mathbf{p} = (p_1, \ldots, p_m)^T\) of \(m\) 
        letters.
      \item Encryption: \(\mathbf{c} = K \cdot \mathbf{p} \bmod n\).
      \item Decryption: \(\mathbf{p} = K^{-1} \cdot \mathbf{c} \bmod n\).
    \end{itemize}
  \end{definition}

  \pause

  \begin{remark}
    This hides individual frequencies.
    This fixes the problem that broke Vigénère.
  \end{remark}
\end{frame}

The Hill cipher successfully hides single-letter frequencies: a single 
plaintext letter affects multiple ciphertext letters, and vice versa.
However, it has a fatal flaw: it is \emph{linear}.
This makes it vulnerable to \emph{known-plaintext attacks}.

\begin{frame}
  \begin{block}{Known-plaintext attack on Hill cipher}
    \begin{itemize}
      \item Attacker knows \(m\) plaintext-ciphertext pairs: 
        \((\mathbf{p}_1, \mathbf{c}_1), \ldots, (\mathbf{p}_m, \mathbf{c}_m)\).
      \item Form matrices \(P = [\mathbf{p}_1 \cdots \mathbf{p}_m]\) and 
        \(C = [\mathbf{c}_1 \cdots \mathbf{c}_m]\).
      \item Then \(C = K \cdot P\), so \(K = C \cdot P^{-1} \bmod n\).
    \end{itemize}
  \end{block}
\end{frame}

If an attacker knows \(m\) plaintext-ciphertext block pairs (where \(m\) is 
the block size), they can form matrices and solve a system of linear equations 
to recover the key matrix directly.
This attack is practical whenever the attacker can guess or obtain even a 
small amount of plaintext---and in practice, messages often have predictable 
headers, signatures, or standard phrases.

\subsubsection{Attack models}

To analyze cipher security rigorously, we must specify what capabilities we 
grant the adversary.
Different attack models capture different realistic scenarios.

\begin{frame}
  \begin{exercise}
    We broke the classical ciphers using only the ciphertext (frequency 
    analysis).
    But what other information might an attacker realistically have access to?
  \end{exercise}

  \ltnote{%
    Try-first: students brainstorm adversary capabilities before seeing the 
    formal attack model hierarchy.
    This surfaces intuitions about real-world scenarios (known headers, chosen 
    messages, oracle access) that motivate the definitions.
  }
\end{frame}

\begin{frame}
  \begin{block}{Attack models}
    \begin{description}
      \item[Ciphertext-only] Attacker sees only ciphertexts.
      \item[Known-plaintext] Attacker knows some plaintext-ciphertext pairs.
      \item[Chosen-plaintext] Attacker can choose plaintexts and obtain their 
        encryptions.
      \item[Chosen-ciphertext] Attacker can also choose ciphertexts and obtain 
        their decryptions.
    \end{description}
  \end{block}

  \ltnote{%
    \textbf{Variation pattern}: Generalization.

    \textbf{What varies}: The adversary's capabilities (what they can observe 
    or influence).

    \textbf{What remains invariant}: The goal (recover key or plaintext).

    \textbf{Critical aspect}: Security must be defined relative to an attack 
    model; a cipher secure against ciphertext-only attacks may fall to 
    chosen-plaintext attacks.
  }
\end{frame}

In a \emph{ciphertext-only attack}, the adversary sees only the encrypted 
messages.
This is the weakest attack model---if a cipher is insecure here, it is 
useless.
The classical ciphers we studied (Caesar, substitution, Vigenère) all fall to 
ciphertext-only attacks.

In a \emph{known-plaintext attack}, the adversary knows some 
plaintext-ciphertext pairs.
This is realistic: messages often have predictable headers, signatures, or 
standard phrases.
As we saw with the Hill cipher, this can be devastating for ciphers with 
algebraic structure.

In \emph{chosen-plaintext} and \emph{chosen-ciphertext} attacks, the 
adversary has even more power: they can request encryptions or decryptions of 
messages of their choice.
These stronger models are relevant when the attacker can interact with a 
system (for example, sending crafted messages to a server).
Modern ciphers are designed to resist even adaptive chosen-ciphertext 
attacks, where the attacker can make queries before and after receiving the 
challenge ciphertext.

\subsection{Security properties and pitfalls}

\mode<presentation>{%
\begin{frame}
  \begin{definition}[Perfect secrecy]
    \begin{itemize}
      \item Random variables \(M\) (message) and \(C\) (ciphertext).
      \item Perfect secrecy means: \(\Pr(M=m\mid C=c)=\Pr(M=m)\) for all
        \(m,c\).
    \end{itemize}
  \end{definition}

  \pause

  \begin{remark}
    Perfect secrecy is possible (one-time pad), but expensive in key material.
  \end{remark}
\end{frame}

\begin{frame}
  \begin{block}{\Ac{OTP}}
    \begin{itemize}
      \item Key is as long as the message, uniformly random, used once.
      \item Gives perfect secrecy.
      \item Key distribution and key reuse are the practical obstacles.
    \end{itemize}
  \end{block}

  \begin{remark}
    What broke Vigénère was reusing the key!
  \end{remark}
\end{frame}

\begin{frame}
  \begin{block}{Key length and brute force}
    \begin{itemize}
      \item If the best attack is brute force, security grows with key length.
      \item Doubling key length squares brute-force work.
      \item Symmetric keys (e.g., 128--256 bits) can be enough in practice.
    \end{itemize}
  \end{block}
\end{frame}
}


\begin{frame}
  \begin{exercise}
    Alice sends Bob an encrypted message.

    An attacker flips a few bits in the ciphertext.

    What do you think Bob will notice when decrypting?
  \end{exercise}

  \ltnote{%
    Try-first: many students implicitly assume encryption implies
    tamper-detection. This question surfaces that misconception before we state
    the separation between confidentiality and integrity.
  }
\end{frame}

\begin{frame}
  \mode<presentation>{%
    \begin{block}{Encryption does not give integrity}
      \begin{itemize}
        \item Encryption hides content, but does not necessarily detect tampering.
        \item Bob can often decrypt a modified ciphertext into \enquote{garbage}
          without knowing it was modified.
        \item Integrity requires additional mechanisms (covered later).
      \end{itemize}
    \end{block}
  }
  \mode<article>{%
    \begin{block}{Encryption does not give integrity}
      The purpose of encryption is confidentiality: an eavesdropper should not
      learn the message.
      But confidentiality alone does not mean the receiver can detect if the
      ciphertext was modified.

      Many encryption schemes are \emph{malleable}: an attacker can change the
      ciphertext in a way that results in a predictable (or at least different)
      plaintext after decryption.
      To get tamper detection we need integrity protection, typically by using a
      \ac{MAC} or an authenticated-encryption scheme.
    \end{block}
  }
\end{frame}

The consequences of bit flipping depend on the cipher's structure.
In \ac{SPN}-style block ciphers like \ac{AES}, the diffusion property means 
that flipping a single bit in the ciphertext affects many bits in the 
decrypted plaintext---the result is essentially random garbage.
The attacker cannot predict or control what the modified plaintext will be.

Stream ciphers, including the \ac{OTP} and block ciphers in \ac{CTR} mode, 
behave very differently.
Since encryption is simply XOR with a keystream, flipping a bit in the 
ciphertext flips exactly the corresponding bit in the plaintext.
This is called \emph{bit-by-bit malleability}: the attacker has precise 
control over which plaintext bits change.
If the attacker knows (or can guess) part of the plaintext, they can modify 
it to any value they choose.

\begin{frame}
  \begin{block}{Bit flipping: block ciphers vs stream ciphers}
    \begin{description}
      \item[\Ac{SPN} block ciphers (\acs{AES})] Diffusion spreads the effect.
        Flipping one ciphertext bit corrupts many plaintext bits 
        unpredictably.
      \item[Stream ciphers (\acs{OTP}, \acs{CTR} mode)] No diffusion across 
        bits.
        Flipping one ciphertext bit flips exactly one plaintext bit.
        This is \emph{bit-by-bit malleability}.
    \end{description}
  \end{block}

  \pause

  \begin{remark}
    Neither detects tampering!
    \Ac{SPN} ciphers just make tampering less predictable, not impossible.
  \end{remark}
\end{frame}

\subsection{Building modern block ciphers}

The failures of classical ciphers teach us what properties a secure cipher 
needs.
Shannon identified two key principles: \emph{confusion} and \emph{diffusion}.

\begin{frame}
  \begin{block}{Shannon's principles: confusion and diffusion}
    \begin{description}
      \item[Confusion] The relationship between key and ciphertext should be 
        complex.
        Each ciphertext bit should depend on many key bits.
      \item[Diffusion] The relationship between plaintext and ciphertext should 
        spread information.
        Changing one plaintext bit should affect many ciphertext bits.
    \end{description}
  \end{block}
\end{frame}

Confusion makes it hard to deduce the key from the ciphertext.
Diffusion spreads the statistical structure of the plaintext across the 
ciphertext, defeating frequency analysis.
Modern block ciphers achieve both by composing simple operations in multiple 
rounds.

\begin{frame}
  \begin{block}{\Acp{SPN}}
    \begin{itemize}
      \item Alternate between substitution (S-boxes) and permutation (P-boxes).
      \item S-boxes provide confusion: nonlinear mappings.
      \item P-boxes provide diffusion: rearrange bit positions.
      \item Repeat for many rounds to achieve security.
    \end{itemize}
  \end{block}

  \ltnote{%
    \textbf{Variation pattern}: Generalization.

    \textbf{What varies}: The specific S-boxes and P-boxes used.

    \textbf{What remains invariant}: The \ac{SPN} structure.

    \textbf{Critical aspect}: Students should discern that modern ciphers 
    \emph{compose} simple building blocks rather than relying on one clever 
    trick.
  }
\end{frame}

\begin{frame}
  \begin{figure}
    \begin{sidecaption}{%
      Structure of a \acl{SPN}.
      Each round applies substitution (S-boxes) for confusion and permutation 
      (P-boxes) for diffusion, with round keys mixed in.
    }[fig:spn]
      \includegraphics[width=0.7\linewidth]{figs/spn.png}
    \end{sidecaption}
  \end{figure}
\end{frame}

The \ac{AES} is the most widely used block cipher today.
It uses a 128-bit block size and supports key sizes of 128, 192, or 256 bits.
\Ac{AES} is an \ac{SPN}: each round applies a nonlinear substitution (SubBytes), 
a permutation (ShiftRows), a mixing operation (MixColumns), and a key addition 
(AddRoundKey).
After 10--14 rounds (depending on key size), the result is a ciphertext that 
appears random to anyone without the key.

\begin{frame}
  \begin{block}{\Ac{AES} at a glance}
    \begin{itemize}
      \item Block size: 128 bits.
      \item Key sizes: 128, 192, or 256 bits.
      \item Rounds: 10, 12, or 14 (depending on key size).
      \item Operations per round: SubBytes, ShiftRows, MixColumns, AddRoundKey.
    \end{itemize}
  \end{block}
\end{frame}

\subsection{Key size and security level}

How do we choose a key size?
The \emph{security level} of a cipher is the logarithm (base 2) of the number 
of operations needed to break it.
A cipher with \(n\)-bit security requires approximately \(2^n\) operations to 
break.

\begin{frame}
  \begin{block}{Security level}
    \begin{itemize}
      \item \(n\)-bit security means \(\approx 2^n\) operations to break.
      \item A 128-bit key gives (at most) 128-bit security.
      \item For brute force: \(2^{128}\) is astronomically large 
        (\(\approx 10^{38}\)).
    \end{itemize}
  \end{block}
\end{frame}

For a good cipher (one where brute force is the best attack), the security 
level equals the key length.
Doubling the key length squares the work for brute force.
In practice, 128-bit keys are sufficient for most applications, and 256-bit 
keys provide a comfortable margin.

But what about quantum computers?

\begin{frame}
  \begin{block}{Quantum threat: Grover's algorithm}
    \begin{itemize}
      \item Grover's algorithm searches \(N\) items in \(O(\sqrt{N})\) quantum 
        operations.
      \item For brute-forcing a key: \(2^n \to 2^{n/2}\) operations.
      \item A 256-bit key provides only 128-bit security against quantum attack.
      \item \Ac{AES}-256 is quantum-safe at the 128-bit level.
    \end{itemize}
  \end{block}

  \ltnote{%
    \textbf{Variation pattern}: Contrast.

    \textbf{What varies}: The computational model (classical vs quantum).

    \textbf{What remains invariant}: The cipher itself (\ac{AES}-256).

    \textbf{Critical aspect}: The security level depends not just on the cipher 
    but on the adversary's computational capabilities.
  }
\end{frame}

Grover's algorithm provides a quadratic speedup for unstructured search.
This means a quantum computer can brute-force an \(n\)-bit key in 
\(O(2^{n/2})\) operations instead of \(O(2^n)\).
Consequently, to achieve 128-bit security against quantum adversaries, we need 
256-bit keys.
\Ac{AES}-256 is therefore considered quantum-safe at the 128-bit security level.

For asymmetric cryptography, the situation is worse: Shor's algorithm breaks 
RSA and elliptic-curve cryptography entirely.
This is why post-quantum cryptography is an active research area, but for 
symmetric ciphers, simply doubling the key size suffices.


\section{Block modes}

A block cipher encrypts fixed-size blocks (for example, 128 bits for \ac{AES}).
But real messages are rarely exactly one block long.
How do we encrypt longer messages?
The answer is a \emph{mode of operation}: a method for applying the block 
cipher to a sequence of blocks.

Unlike the \ac{OTP}, a block cipher like \ac{AES} is designed to be used with 
the \emph{same key} for multiple blocks and multiple messages.
This is what makes it practical: a short key (128--256 bits) can protect 
gigabytes of data.
But this reuse comes with risks that must be managed carefully through the 
choice of mode.

The choice of mode matters enormously.
A poorly chosen mode can leak information even if the underlying block cipher 
is perfectly secure.

\subsection{Why modes?}

\mode<presentation>{%
\begin{frame}
  \begin{block}{Block ciphers need a mode of operation}
    \begin{itemize}
      \item Real messages are longer than one block.
      \item A \emph{mode} specifies how to encrypt many blocks.
      \item A bad mode can leak patterns even if the block cipher is strong.
    \end{itemize}
  \end{block}
\end{frame}
}

Consider what happens if we simply encrypt each block independently.

\begin{frame}
  \begin{exercise}
    Suppose you encrypt a bitmap image block-by-block.

    What would you expect the ciphertext image to look like if you encrypt each
    block independently with the same key?
  \end{exercise}

  \ltnote{%
    Try-first before \ac{ECB}: students can reason about repetition/patterns 
    without knowing the term \enquote{\ac{ECB}}. This sets up the contrast that 
    motivates modes.
  }
\end{frame}

This naive approach is called \ac{ECB} mode.
Each plaintext block is encrypted independently to produce a ciphertext block.
The problem is that identical plaintext blocks produce identical ciphertext 
blocks.
Any structure in the plaintext---repeated blocks, patterns, regions of 
uniform data---is preserved in the ciphertext.

\mode<presentation>{%
\begin{frame}
  \begin{block}{\Ac{ECB} leaks patterns}
    \begin{itemize}
      \item Each block encrypted independently.
      \item Equal plaintext blocks \(\Rightarrow\) equal ciphertext blocks.
      \item Structure in the plaintext becomes visible.
    \end{itemize}
  \end{block}
\end{frame}
}

The famous \enquote{\ac{ECB} penguin} demonstrates this dramatically.
When a bitmap image of the Linux mascot Tux is encrypted with \ac{ECB} mode, 
the penguin is still clearly visible in the ciphertext.
With a proper mode like \ac{CTR}, the ciphertext appears random.

\begin{frame}
  \begin{figure}
    \begin{sidecaption}[Tux encrypted using different modes.]{%
      Tux encrypted using different block modes of operation.
      When using \ac{ECB} mode, we can still distinguish Tux.
      With \ac{CTR}, the ciphertext looks random.
    }[fig:EncModes]
    \hfill
    \begin{subfigure}[t]{0.3\linewidth}
      \includegraphics[width=\linewidth]{../../../project/figs/Tux.pdf}
      \caption{Original Tux.
      Image by: Larry Ewing, Simon Budig, Garrett LeSage.}
    \end{subfigure}
    \hfill
    \begin{subfigure}[t]{0.3\linewidth}
      \includegraphics[width=\linewidth]{../../../project/figs/Tux_encrypted_ecb.png}
      \caption{\Ac{ECB}-encrypted Tux.
      Image by: RFL890.}
    \end{subfigure}
    \hfill
    \begin{subfigure}[t]{0.3\linewidth}
      \includegraphics[width=\linewidth]{../../../project/figs/Tux_encrypted_ctr.png}
      \caption{\Ac{CTR}-encrypted Tux.
      Image by: RFL890.}
    \end{subfigure}
    \hspace{\fill}
    \end{sidecaption}
  \end{figure}

  \ltnote{%
    Core contrast example: keep key/plaintext invariant and vary only the mode
    (\ac{ECB} vs \ac{CTR}).
  }
\end{frame}

\subsection{Common modes (properties)}

There are many modes of operation, each with different properties.
Rather than detailing their internal workings, we focus on the properties that 
matter for practical use: how randomness is introduced, whether operations can 
be parallelized, how errors propagate, and whether random access is possible.

% explain how they function, what they do to achieve this.

\begin{frame}
  \mode<presentation>{%
    \begin{block}{\Ac{CBC}, \ac{CTR}, \ac{CFB}, \ac{OFB}: what differs?}
      \begin{itemize}
        \item How randomness enters (\ac{IV}/nonce).
        \item Whether encryption can be parallelised.
        \item Error propagation (a bit flip affects how much?).
        \item Random access: can you decrypt block \(i\) without earlier blocks?
      \end{itemize}
    \end{block}
  }
  \mode<article>{%
    \begin{block}{\Ac{CBC}, \ac{CTR}, \ac{CFB}, \ac{OFB}: what differs?}
      These modes combine a block cipher with an \ac{IV}/nonce and feedback in
      different ways.
      We focus on high-level properties: which misuse breaks confidentiality,
      and which modes are convenient for implementation (parallelism, random
      access, and error propagation).
    \end{block}
  }

  \ltnote{%
    We avoid construction details and focus on properties that students need for
    safe use.
  }
\end{frame}

\subsection{\Acl{CBC}}

\Ac{CBC} mode chains the encryption of each block to the previous one.
The first plaintext block is XORed with a random \ac{IV} before encryption; 
each subsequent block is XORed with the previous ciphertext block.
Mathematically:
\begin{align*}
  C_0 &= \Enc_k(P_0 \oplus \text{IV}) \\
  C_i &= \Enc_k(P_i \oplus C_{i-1}) \quad \text{for } i \geq 1
\end{align*}
This chaining ensures that identical plaintext blocks produce different 
ciphertext blocks (unless they follow identical histories), defeating the 
pattern leakage of \ac{ECB}.

\begin{frame}
  \frametitle{\Ac{CBC} mode encryption}
  \begin{figure}
    \begin{sidecaption}{%
      \Ac{CBC} encryption: each plaintext block is XORed with the previous 
      ciphertext before encryption.
    }[fig:cbc-enc]
      \flushright
      % CBC encryption diagram
\begin{tikzpicture}[
    >=Stealth,
    box/.style={draw, minimum width=1.2cm, minimum height=0.8cm},
    xor/.style={draw, circle, minimum size=0.5cm, inner sep=0pt},
    every node/.style={font=\small},
    node distance=0.8cm and 1.8cm
  ]
  
  % Block 0
  \node (p0) {$P_0$};
  \node[xor, below=of p0] (xor0) {$\oplus$};
  \node[box, below=of xor0] (enc0) {$\Enc_k$};
  \node[below=of enc0] (c0) {$C_0$};
  \node[left=1.2cm of xor0] (iv) {IV};
  
  % Block 1
  \node[right=of p0] (p1) {$P_1$};
  \node[xor, below=of p1] (xor1) {$\oplus$};
  \node[box, below=of xor1] (enc1) {$\Enc_k$};
  \node[below=of enc1] (c1) {$C_1$};
  
  % Block 2
  \node[right=of p1] (p2) {$P_2$};
  \node[xor, below=of p2] (xor2) {$\oplus$};
  \node[box, below=of xor2] (enc2) {$\Enc_k$};
  \node[below=of enc2] (c2) {$C_2$};
  
  % Arrows for block 0
  \draw[->] (p0) -- (xor0);
  \draw[->] (iv) -- (xor0);
  \draw[->] (xor0) -- (enc0);
  \draw[->] (enc0) -- (c0);
  
  % Arrows for block 1
  \draw[->] (p1) -- (xor1);
  \draw[->] (xor1) -- (enc1);
  \draw[->] (enc1) -- (c1);
  
  % Arrows for block 2
  \draw[->] (p2) -- (xor2);
  \draw[->] (xor2) -- (enc2);
  \draw[->] (enc2) -- (c2);
  
  % Chaining arrows (from ciphertext to next XOR)
  \draw[->] ($(enc0.south)!0.5!(c0.north)$) -| ($(xor1.west)$);
  \draw[->] ($(enc1.south)!0.5!(c1.north)$) -| ($(xor2.west)$);
  
\end{tikzpicture}

    \end{sidecaption}
  \end{figure}
\end{frame}

Decryption reverses the process: each ciphertext block is decrypted, then 
XORed with the previous ciphertext block (or the \ac{IV} for the first block) 
to recover the plaintext:
\begin{align*}
  P_0 &= \Dec_k(C_0) \oplus \text{IV} \\
  P_i &= \Dec_k(C_i) \oplus C_{i-1} \quad \text{for } i \geq 1
\end{align*}

\begin{frame}
  \frametitle{\Ac{CBC} mode decryption}
  \begin{figure}
    \begin{sidecaption}{%
      \Ac{CBC} decryption: each ciphertext block is decrypted, then XORed with 
      the previous ciphertext.
    }[fig:cbc-dec]
      \flushright
      % CBC decryption diagram
\begin{tikzpicture}[
    >=Stealth,
    box/.style={draw, minimum width=1.2cm, minimum height=0.8cm},
    xor/.style={draw, circle, minimum size=0.5cm, inner sep=0pt},
    every node/.style={font=\small},
    node distance=0.8cm and 1.8cm
  ]
  
  % Block 0
  \node (c0) {$C_0$};
  \node[box, below=of c0] (dec0) {$\Dec_k$};
  \node[xor, below=of dec0] (xor0) {$\oplus$};
  \node[below=of xor0] (p0) {$P_0$};
  \node[left=1.2cm of xor0] (iv) {IV};
  
  % Block 1
  \node[right=of c0] (c1) {$C_1$};
  \node[box, below=of c1] (dec1) {$\Dec_k$};
  \node[xor, below=of dec1] (xor1) {$\oplus$};
  \node[below=of xor1] (p1) {$P_1$};
  
  % Block 2
  \node[right=of c1] (c2) {$C_2$};
  \node[box, below=of c2] (dec2) {$\Dec_k$};
  \node[xor, below=of dec2] (xor2) {$\oplus$};
  \node[below=of xor2] (p2) {$P_2$};
  
  % Arrows for block 0
  \draw[->] (c0) -- (dec0);
  \draw[->] (dec0) -- (xor0);
  \draw[->] (iv) -- (xor0);
  \draw[->] (xor0) -- (p0);
  
  % Arrows for block 1
  \draw[->] (c1) -- (dec1);
  \draw[->] (dec1) -- (xor1);
  \draw[->] (xor1) -- (p1);
  
  % Arrows for block 2
  \draw[->] (c2) -- (dec2);
  \draw[->] (dec2) -- (xor2);
  \draw[->] (xor2) -- (p2);
  
  % Chaining arrows (from ciphertext to next XOR)
  \draw[->] ($(c0.south)!0.5!(dec0.north)$) -| ($(xor1.west)$);
  \draw[->] ($(c1.south)!0.5!(dec1.north)$) -| ($(xor2.west)$);
  
\end{tikzpicture}

    \end{sidecaption}
  \end{figure}
\end{frame}

The \ac{IV} in \ac{CBC} must be \emph{unpredictable}, not merely unique.
If an attacker can predict the \ac{IV}, they can craft plaintexts that interact 
with it in controlled ways, leading to chosen-plaintext attacks.
In practice, the \ac{IV} is typically generated randomly and transmitted 
alongside the ciphertext.

\begin{frame}
  \begin{block}{\Ac{CBC} --- properties}
    \begin{itemize}
      \item Uses a random/unpredictable \ac{IV}.
      \item Hides patterns across blocks.
      \item Encryption is sequential (depends on previous ciphertext).
      \item Bit flips affect the current block and the next block.
    \end{itemize}
  \end{block}
\end{frame}

\subsection{\Acl{CTR}}

\Ac{CTR} mode takes a completely different approach: it turns the block cipher 
into a stream cipher.
Instead of encrypting the plaintext directly, the cipher encrypts a sequence 
of counter values to produce a \emph{keystream}, which is then XORed with the 
plaintext:
\[
  C_i = P_i \oplus \Enc_k(\text{nonce} \| i)
\]
The nonce (number used once) is a value that must be unique for each message 
encrypted under the same key; the counter \(i\) increments for each block 
within a message.

\begin{frame}
  \frametitle{\Ac{CTR} mode encryption}
  \begin{figure}
    \begin{sidecaption}{%
      \Ac{CTR} encryption: the block cipher encrypts counter values to produce 
      a keystream, which is XORed with the plaintext.
    }[fig:ctr-enc]
      \flushright
      \input{figs/ctr-enc.tex}
    \end{sidecaption}
  \end{figure}
\end{frame}

Decryption in \ac{CTR} mode is identical to encryption: generate the same 
keystream and XOR it with the ciphertext.
Since XOR is its own inverse, this recovers the plaintext:
\[
  P_i = C_i \oplus \Enc_k(\text{nonce} \| i)
\]
Note that decryption uses the \emph{encryption} function of the block cipher, 
not the decryption function.
This means \ac{CTR} mode only requires implementing the encryption direction.

\begin{frame}
  \frametitle{\Ac{CTR} mode decryption}
  \begin{figure}
    \begin{sidecaption}{%
      \Ac{CTR} decryption: identical to encryption---the same keystream is 
      XORed with the ciphertext.
    }[fig:ctr-dec]
      \flushright
      % CTR decryption diagram
\begin{tikzpicture}[
    >=Stealth,
    box/.style={draw, minimum width=1.2cm, minimum height=0.8cm},
    xor/.style={draw, circle, minimum size=0.5cm, inner sep=0pt},
    every node/.style={font=\small},
    node distance=0.8cm and 1.8cm
  ]
  
  % Block 0
  \node (nonce0) {nonce $\|$ 0};
  \node[box, below=of nonce0] (enc0) {$\Enc_k$};
  \node[xor, below=of enc0] (xor0) {$\oplus$};
  \node[below=of xor0] (p0) {$P_0$};
  \node[left=1.2cm of xor0] (c0) {$C_0$};
  
  % Block 1
  \node[right=of nonce0] (nonce1) {nonce $\|$ 1};
  \node[box, below=of nonce1] (enc1) {$\Enc_k$};
  \node[xor, below=of enc1] (xor1) {$\oplus$};
  \node[below=of xor1] (p1) {$P_1$};
  \node[left=1.2cm of xor1] (c1) {$C_1$};
  
  % Block 2
  \node[right=of nonce1] (nonce2) {nonce $\|$ 2};
  \node[box, below=of nonce2] (enc2) {$\Enc_k$};
  \node[xor, below=of enc2] (xor2) {$\oplus$};
  \node[below=of xor2] (p2) {$P_2$};
  \node[left=1.2cm of xor2] (c2) {$C_2$};
  
  % Arrows for block 0
  \draw[->] (nonce0) -- (enc0);
  \draw[->] (enc0) -- (xor0);
  \draw[->] (c0) -- (xor0);
  \draw[->] (xor0) -- (p0);
  
  % Arrows for block 1
  \draw[->] (nonce1) -- (enc1);
  \draw[->] (enc1) -- (xor1);
  \draw[->] (c1) -- (xor1);
  \draw[->] (xor1) -- (p1);
  
  % Arrows for block 2
  \draw[->] (nonce2) -- (enc2);
  \draw[->] (enc2) -- (xor2);
  \draw[->] (c2) -- (xor2);
  \draw[->] (xor2) -- (p2);
  
\end{tikzpicture}

    \end{sidecaption}
  \end{figure}
\end{frame}

Because each block's keystream depends only on the nonce and counter (not on 
other blocks), \ac{CTR} mode is fully parallelizable and supports random 
access: you can decrypt block \(i\) without processing blocks \(0\) through 
\(i-1\).
However, the nonce must \emph{never} be reused with the same key.

\begin{frame}
  \begin{block}{\Ac{CTR} --- properties}
    \begin{itemize}
      \item Uses a unique nonce (and counter) to create a keystream.
      \item Parallelisable and supports random access.
      \item Bit flips in ciphertext flip the corresponding plaintext bits.
      \item Nonce reuse is catastrophic.
    \end{itemize}
  \end{block}
\end{frame}

\begin{frame}
  \mode<presentation>{%
    \begin{block}{\Ac{CFB} and \ac{OFB} --- properties}
      \begin{itemize}
        \item Turn a block cipher into a stream-cipher-like scheme.
        \item \Ac{CFB} is self-synchronising; \ac{OFB} is not.
        \item Both require \ac{IV}/nonce discipline to avoid reuse.
      \end{itemize}
    \end{block}
  }
  \mode<article>{%
    \begin{block}{\Ac{CFB} and \ac{OFB} --- properties}
      \Ac{CFB} and \ac{OFB} are older modes that effectively use a block cipher 
      to generate a keystream.
      They are less common in modern protocols than \ac{CTR} and \ac{AEAD} 
      modes, but they illustrate the design space: how feedback affects 
      synchronisation and error propagation.
    \end{block}
  }
\end{frame}

\subsection{\Acsp{IV}/nonces and misuse}

All secure modes require some form of randomization to avoid deterministic 
encryption.
This is provided by an \ac{IV} or \emph{nonce}.
The exact requirements vary by mode: \ac{CBC} needs an unpredictable \ac{IV}, 
while \ac{CTR} only requires the nonce to be unique (never reused with the 
same key).

\begin{frame}
  \begin{block}{\Ac{IV}/nonce: what it is and why it matters}
    \begin{itemize}
      \item A public value used to randomise encryption.
      \item Must be unique (and sometimes unpredictable), depending on the mode.
      \item Reuse can destroy confidentiality.
    \end{itemize}
  \end{block}
\end{frame}

\begin{frame}
  \begin{exercise}
    In \ac{CTR} mode, encryption looks like
    \(c = m \oplus \text{keystream}(k,\text{nonce})\).

    What do you think happens if the same nonce is reused for two different
    messages?
  \end{exercise}

  \ltnote{%
    Try-first: students can compute \(c_1\oplus c_2\) themselves and discover
    the key invariant (same keystream) before we state the rule.
  }
\end{frame}

The consequences of nonce reuse in \ac{CTR} mode are immediate and devastating.
If two messages \(m_1\) and \(m_2\) are encrypted with the same nonce, then 
\(c_1 \oplus c_2 = m_1 \oplus m_2\).
This XOR of plaintexts often contains enough structure to recover both 
messages, especially if one is partially known or follows a predictable 
format.

\mode<presentation>{%
\begin{frame}
  \begin{block}{\Ac{CTR} nonce reuse: the disaster in one line}
    \begin{itemize}
      \item \Ac{CTR} gives: \(c = m \oplus \text{keystream}(k,\text{nonce})\).
      \item If nonce reused: \(c_1\oplus c_2 = m_1\oplus m_2\).
      \item This leaks relations between messages and is often enough to recover
        both.
    \end{itemize}
  \end{block}
\end{frame}
}

\begin{frame}
  \begin{exercise}
    What must be true about \ac{IV}/nonce values for repeated encryptions under 
    the same key?
  \end{exercise}

  \ltnote{%
    Post-test: students should now articulate \ac{IV}/nonce uniqueness without 
    being cued by a specific scenario.
  }
\end{frame}

The lesson is clear: \ac{IV}/nonce discipline is not optional---it is part of 
the 
security definition.
Failing to maintain uniqueness can completely compromise confidentiality, 
regardless of how strong the underlying cipher is.

In practice, we rarely use bare encryption.
We usually want \emph{authenticated encryption}, which provides both 
confidentiality and integrity in a single operation.
Modes like \ac{GCM} are widely used for this purpose.
For specialized applications like disk encryption, modes like \ac{XTS} are 
designed to handle the unique constraints (random access, no per-sector 
storage for nonces).

\begin{frame}
  \mode<presentation>{%
    \begin{block}{Later: authenticated encryption and storage modes}
      \begin{itemize}
        \item \Ac{AEAD}: encryption + integrity in one scheme.
        \item \Ac{GCM}: widely used \ac{AEAD} mode.
        \item \Ac{XTS}: disk/storage encryption mode.
      \end{itemize}
    \end{block}
  }
  \mode<article>{%
    \begin{block}{Later: authenticated encryption and storage modes}
      In practice we rarely use \enquote{bare} confidentiality.
      We often need authenticated encryption, where the receiver can both 
      decrypt and detect tampering.
      This is commonly provided by \ac{AEAD} schemes, such as \ac{GCM}.

      For disk encryption the requirements are different (random access, no
      per-sector nonces), and specialised modes such as \ac{XTS} are used.
      We will return to these later.
    \end{block}
  }
\end{frame}

\mode<presentation>{%
\begin{frame}
  \begin{block}{Take-away}
    \begin{itemize}
      \item Modes matter: \ac{ECB} is not acceptable for structured data.
      \item Pick a mode with good properties for your setting (parallelism,
        random access, error propagation).
      \item \Ac{IV}/nonce discipline is part of the security definition.
    \end{itemize}
  \end{block}
\end{frame}

\begin{frame}
  \begin{center}
    Questions?
  \end{center}
\end{frame}
}

\section{Summary}

\begin{frame}
  \begin{block}{Symmetric encryption in one line}
    \begin{itemize}
      \item Same key for encryption and decryption.
      \item Main practical challenge: \emph{key distribution}.
    \end{itemize}
  \end{block}
\end{frame}


