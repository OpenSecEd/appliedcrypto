\title{Ciphers and shared-key encryption}
\author{Daniel Bosk}
\institute{KTH EECS}

\mode*

\begin{frame}
  \maketitle
\end{frame}

\begin{abstract}
  Public-key cryptography has a central role in modern cryptographic solutions, 
it ranges from HTTPS to electronic ID systems.

The \acp{ILO} are that you should be able
\begin{itemize}
  \item to \emph{apply} various public-key cryptography primitives in 
    situations based on their high-level properties.
\end{itemize}

The basics are covered by
Chapter 5 in Anderson's \citetitle{Anderson2008sea}~\cite{Anderson2008sea} and
Chapter 14 in Gollmann's \citetitle{Gollmann2011cs}~\cite{Gollmann2011cs}.
Furthermore, we refer to the \citetitle{EOCS}~\cite{EOCS}.

\end{abstract}

\clearpage

\begin{frame}
  \only<presentation>{\tableofcontents[hideallsubsections]}
  \only<article>{\tableofcontents}
\end{frame}

\clearpage

\section{Background and principles}

\subsection{What is cryptography?}

\begin{frame}
  \begin{exercise}
    When you hear \enquote{cryptography}, what do you think it does?
    Take 30 seconds and write down a one-sentence answer.
  \end{exercise}

  \ltnote{%
    Try-first: activate prior knowledge before terminology.
    The contrast between student intuition and the definitions below makes the
    scope (contents vs existence) discernible.
  }
\end{frame}

\begin{frame}
  \begin{itemize}
    \item Cryptography aims to protect information against an adversary.
    \item Steganography hides \emph{existence}; cryptography hides \emph{contents}.
    \item Today: mostly a science (definitions, models, proofs), not
      \enquote{clever} tricks.
  \end{itemize}
\end{frame}

\begin{frame}
  \begin{block}{Kerckhoffs' principle}
    \begin{itemize}
      \item No security-by-obscurity.
      \item The \emph{key} should be the only secret.
    \end{itemize}
  \end{block}
\end{frame}

\section{Ciphers (shared-key encryption)}

\subsection{The interface and the goal}

\begin{frame}
  \begin{idea}
    \begin{itemize}
      \item Alice and Bob share a (small) secret key \(k\).
      \item Alice computes ciphertext \(c = \Enc_k(m)\) and sends \(c\).
      \item Bob computes \(m = \Dec_k(c)\).
      \item An eavesdropper should learn (essentially) nothing about \(m\) from
        \(c\).
    \end{itemize}
  \end{idea}
\end{frame}

\begin{frame}
  \begin{block}{Symmetric encryption in one line}
    \begin{itemize}
      \item Same key for encryption and decryption.
      \item Main practical challenge: \emph{key distribution}.
    \end{itemize}
  \end{block}
\end{frame}

\begin{frame}
  \begin{block}{Block-cipher encryption}
    \begin{description}
      \item[Input] key \(k\), plaintext block \(p\).
      \item[Output] ciphertext block \(c\).
      \item[Notation] \(\Enc[_k][p] = c\)
    \end{description}
  \end{block}

  \pause

  \begin{block}{Block-cipher decryption}
    \begin{description}
      \item[Input] key \(k\), ciphertext block \(c\).
      \item[Output] plaintext block \(p\).
      \item[Notation] \(\Dec[_k][c] = p\)
    \end{description}
  \end{block}
\end{frame}

\subsection{Why classical ciphers fail}

\begin{frame}
  \begin{definition}[Shift cipher (Caesar)]
    \begin{itemize}
      \item Think of letters as integers modulo \(n\) (e.g., \(n=26\) or
        include space).
      \item Key \(k\in\mathbb{Z}_n\).
      \item \(\Enc[_k][m] = (m+k)\bmod n\), \(\Dec[_k][c] = (c-k)\bmod n\).
    \end{itemize}
  \end{definition}
\end{frame}

\begin{frame}
  \begin{exercise}
    You intercept a Caesar-encrypted message, but you do not know the key.

    How would you try to break it?
  \end{exercise}

  \ltnote{%
    Try-first: gives students a chance to propose brute force vs statistics.
    In discussion, collect both and use them to motivate the concept of
    \enquote{keyspace} and \enquote{structure leakage}.
  }
\end{frame}

\begin{frame}
  \begin{block}{Cryptanalysis of Caesar}
    \begin{itemize}
      \item Brute force: try all keys (tiny keyspace).
      \item Statistics: letter frequencies are preserved (just ``rotated'').
      \item If we guess one plaintext letter \(\alpha\) maps to ciphertext
        letter \(\beta\), then \(k = \beta - \alpha \bmod n\).
    \end{itemize}
  \end{block}
\end{frame}

\begin{frame}
  \begin{definition}[Substitution cipher]
    \begin{itemize}
      \item Key: a permutation \(\sigma\) of the alphabet.
      \item Encryption: \(c_i = \sigma(m_i)\).
      \item Decryption: \(m_i = \sigma^{-1}(c_i)\).
    \end{itemize}
  \end{definition}
\end{frame}

\begin{frame}
  \begin{block}{Cryptanalysis: structure still leaks}
    \begin{itemize}
      \item Single-letter frequencies leak (A/E/T/... are uneven).
      \item Digrams/trigrams (``th'', ``the'', ...) give more clues.
      \item A large keyspace is not enough if the scheme leaks structure.
    \end{itemize}
  \end{block}
\end{frame}

\begin{frame}
  \begin{definition}[Vigen\`ere cipher]
    \begin{itemize}
      \item Key: a short sequence \(k_0,\ldots,k_{\ell-1}\) reused periodically.
      \item Encryption: \(c_i = m_i + k_{i\bmod\ell} \bmod n\).
      \item Decryption: \(m_i = c_i - k_{i\bmod\ell} \bmod n\).
    \end{itemize}
  \end{definition}
\end{frame}

\begin{frame}
  \begin{block}{Cryptanalysis of Vigen\`ere (simple version)}
    \begin{itemize}
      \item If we guess the key length \(\ell\), then we split the ciphertext
        into \(\ell\) columns.
      \item Each column is a Caesar cipher with its own shift.
      \item Try plausible \(\ell\) and test for language-like statistics.
    \end{itemize}
  \end{block}
\end{frame}

\subsection{Security properties and pitfalls}

\begin{frame}
  \begin{definition}[Perfect secrecy]
    \begin{itemize}
      \item Random variables \(M\) (message) and \(C\) (ciphertext).
      \item Perfect secrecy means: \(\Pr(M=m\mid C=c)=\Pr(M=m)\) for all
        \(m,c\).
    \end{itemize}
  \end{definition}

  \pause

  \begin{remark}
    Perfect secrecy is possible (one-time pad), but expensive in key material.
  \end{remark}
\end{frame}

\begin{frame}
  \begin{block}{One-time pad (OTP)}
    \begin{itemize}
      \item Key is as long as the message, uniformly random, used once.
      \item Gives perfect secrecy.
      \item Key distribution and key reuse are the practical obstacles.
    \end{itemize}
  \end{block}
\end{frame}

\begin{frame}
  \begin{block}{Key length and brute force}
    \begin{itemize}
      \item If the best attack is brute force, security grows with key length.
      \item Doubling key length squares brute-force work.
      \item Symmetric keys (e.g., 128--256 bits) can be enough in practice.
    \end{itemize}
  \end{block}
\end{frame}

\begin{frame}
  \begin{block}{Block ciphers need a mode of operation}
    \begin{itemize}
      \item Real messages are longer than one block.
      \item A mode defines how to encrypt many blocks.
      \item Bad modes leak patterns.
    \end{itemize}
  \end{block}
\end{frame}

\begin{frame}
  \begin{exercise}
    Suppose you encrypt a bitmap image block-by-block.

    What would you expect the ciphertext image to look like if you encrypt each
    block independently with the same key?
  \end{exercise}

  \ltnote{%
    Try-first before ECB: students can reason about repetition/patterns without
    knowing the term \enquote{ECB}. This sets up the contrast that motivates
    modes of operation.
  }
\end{frame}

\begin{frame}
  \begin{block}{ECB (Electronic Codebook) leaks patterns}
    \begin{itemize}
      \item Each block encrypted independently.
      \item Equal plaintext blocks \(\Rightarrow\) equal ciphertext blocks.
      \item Structure in the plaintext becomes visible.
    \end{itemize}
  \end{block}
\end{frame}

\begin{frame}
  \mode<presentation>{%
    \begin{block}{CBC and CTR avoid pattern leakage (high level)}
      \begin{itemize}
        \item CBC mixes each plaintext block with previous ciphertext.
        \item CTR uses a counter to create a keystream.
        \item Both rely on an \emph{IV/nonce} so encryptions of the same
          message look different.
      \end{itemize}
    \end{block}
  }
  \mode<article>{%
    \begin{block}{CBC and CTR avoid pattern leakage (high level)}
      The key idea is to make encryption of identical plaintext blocks produce
      different ciphertext blocks.
      CBC does this by chaining blocks (each block depends on the previous
      ciphertext), while CTR does this by XORing the plaintext with a keystream
      generated from a counter.

      In both cases we need an IV/nonce to ensure the chaining/keystream starts
      differently each time.
    \end{block}
  }

  \ltnote{%
    Article-mode prose explains the \emph{why} without diving into construction.
    Slides keep it short for narration.
  }
\end{frame}

\begin{frame}
  \begin{block}{IV/nonce: what it is and why it matters}
    \begin{itemize}
      \item A public value used to randomize encryption.
      \item Must be unique (and often unpredictable) depending on the mode.
      \item Reuse can destroy confidentiality.
    \end{itemize}
  \end{block}
\end{frame}

\begin{frame}
  \begin{exercise}
    In CTR mode, encryption looks like
    \(c = m \oplus \text{keystream}(k,\text{nonce})\).

    What do you think happens if the same nonce is reused for two different
    messages?
  \end{exercise}

  \ltnote{%
    Try-first: students can compute \(c_1\oplus c_2\) themselves and discover
    the key invariant (same keystream) before we state the rule.
  }
\end{frame}

\begin{frame}
  \begin{block}{CTR nonce reuse: the disaster in one line}
    \begin{itemize}
      \item CTR gives: \(c = m \oplus \text{keystream}(k,\text{nonce})\).
      \item If nonce reused: \(c_1\oplus c_2 = m_1\oplus m_2\).
      \item This leaks relations between messages and is often enough to recover
        both.
    \end{itemize}
  \end{block}
\end{frame}

\begin{frame}
  \begin{exercise}
    Alice sends Bob an encrypted message.

    An attacker flips a few bits in the ciphertext.

    What do you think Bob will notice when decrypting?
  \end{exercise}

  \ltnote{%
    Try-first: many students implicitly assume encryption implies
    tamper-detection. This question surfaces that misconception before we state
    the separation between confidentiality and integrity.
  }
\end{frame}

\begin{frame}
  \mode<presentation>{%
    \begin{block}{Encryption does not give integrity}
      \begin{itemize}
        \item Encryption hides content, but does not necessarily detect tampering.
        \item Bob can often decrypt a modified ciphertext into \enquote{garbage}
          without knowing it was modified.
        \item Integrity requires additional mechanisms (covered later).
      \end{itemize}
    \end{block}
  }
  \mode<article>{%
    \begin{block}{Encryption does not give integrity}
      The purpose of encryption is confidentiality: an eavesdropper should not
      learn the message.
      But confidentiality alone does not mean the receiver can detect if the
      ciphertext was modified.

      Many encryption schemes are \emph{malleable}: an attacker can change the
      ciphertext in a way that results in a predictable (or at least different)
      plaintext after decryption.
      To get tamper detection we need integrity protection, typically by using a
      MAC or an authenticated-encryption scheme.
    \end{block}
  }
\end{frame}

\section{Case study: police body cameras}

\subsection{ECB vs CTR in practice}

\begin{frame}
  \begin{block}{Motivation}
    Police body cameras record sensitive material.
    We want confidentiality of stored footage, but encryption must be used
    correctly.
  \end{block}
\end{frame}

\begin{frame}
  \begin{figure}
    \begin{sidecaption}[Tux encrypted using different modes.]{%
      Tux encrypted using different block modes of operation.
      When using ECB mode, we can still distinguish Tux.
      But we expect encryption to have the effect as shown for CTR.
    }[fig:EncModes]
    \hfill
    \begin{subfigure}[t]{0.3\linewidth}
      \includegraphics[width=\linewidth]{../../../project/figs/Tux.pdf}
      \caption{Original Tux.
      Image by: Larry Ewing, Simon Budig, Garrett LeSage.}
    \end{subfigure}
    \hfill
    \begin{subfigure}[t]{0.3\linewidth}
      \includegraphics[width=\linewidth]{../../../project/figs/Tux_encrypted_ecb.png}
      \caption{ECB-encrypted Tux.
      Image by: RFL890.}
    \end{subfigure}
    \hfill
    \begin{subfigure}[t]{0.3\linewidth}
      \includegraphics[width=\linewidth]{../../../project/figs/Tux_encrypted_ctr.png}
      \caption{CTR-encrypted Tux.
      Image by: RFL890.}
    \end{subfigure}
    \hspace{\fill}
    \end{sidecaption}
  \end{figure}

  \ltnote{%
    This is the core contrast example for modes.
    We keep the key and plaintext image invariant and vary only the mode
    (ECB vs CTR), making pattern leakage discernible.
  }
\end{frame}

\begin{frame}
  \begin{exercise}
    The camera uses CTR mode.

    What must be true about the IV/nonce values to keep confidentiality?

    \medskip
    Bonus: what could go wrong if the camera reboots and starts counting from
    the same nonce again?
  \end{exercise}

  \ltnote{%
    This ties the abstract nonce-reuse rule to a concrete operational scenario
    (device reboot/state loss). Students often know \enquote{nonce must be
    unique} but miss why systems engineering matters.
  }
\end{frame}

\begin{frame}
  \begin{block}{Take-away}
    \begin{itemize}
      \item Modes matter: ECB is not acceptable for structured data.
      \item IV/nonce discipline is part of the security definition.
      \item Shared-key encryption is fast and practical, but key distribution and
        correct use are crucial.
    \end{itemize}
  \end{block}
\end{frame}

\begin{frame}
  \begin{center}
    Questions?
  \end{center}
\end{frame}
