% CBC decryption diagram
\begin{tikzpicture}[
    >=Stealth,
    box/.style={draw, minimum width=1.2cm, minimum height=0.8cm},
    xor/.style={draw, circle, minimum size=0.5cm, inner sep=0pt},
    every node/.style={font=\small},
    node distance=0.8cm and 1.8cm
  ]
  
  % Block 0
  \node (c0) {$C_0$};
  \node[box, below=of c0] (dec0) {$\Dec_k$};
  \node[xor, below=of dec0] (xor0) {$\oplus$};
  \node[below=of xor0] (p0) {$P_0$};
  \node[left=1.2cm of xor0] (iv) {IV};
  
  % Block 1
  \node[right=of c0] (c1) {$C_1$};
  \node[box, below=of c1] (dec1) {$\Dec_k$};
  \node[xor, below=of dec1] (xor1) {$\oplus$};
  \node[below=of xor1] (p1) {$P_1$};
  
  % Block 2
  \node[right=of c1] (c2) {$C_2$};
  \node[box, below=of c2] (dec2) {$\Dec_k$};
  \node[xor, below=of dec2] (xor2) {$\oplus$};
  \node[below=of xor2] (p2) {$P_2$};
  
  % Arrows for block 0
  \draw[->] (c0) -- (dec0);
  \draw[->] (dec0) -- (xor0);
  \draw[->] (iv) -- (xor0);
  \draw[->] (xor0) -- (p0);
  
  % Arrows for block 1
  \draw[->] (c1) -- (dec1);
  \draw[->] (dec1) -- (xor1);
  \draw[->] (xor1) -- (p1);
  
  % Arrows for block 2
  \draw[->] (c2) -- (dec2);
  \draw[->] (dec2) -- (xor2);
  \draw[->] (xor2) -- (p2);
  
  % Chaining arrows (from ciphertext to next XOR)
  \draw[->] ($(c0.south)!0.5!(dec0.north)$) -| ($(xor1.west)$);
  \draw[->] ($(c1.south)!0.5!(dec1.north)$) -| ($(xor2.west)$);
  
\end{tikzpicture}
