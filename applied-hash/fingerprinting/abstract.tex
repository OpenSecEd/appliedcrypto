% What's the problem?
% Why is it a problem? Research gap left by other approaches?
% Why is it important? Why care?
% What's the approach? How to solve the problem?
% What's the findings? How was it evaluated, what are the results, limitations, 
% what remains to be done?

% XXX Summary
\emph{Summary:}
We talk about the fingerprinting property of hash functions.
We look at two example use cases: BitTorrent and Git.

% XXX Motivation and intended learning outcomes
\emph{Intended learning outcomes:}
You should be able to apply hash functions in various settings.

% XXX Prerequisites
\emph{Prerequisites:}
We assume that you know about hash functions in general, but we recap the main 
properties.

% XXX Reading material
\emph{Reading:}
You can learn more about Git's internals here:
\begin{center}
  \url{https://git-scm.com/book/en/v2/Git-Internals-Git-Objects}
\end{center}
A good starting point to get an overview of BitTorrent is its page on 
Wikipedia:
\begin{center}
  \url{https://en.wikipedia.org/wiki/BitTorrent}
\end{center}
Otherwise, the BitTorrent specification is the authoritative source:
\begin{center}
  \url{https://www.bittorrent.org/beps/bep_0003.html}
\end{center}
