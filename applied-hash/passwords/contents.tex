\mode*

\mode<all>{\mode*

\section[Hash functions]{What was a hash function now again?}

%\begin{frame}
%  \begin{definition}[One-way function\footfullcite{GoldreichFOC-1}]
%    \begin{itemize}
%      \item Let \(h\colon \{0,1\}^*\to \{0,1\}^*\).
%      \item \(h\) is \emph{one-way} if
%        \begin{enumerate}
%          \item there exists an efficient algorithm \(A\) such that \(A(x) 
%              = h(x)\);
%          \item for every efficient algorithm \(A^\prime\), every positive 
%            polynomial \(p(\cdot)\) and all sufficiently large \(n\)'s
%            \[\Prob{A^\prime(h(x), 1^n) \in h^{-1}(h(x))} < \frac{1}{p(n)}\]
%        \end{enumerate}
%    \end{itemize}
%  \end{definition}
%\end{frame}

\begin{frame}
  \begin{definition}[Preimage resistance {[one way]}]
    \begin{description}
      \item[Input] hash function~\(H\), value~\(y\).
      \item[Output] Any \(x\) such that \(H(x) = y\).
    \end{description}
  \end{definition}

  \begin{definition}[Second preimage resistance {[weak collision resistance]}]
    \begin{description}
      \item[Input] hash function~\(H\), value \(x\).
      \item[Output] Any value \(x'\) such that \(H(x) = H(x')\).
    \end{description}
  \end{definition}

  \begin{definition}[Collision resistance {[strong collision resistance]}]
    \begin{description}
      \item[Input] hash function~\(H\).
      \item[Output] Any two \(x, x'\) such that \(H(x) = H(x')\).
    \end{description}
  \end{definition}
\end{frame}

}


\section{Something you know}

\subsection{\enquote{Proof of knowledge}}

\begin{frame}
  \begin{idea}[Something you know]
    \begin{itemize}
      \item We have a prover and a verifier.
      \item Prover must convince verifier he knows some 
        secret.
    \end{itemize}
  \end{idea}

  \begin{idea}[Password]
    \begin{itemize}
      \item Prover and verifier shares a secret value.

      \item Prover tells verifier the value to convince the verifier.
    \end{itemize}
  \end{idea}
\end{frame}

\begin{frame}
  \begin{remark}
    \begin{itemize}
      \item If the adversary learns the secret, he can convince the verifier he 
        is the prover.
    \end{itemize}
  \end{remark}

  \begin{example}
    \begin{itemize}
      \item Adversary might \enquote{overhear the conversation}.
      \item Adversary might \enquote{trick} the prover to reveal the secret.
      \item Adversary might guess the secret.
    \end{itemize}
  \end{example}
\end{frame}

\subsection{Online or offline guessing?}

\begin{frame}
  \begin{definition}[Online]
    \begin{itemize}
      \item The adversary must interact with the system for each guess.
    \end{itemize}
  \end{definition}

  \pause

  \begin{example}[Online]
    \begin{itemize}
      \item Guessing the password of a Google account.
      \item Must submit each guess to Google.
    \end{itemize}
  \end{example}
\end{frame}

\begin{frame}
  \begin{definition}[Offline]
    \begin{itemize}
      \item The adversary can verify the guess himself.
    \end{itemize}
  \end{definition}

  \pause

  \begin{example}[Offline]
    \begin{itemize}
      \item Guessing the password of an encrypted file.
      \item For each guess, try to decrypt.
    \end{itemize}
  \end{example}

  \begin{example}[Offline]
    \begin{itemize}
      \item Guessing the password corresponding to a hash.
      \item For each guess, hash and compare the result.
    \end{itemize}
  \end{example}
\end{frame}

\subsection{Storing secrets}

\begin{frame}
  \begin{remark}
    \begin{itemize}
      \item The user can store the secret in its mind.
      \item This is assumed inaccessible (for now).
    \end{itemize}
  \end{remark}

  \pause

  \begin{question}
    \begin{itemize}
      \item The verifier is a machine.
      \item The verifier must verify what the prover says.
      \item This means that the verifier must have some data to check against.
      \item How should this be stored?
    \end{itemize}
  \end{question}
\end{frame}

\begin{frame}
  \begin{remark}
    \begin{itemize}
      \item Our concern is that someone can read this data.
      \item Password reuse for other services?
    \end{itemize}
  \end{remark}
\end{frame}

\begin{frame}
  \begin{idea}[Password hashes]
    \begin{itemize}
      \item We want to compare user-entered and stored password.
      \item We do an irreversible one-way transformation on both.
      \item Then they are still comparable.
      \item The preimage cannot be gained from storage.
    \end{itemize}
  \end{idea}

  \pause{}

  \begin{example}
    \begin{itemize}
      \item Cryptographic hash function \(h\colon \bin^*\to \bin^n\).
      \item On registration, store \(y = h(p)\).
      \item User authenticates with \(p'\), check if \(h(p') \stackrel{?}{=}  
        y\) equals what we stored.
    \end{itemize}
  \end{example}
\end{frame}

\begin{frame}
  \begin{remark}
    \begin{itemize}
      \item Consider guessing again.
      \item The used password space is small.
      \item We only need to evaluate a subset: \(h\colon 
          \bin^{\color{red}{m}}\to \bin^n\).
      \item With faster computers we can guess a lot.
    \end{itemize}
  \end{remark}

  \pause

  \begin{solution}
    \begin{itemize}
      \item Choose \(h\) to be slow to compute.
      \item \Eg iterate it over itself 10\,000 times (\(h^{10000}(p)\)).
      \item This will slow down guessing attacks.
    \end{itemize}
  \end{solution}
\end{frame}

\begin{frame}
  \begin{remark}
    \begin{itemize}
      \item A list of password hashes reveals if two users have the same 
        password.
      \item Can guess the password for all users at once:
        \begin{enumerate}
          \item Make a guess, compute the hash.
          \item Check if it matches \emph{any} user's password.
        \end{enumerate}
    \end{itemize}
  \end{remark}

  \pause

  \begin{solution}
    \begin{itemize}
      \item Add a \emph{salt}: a small random value (\eg 128 bits) unique for 
        each user.
      \item Salt~\(s\rgets \bin^{128}\), change hash to \(h(s, p)\).
      \item Now all hashes will be unique.
    \end{itemize}
  \end{solution}
\end{frame}

\begin{frame}
  \begin{remark}
    \begin{itemize}
      \item The salt is not a secret, it just adds uniqueness.
      \item It can be stored in plain text along with the password hash.
    \end{itemize}
  \end{remark}
\end{frame}

\begin{frame}
  \begin{example}
    \begin{itemize}
      \item There are many libraries.
      \item bcrypt~\cite{bcrypt} implements all this functionality.
      \item Argon2 is another, more recent technique.
      \item They should also be available in most languages and libraries.
    \end{itemize}
  \end{example}
\end{frame}

