% What's the problem?
% Why is it a problem? Research gap left by other approaches?
% Why is it important? Why care?
% What's the approach? How to solve the problem?
% What's the findings? How was it evaluated, what are the results, limitations, 
% what remains to be done?

% XXX Summary
\emph{Summary:}
In this document we explore solutions to old INL1Written problems.

% XXX Prerequisites
\emph{Prerequisites:}
We require the following learning objectives to be fulfilled:
\begin{restatable}{lo}{LOterminology}\label{LOterminology}
  The student should be albe to \emph{use} basic terminology in computer 
  security and cryptography correctly
\end{restatable}
\begin{restatable}{lo}{LOdescribe}\label{LOdescribe}
  The student should be albe to \emph{describe} cryptographic concepts and 
  \emph{explain} their security properties.
\end{restatable}

% XXX Motivation and intended learning outcomes
\emph{Intended learning outcomes:}
The material is designed so that the student should learn the following%
\footnote{%
  \Cref{LOapply,LOanalyse} are restatements of the learning objective in the 
  syllabus of DD2520 at KTH.
  That learning objective say that the student should be able to 
  \enquote{identify and categorise threats against a cryptographic IT-system at 
  a conceptual level, suggest appropriate countermeasures and present the 
reasoning to others}.
}.
\begin{restatable}{lo}{LOapply}\label{LOapply}
  The student should be albe to \emph{construct} solutions to relevant problems 
  by combining cryptographic primitives into a secure system.
\end{restatable}
\begin{restatable}{lo}{LOanalyse}\label{LOanalyse}
  The student should be albe to \emph{evaluate} cryptographic constructs 
  regarding how well they protect from threats.
\end{restatable}

\ltnote{%
  We will use the variation theory of learning
  \parencite{NecessaryConditionsOfLearning}
  as the foundation for our analysis and design of the material.

  We also use an open-ended variant of (pure) question-based learning
  \parencite{pQBL}
  that has its origins in an experiment by
  \cite{Szekely1950}.
}
