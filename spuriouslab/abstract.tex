This lab focuses on brute forcing approaches, specifically the probability of 
finding \emph{the} correct solution.
It begins by finding the unknown plaintext of a given ciphertext and follows by
some time to reflect on the probability of the plaintext indeed being the 
correct plaintext.
The next step is to generate two keys, which given a ciphertext, will yield two
equaly probable plaintexts.

We use some classical ciphers.
Those are best covered in Chap.~1 of 
\citetitle{Stinson2006cta}~\cite{Stinson2006cta} or 
\citetitle{Bosk2013itn}~\cite{Bosk2013itn}.
Finally, you should read about spurious keys and unicity distance.
The recommended literature is Chap.~2 in~\cite{Stinson2006cta}.
