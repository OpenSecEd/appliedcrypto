\documentclass[a4paper]{llncs}
\usepackage[utf8]{inputenc}
\usepackage[T1]{fontenc}
\usepackage[swedish,english]{babel}
\usepackage[hyphens]{url}
\usepackage[hidelinks]{hyperref}
\usepackage{varioref}
\usepackage{cleveref}
\usepackage{amssymb}
\usepackage{csquotes}
\usepackage{verbatim}

\usepackage[natbib,style=numeric-comp,maxbibnames=99]{biblatex}
\addbibresource{spuriouslab.bib}

\usepackage[lncs,crypto]{bibsp}

%\printanswers
\pagestyle{plain}

\title{Lab: May the (Brute) Force Be with You}
\subtitle{A lab to reflect on forcing solutions}
\author{%
  Daniel Bosk\inst{1}\fnmsep\inst{2}
}
\institute{%
  School of Computer Science and Communication\\
  KTH Royal Institute of Technology, SE-100\,44 Stockholm
  \and
  Department of Information and Communication Systems\\
  Mid Sweden University, SE-851\,70 Sundsvall
}
\date{14th April 2015}

\begin{document}
\maketitle
\begin{abstract}
  Cryptography has a central role in security.
To fully understand how many security mechanisms can be implemented we need 
cryptography.
For this reason, we also need higher-level knowledge about what can be achieved 
with cryptography to not limit our thoughts about possible solutions.
This learning session is intended to give a high-level overview of 
cryptography: \ac{SKE}, \ac{PKE}, digital signatures, \ac{ZKP} and \ac{MPC}.
In particular, the \acp{ILO} are that you should be able to
\begin{itemize}
  \item \emph{understand} what properties can be achieved with cryptography.
  \item \emph{analyse} a situation and \emph{suggest} what cryptographic 
    properties are desirable.
\end{itemize}

We will treat
Chapter 5 in Anderson's \citetitle{Anderson2008sea}~\cite{Anderson2008sea} and
Chapter 14 in Gollmann's \citetitle{Gollmann2011cs}~\cite{Gollmann2011cs}.
To practice your understanding of these mechanisms it is recommended to do 
exercises 14.2, 14.3 and 14.7 in~\cite{Gollmann2011cs}.
We will also cover some topics from 
\citetitle{Katz-Lindell}~\cite{Katz-Lindell} and 
\citetitle{GoldreichFOC-1}~\cite{GoldreichFOC-1}.

\end{abstract}


\section{Introduction}

The idea of this assignment is to introduce the concept of brute forcing 
security mechanisms.
The mechanism in question here is a simple monoalphabetic cryptographic 
algorithm.
It also serves to help you reflect on deniability: how can you be sure that 
your solution is the correct one?

\subsection{Aims}

The aim of this assignment is to examine that you are:
\begin{itemize}
  \item Able to reason about the security of basic security mechanisms.
\item Have an understanding for plausible deniability.
\item Able to make a proof of concept of how to break a simple and insecure 
mechanism.

\end{itemize}


\section{Theory}

If you do not have probability theory and statistics fresh in memory you are 
recommended to revise that.
% XXX replace Swedish text with English counterpart
The text \citetitle{kthsannolikhet} by \citet{kthsannolikhet} (in Swedish) 
treats this subject, you are recommended to read Sect.~1--4.

If you have previously taken (or are currently taking) a course on 
cryptography, the material from that course covering classical cryptography is 
enough.
% XXX replace Swedish text with English counterpart
Otherwise you are recommended to read 
\citetitle{Bosk2013itn}~\cite{Bosk2013itn} (in Swedish) or Chap.~1 in 
\citetitle{Stinson2006cta} by \citet{Stinson2006cta}.

Finally, you should read about spurious keys and unicity distance.
The recommended literature is Chap.~2 in~\cite{Stinson2006cta}.


\section{Assignment}

The first part of the assignment is to break a monoalphabetic cipher.
The intercepted text is the following:
%\begin{quote}
%  TSVCFMSFQ OÅ CFMMB LVQT TLB UBQB UÄK EÖQSQPHMB NFC OQPHQBNNFQJMH PDG 
%  NBSFNBSJL PDG LQXOSPHQBEJ.
%\end{quote}
\begin{verbatim}
  TSVCFMSFQOÅCFMMBLVQTTLBUBQBUÄKEÖQSQPHMBNFCOQPHQBNNFQJMHPDG
  NBSFNBSJLPDGLQXOSPHQBEJ
\end{verbatim}
Find the corresponding plaintext of this ciphertext.
When you have found a plaintext and the key, think about how certain you can be 
that this is indeed the correct key (and thus correct plaintext).

The second part of the assignment is about spurious keys 
\cite[Chap.~2]{Stinson2006cta}.
By spurious keys we mean a set of \(n\) keys \(k_1, k_2, \ldots, k_n\) which 
all decrypt a ciphertext \(c\) to meaningful plaintexts \(m_1, m_2, \ldots, 
m_n\).
Your job is to construct such a ciphertext with two spurious keys \(k_1\) and 
\(k_2\) for the cryptosystem used in the first part of the assignment.
The texts should be as long as possible (the longest meaningful plaintexts will 
receive an award), and they do not have to be both in Swedish or English---one 
plaintext in English and one in Swedish is fine.

Algorithmically finding a spurious key should be possible, in this case, by 
generating plaintext using \(n\)-grams.
However, using pen and paper is probably the most straightforward way, and 
probably the fastest for this short assignment.


\section{Examination}

You must submit your solutions to the assignment in a report (PDF-format) in 
the course platform.
The report must contain the following:
\begin{enumerate}
  \item The plaintext corresponding to the cryptotext given above with an 
    explanation of what makes you sure this is the correct plaintext.

%    \begin{solution}
%      % XXX construct a ciphertext which can decrypt into two plaintexts
%      % XXX change the plain- and ciphertexts to English
%      The key is: \enquote{badcfehgjilknmporqtsvuwyxzåäö}.
%      The plaintext is \enquote{studenter på denna kurs ska vara väl förtrogna 
%      med programmering och matematik och kryptografi}.
%      They should motivate why this is a probably-correct decryption.
%    \end{solution}

  \item One ciphertext \(c\), two keys \(k_1\) and \(k_2\) and the 
    corresponding plaintexts \(m_1\) and \(m_2\), such that \(\Enc[k_1]{m_1} 
      = c = \Enc[k_2]{m_2}\).
    Also explain your method for creating \(c, k_1, k_2, m_1. m_2\) and why we 
    want to have spurious keys.
    Also, how is the length of the message affecting the spurious keys?

%    \begin{solution}
%      Verify that \( \Enc[k_1]{m_1} = c = \Enc[k_2]{m_2}\).
%    \end{solution}
\end{enumerate}


\printbibliography{}
\end{document}
