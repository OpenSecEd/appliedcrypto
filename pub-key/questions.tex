\question[3]
% tags: crypto:pub-key
% tags: A:C:E
Briefly describe the problem of key exchange and in what way public-key 
cryptography might help.

\begin{solution}
  For Alice and Bob to communicate securely (over an encrypted channel) they 
  must agree on a secret key.
  The trivial way is that they meet in person to agree on a key.
  But this get inconvenient if Alice and Bob live on different continents or 
  Bob runs a server farm and must communicate with many people.

  Public-key crypto allows Alice and Bob to agree on a secret over a public 
  channel without Eve learning it.
  One example is the Diffie--Hellman key-exchange protocol.

  However, Alice and Bob need digital signatures to be sure that they are 
  indeed talking to each other.
\end{solution}


\question
% tags: crypto
% tags: E
The security of many cryptographic constructions is based on
\begin{choices}
  \choice that the adversary doesn't know how the system works.

  \CorrectChoice that a problem is hard for anyone who doesn't know the secret, 
  but easy for everyone who knows the secret.

  \choice a hard problem.
\end{choices}

\begin{solution}
  The first option violates Kerckhoff's principle.
  The second is correct.
  The problem must have some easy aspect to make it easy for Alice and Bob but 
  hard for Eve.
\end{solution}


\question[3]
% tags: crypto:pub-key
% tags: A:C:E
The Diffie--Hellman key-exchange works as follows.
Alice chooses \(x\) and sends \(g^x \mod p\) to Bob.
Bob chooses \(y\) and sends \(g^y \mod p\) to Alice.
Now, both Alice and Bob can compute a joint key~\(k = (g^x)^y = (g^y)^x\).

According to the Diffie--Hellman problem, Eve cannot compute \(k = g^{xy}\) 
since she only has \(g^x\) and \(g^y\).
But Eve can do something else to eavesdrop on Alice and Bob, what?
(Hint: can Alice and Bob tell who sent \(g^y\) and \(g^x\), respectively?)

\begin{solution}
  Eve can intercept Alice's \(g^x\), send \(g^x'\) to Bob.
  Intercept Bob's \(g^y\), send \(g^y'\) to Alice.
  Now Alice will have \(k = (g^y')^x\), Eve will have \(k = (g^x)^y'\),
  Bob will have \(k' = (g^x')^y\), Eve will have \(k' = (g^y)^x'\).
  Alice thinks she's talking to Bob, Bob thinks he's talking to Alice.
  Eve can just relay the messages after reading them.
\end{solution}


\question[3]
% tags: crypto:pub-key
% tags: A:C:E
What does it mean that an encryption scheme has a homomorphic property?
Is that a good or a bad thing?

\begin{solution}
  It means that some operation on the ciphertext translates to an operation on 
  the plaintext, similar to how \(\log x + \log y = \log xy\).

  In general, it's a bad thing.
  However, it can be a useful property too; for example, in electronic voting 
  to shuffle votes, in multiparty computations to add or multiply without 
  revealing the operands.
\end{solution}


\question[3]
% tags: crypto:pub-key
% tags: A:C:E
Explain why it's really bad for a digital signature scheme to have a 
homomorphism.

\begin{solution}
  For example, we have the following cheque system.
  Alice issues a cheque of \(100\) by signing \(100\).
  She gives the check to Bob.
  Say that it's a multiplicative homomorphism, now Bob can square Alice's 
  cheque and get a cheque worth \(10000\) instead.
\end{solution}
