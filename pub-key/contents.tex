\mode*

% Since this a solution template for a generic talk, very little can
% be said about how it should be structured. However, the talk length
% of between 15min and 45min and the theme suggest that you stick to
% the following rules:  

% - Exactly two or three sections (other than the summary).
% - At *most* three subsections per section.
% - Talk about 30s to 2min per frame. So there should be between about
%   15 and 30 frames, all told.


\section{Public-key cryptography}

\subsection{Key-exchange schemes}

\begin{frame}
  \begin{idea}
    \begin{itemize}
      \item It's difficult to have to exchange keys in advance.

        \pause{}

      \item What if we could securely exchange keys at a distance?
      \item If we could do it just before we use them?
    \end{itemize}
  \end{idea}

%  \pause{}
%
%  \begin{exercise}
%    \begin{itemize}
%      \item Say Alice and Bob want to communicate.
%      \item Eve wants to eavesdrop as usual.
%      \item What are the requirements of such a system?
%    \end{itemize}
%  \end{exercise}
\end{frame}

\begin{frame}
  \begin{solution}[Requirements]
    \begin{itemize}
      \item We need a problem that is easy for Alice and Bob.
      \item It should be hard for Eve.
    \end{itemize}
  \end{solution}
\end{frame}

\begin{frame}
  \begin{definition}[\Acl{DLP}, \acs{DLP}]
    \begin{itemize}
      \item Let \(\Z_p^*\) be the multiplicative group of residues modulo 
        \(p\in \N\), where \(p\) is a prime.

        \pause{}

        \begin{description}
          \item[Given] \(g, g^x\in \Z_p^*\)
          \item[Find] \(x\).
        \end{description}

      \item I.e.\ compute \(\log_{g\in \Z_p}(g^x)\).
    \end{itemize}
  \end{definition}
\end{frame}

\begin{frame}
  \begin{definition}[\Acl{DHP}, \acs{DHP}\footfullcite{DiffieHellman}]
    \begin{description}
      \item[Given] \(g, g^x, g^y\in \Z_p^*\)
      \item[Find] \(g^{xy}\)
    \end{description}
  \end{definition}

  \pause{}
  
  \begin{definition}[\Acl{DDH} Problem, \acs{DDH}]
    \begin{description}
      \item[Given] \(g, g^x, g^y, g^z\in \Z_p^*\)
      \item[Decide] \(z \stackrel{?}{=} xy\)
    \end{description}
  \end{definition}
\end{frame}

\begin{frame}
  \begin{itemize}
    \item If we can solve \ac{DLP}, then we can solve \ac{DHP} and \ac{DDH} 
      too.

      \pause{}

    \item Maybe \ac{DHP} and \ac{DDH} can be solved without \ac{DLP}.
    \item We don't know yet.

      \pause{}

    \item We usually assume \ac{DLP}, \ac{DHP} and \ac{DDH} are hard.
  \end{itemize}
\end{frame}

\begin{frame}
  \begin{exercise}
    \begin{itemize}
      \item \citeauthor{DiffieHellman}\footfullcite{DiffieHellman} used 
        \ac{DHP} to create a key-exchange protocol.

        \pause{}

      \item Take some time to figure out how we can use these problems to 
        achieve what we want.
    \end{itemize}
  \end{exercise}

  \begin{block}{Reminder}
    \begin{itemize}
      \item Alice and Bob want to exchange a secret key.
      \item Then they can use the key to encrypt their communications.
    \end{itemize}
  \end{block}
\end{frame}

\begin{frame}
  \begin{definition}[Diffie-Hellman key-exchange]
    \begin{itemize}
      \item Let \(g\in \Z_p^*\) (publicly known, e.g.\ RFC, standard\ dots).

        \pause{}

      \color{green!50!red}
      \item Alice generates random \(0 < x < |\Z_p^*|\).
      \item She send \(g^x\) to Bob.

        \pause{}

      \color{blue}
      \item Bob generates random \(0 < y < |\Z_p^*|\).
      \item He sends \(g^y\) to Alice.

        \pause{}

      \color{green!50!red}
      \item Alice has \(x\) and \(g, g^y\).
      \color{blue}
      \item Bob has \(g, g^x\) and \(y\).
      \color{black}
      \item They both compute \(g^{xy} = {\color{green!50!red} (g^y)^x} 
          = {\color{blue} (g^x)^y}\).

        \pause{}

      \color{red}
      \item Eve has \(g, g^x, g^y\).
      \item By \ac{DHP} she cannot compute \(g^{xy}\).
    \end{itemize}
  \end{definition}
\end{frame}

\subsection{Encryption and decryption}

\begin{frame}
  \begin{idea}
    \begin{itemize}
      \item Fine, we can use \(g^{xy}\) as a key in a cipher.
        \begin{itemize}
          \item \(\Enc[g^{xy}]{m}\), where \(\EncOp\) is a symmetric cipher.
        \end{itemize}
      \item But shouldn't we be able to include a message directly?
    \end{itemize}
  \end{idea}
\end{frame}

\begin{frame}
  \begin{definition}[ElGamal Encryption Scheme\footfullcite{ElGamal}]
    Set-up:
    \begin{itemize}
      \item Let \(g\in \Z_p^*\), randomly choose \(0 < x < |\Z_p^*|\).
      \item Alice publishes \(\Z_p^*, g, g^x\) to everyone.
    \end{itemize}
    Encryption:
    \begin{itemize}
      \item Bob chooses random \(0 < y < |\Z_p^*|\) and computes \(g^y\).
      \item Bob's message \(m\in \Z_p^*\).
      \item He sends \((g^y, m(g^{x})^y)\) to Alice.
    \end{itemize}
    Decryption:
    \begin{itemize}
      \item Alice computes \((g^y)^x\) and \(m(g^x)^y((g^{y})^x)^{-1} = m\).
    \end{itemize}
  \end{definition}
\end{frame}

\subsection{Digital signatures}

\begin{frame}
  \begin{idea}
    \begin{itemize}
      \item Sure, if Bob sends a message to Alice, he's sure she's the only one
        who can decrypt it.

        \pause{}

      \item Can't we turn this around?
        \begin{itemize}
          \item Can't Alice use the same system to ensure Bob knows the message
            came from Alice?
        \end{itemize}
    \end{itemize}
  \end{idea}

  \pause{}

  \begin{exercise}
    \begin{itemize}
      \item Look at the ElGamal encryption scheme for a bit.
      \item Try to find a way to \enquote{run it backwards}.
    \end{itemize}
  \end{exercise}
\end{frame}

\begin{frame}
  \begin{definition}[ElGamal Signature Scheme\footfullcite{ElGamal}]
    Set-up:
    \begin{itemize}
      \item Let \(g\in \Z_p^*\) and \framebox{\(h\) be a one-way function}.
      \item Alice publishes \(\Z_p^*, g, g^x\) to everyone.
    \end{itemize}
    Signing \(m\in \Z_p^*\):
    \begin{itemize}
      \item \framebox{Alice} chooses random \(0 < y < |\Z_p^*|\) and computes 
        \(r = g^y\in \Z_p^*\).
      \item She computes \framebox{\(s = (h(m) - xr) y^{-1} \pmod{|\Z_p^*|}\).}
      \item She sends \((r, s)\) to Bob.
    \end{itemize}
    Verification:
    \begin{itemize}
      \item \framebox{Bob checks if \(g^{h(m)} \stackrel{?}{=}_{\Z_p^*} (g^x)^r 
            r^s\)}\( =_{\Z_p^*}
          (g^x)^{g^y} (g^y)^{(h(m) - xg^y)y^{-1}} =_{\Z_p^*}
          g^{xg^y + h(m) - xg^y}\)
    \end{itemize}
  \end{definition}
\end{frame}

\begin{frame}
  \begin{remark}
    \begin{itemize}
      \item It works without the hash.
      \item But then we can multiply two messages and still get a valid 
        signature.
    \end{itemize}
  \end{remark}
\end{frame}

\subsection{Homomorphic properties}

\begin{frame}
  \begin{definition}[Homomorphism]
    A \emph{homomorphism} is a map (function) that preserves structure between 
    two algebraic structures.
  \end{definition}

  \pause{}

  \begin{example}
    \begin{itemize}
      \item Let \(G_1 = (\R, \cdot)\) and \(G_2 = (\R, +)\) be groups.
      \item \(g_1, g_1^\prime\in G_1\) and \(g_2, g_2^\prime\in G_2\).

        \pause{}

      \item Consider \(\log\colon G_1\to G_2\).
        
        \pause{}

      \item \(\log(g_1\cdot g_1^\prime) = g_2 + g_2^\prime\).
    \end{itemize}
  \end{example}
\end{frame}

\begin{frame}
  \begin{exercise}
    The encryption (decryption) function of the ElGamal cryptosystem is 
    a homomorphism, what structure does it preserve?
  \end{exercise}
\end{frame}

\begin{frame}
  \begin{example}[ElGamal's homomorphism]
    \begin{itemize}
      \item Messages \(m, m^\prime\), ciphertexts \((g^y, m\cdot g^{xy}), 
          (g^{y^\prime}, m^\prime\cdot g^{x y^\prime})\).

      \item Remember: private key \(x\), hence the same.

        \pause{}

      \item Create ciphertext
        \begin{align*}
          (g^y g^{y^\prime}, m\cdot g^{xy}\cdot m^\prime\cdot g^{x y^\prime})
          &= (g^{y + y^\prime}, m\cdot m^\prime\cdot g^{xy + xy^\prime}) \\
          &= (g^{y + y^\prime}, m\cdot m^\prime\cdot g^{x(y + y^\prime)}).
        \end{align*}

        \pause{}

      \item Decryption: take \(g^{y + y^\prime}\), compute \((g^{y+y^\prime})^x 
          = g^{x(y + y^\prime)}\).

      \item Decryption thus yields \(m\cdot m^\prime\).

    \end{itemize}
  \end{example}
\end{frame}

\begin{frame}
  \begin{remark}
    \begin{itemize}
      \item We use a hash function in the signature scheme to counter the 
        homomorphic property.

      \item \(h(m)\cdot h(m^\prime)\neq h(m\cdot m^\prime)\).

        \pause{}

      \item Without the hash function we could create a valid signature for 
        a new message \emph{without knowing the signature key!}
    \end{itemize}
  \end{remark}
\end{frame}

\begin{frame}
  \begin{remark}
    \begin{itemize}
      \item There are many schemes with different homomorphic properties.
      \item There is even \emph{fully homomorphic 
          encryption}~\cite{GentryFullyHomomorphicEncryption}.
    \end{itemize}
  \end{remark}
\end{frame}


%%%%%%%%%%%%%%%%%%%%%%

\begin{frame}[allowframebreaks]
  \printbibliography
\end{frame}

