% $Id$
%\documentclass[handout]{beamer}
\documentclass{beamer}
\usepackage[utf8]{inputenc}
\usepackage[T1]{fontenc}
\usepackage[ibycus,english,swedish]{babel}
\usepackage{url}
\usepackage{graphicx}
\usepackage{color}
\usepackage{subfig}
\usepackage{multicol}
\usepackage{amssymb,amsmath,amsthm}
\usepackage{booktabs}
\usepackage[squaren,binary]{SIunits}
\usepackage{verbatim}

\setbeamertemplate{bibliography item}[text]
\usepackage[natbib,style=alphabetic,maxbibnames=99]{biblatex}
\addbibresource{symcrypt.bib}

%\newtheorem{theorem}{Sats}%[chapter]
\providetranslation[to=swedish]{Theorem}{Sats}
%\newtheorem{lemma}{Lemma}
%\newtheorem{corollary}{Korollarium}
\providetranslation[to=swedish]{Corollary}{Korollarium}
\theoremstyle{definition}
%\newtheorem{definition}{Definition}
%\newtheorem{axiom}[definition]{Axiom}
\newenvironment{axiom}[1]{\begin{block}{Postulat (#1)}}{\end{block}}
%\newtheorem{example}{Exempel}
\providetranslation[to=swedish]{Example}{Exempel}
\newtheorem{exercise}{Övning}

\DeclareMathOperator{\p}{\mathcal{P}}
\let\P\p
\DeclareMathOperator{\C}{\mathcal{C}}
\DeclareMathOperator{\K}{\mathcal{K}}
\DeclareMathOperator{\E}{\mathcal{E}}
\DeclareMathOperator{\D}{\mathcal{D}}
\DeclareMathOperator{\X}{\mathcal{X}}
\DeclareMathOperator{\Y}{\mathcal{Y}}
\DeclareMathOperator{\h}{\mathcal{H}}
\let\H\h

\DeclareMathOperator{\N}{\mathbb{N}}
\DeclareMathOperator{\Z}{\mathbb{Z}}
\DeclareMathOperator{\R}{\mathbb{R}}

\let\stoch\mathbf

\renewcommand{\p}{\stoch P}
\renewcommand{\c}{\stoch C}
\renewcommand{\k}{\stoch K}
\newcommand{\x}{\stoch X}
\newcommand{\y}{\stoch Y}
\newcommand{\e}{\stoch E}
\newcommand{\s}{\stoch S}

\DeclareMathOperator{\lequiv}{\Longleftrightarrow}
\DeclareMathOperator{\xor}{\oplus}

\renewcommand{\qedsymbol}{Q.E.D.}

\DeclareMathOperator{\hmac}{HMAC}
\DeclareMathOperator{\concat}{||}

\mode<presentation>
{
  \usetheme{Frankfurt}
  \setbeamercovered{transparent}
  \usecolortheme{seagull}
}
\setbeamertemplate{footline}{\insertframenumber}

\title{%
  Symmetrisk kryptografi
}
\author{Daniel Bosk\footnote{%
  Detta verk är tillgängliggjort under licensen Creative Commons 
  Erkännande-DelaLika 2.5 Sverige (CC BY-SA 2.5 SE).
  För att se en sammanfattning och kopia av licenstexten besök URL 
  \url{http://creativecommons.org/licenses/by-sa/2.5/se/}.
}}
\institute[MIUN IKS]{%
  %Department of Information and Communication Systems (ICS),\\
  %Mid Sweden University, Sundsvall.
  %
  Avdelningen för informations- och kommunikationssytem (IKS),\\
  Mittuniversitetet, Sundsvall.
}
\date{\today}

%\pgfdeclareimage[height=0.65cm]{university-logo}{MU_logotyp_int_CMYK.pdf}
%\logo{\pgfuseimage{university-logo}}

\AtBeginSection[]{%
  \begin{frame}<beamer>{Översikt}
    \small
    \tableofcontents[currentsection]
  \end{frame}
}
\AtBeginSubsection[]{%
  \begin{frame}<beamer>{Översikt}
    \small
    \tableofcontents[currentsubsection,sectionstyle=shaded]
  \end{frame}
}

\begin{document}

\begin{frame}
  \titlepage
\end{frame}

\begin{frame}{Översikt}
  \tableofcontents
\end{frame}
\begin{frame}{Litteratur}
  % $Id$
The lecture covers chapter 3 ``Public-Key Cryptography and Message 
Authentication'' in \cite{Stallings2013nse} and
the remaining parts of chapter 5 ``Cryptography'' in \cite{Anderson2008sea}.

After this you should solve problems 3.1, 3.2, 3.5, 3.7, 3.8, 3.9 and 3.10 in 
\cite{Stallings2013nse}.

\end{frame}


% Since this a solution template for a generic talk, very little can
% be said about how it should be structured. However, the talk length
% of between 15min and 45min and the theme suggest that you stick to
% the following rules:  

% - Exactly two or three sections (other than the summary).
% - At *most* three subsections per section.
% - Talk about 30s to 2min per frame. So there should be between about
%   15 and 30 frames, all told.


\section{Klassisk kryptografi}

\subsection{Kryptosystem}

\begin{frame}{\insertsubsectionhead}
  \begin{itemize}
    \item Ordet kryptografi kommer från grekiskans \ibygr{krupto's} 
      (\emph{kryptos}) och \ibygr{gra'fos} (\emph{graphos}) \cite{OED2013cg}.

    \item Dessa betyder \emph{gömd} eller \emph{hemlig} \cite{OED2013c} 
      respektive \emph{skrift} \cite{OED2013g}.

    \item Ordet kryptografi betyder följaktligen \emph{hemlig skrift}.

    \item I modern tid är kryptografin ett högst matematiskt område.

    \item Det finns inte utrymme för annat än matematisk precision.

  \end{itemize}
\end{frame}

\begin{frame}{\insertsubsectionhead}
  \begin{figure}
    \includegraphics[width=\textwidth]{symmetric.eps}
    \caption{Symmetrisk kryptering.
    Bild: \cite{Stallings2011can}.}
  \end{figure}
\end{frame}

\begin{frame}{\insertsubsectionhead}
  \begin{definition}\label{def:kryptosystem}
    Ett \emph{kryptosystem} är en tupel \((\P, \C, \K, \E, \D)\) där följande 
    gäller:
    \begin{enumerate}
      \item \(\P\) är en ändlig mängd av möjliga klartexter.

      \item \(\C\) är en ändlig mängd av möjliga kryptotexter.

      \item \(\K\), kallad \emph{nyckelrymden}, är en ändlig mängd av möjliga 
        nycklar.

      \item För varje \(k\in \K\) finns en \emph{krypteringsregel} \(e_k\in 
        \E\) och motsvarande \emph{avkrypteringsregel} \(d_k\in \D\).
        Varje \(e_k\colon \P\to \C\) och \(d_k\colon \C\to \P\) är funktioner 
        sådana att \(d_k(e_k(p)) = p\) för alla klartexter \(p\in \P\).

    \end{enumerate}
  \end{definition}
\end{frame}

\begin{frame}{\insertsubsectionhead}
  \begin{itemize}
    \item Typer av operationer för att transformera klartext till kryptotext.

    \item Antalet nycklar som används.

    \item Sätt att processa klartexten:
      \begin{itemize}
        \item Blockchiffer.
        \item Strömchiffer.
      \end{itemize}

  \end{itemize}
\end{frame}

\begin{frame}{\insertsubsectionhead}{Kryptanalys}
  \begin{itemize}
    \item Enbart chiffertext (ciphertext only).
    \item Känd klartext (known plaintext).
    \item Vald klartext (chosen plaintext).
    \item Vald kryptotext (chosen ciphertext).
    \item Vald text (chosen text).
  \end{itemize}
\end{frame}

\begin{frame}{\insertsubsectionhead}
  \begin{definition}
    Ett kryptosystem är beräkningsmässigt säkert om det uppfyller någon eller 
    båda av följande:
    \begin{itemize}
      \item Kostnaden för att knäcka chiffret är högre än värdet på 
        informationen det skyddar.
      \item Tiden det tar att knäcka chiffret är längre än tiden informationen 
        är värdefull.
    \end{itemize}
  \end{definition}
\end{frame}

\subsection{Substitutionschiffer}

\begin{frame}{\insertsubsectionhead}{Caesarchiffer}
  \begin{definition}[Skiftchiffer]\label{def:shiftCipher}
    Låt \(\P = \C = \K = \Z_{29}\) och låt varje bokstav i det svenska 
    alfabetet motsvara ett unikt tal i \(\Z_{29}\).
    För alla \(k\in \K\) definierar vi
    \begin{align}
      \nonumber
      e_k(p) &= (p + k) \bmod 29, \text{ och } \\
      \nonumber
      d_k(c) &= (c - k) \bmod 29,
    \end{align}
    där \(p\in \P\) är en klartextbokstav och \(c = e_k(p)\in \C\) är 
    motsvarande kryptotextbokstav.
  \end{definition}
\end{frame}

\begin{frame}{\insertsubsectionhead}{Caesarchiffer}
  \begin{example}\label{ex:shiftdef}
    Låt oss numrera bokstäverna i det svenska alfabetet enligt index med start 
    från noll.
    Då får vi att textsträngen ''hej'' skulle kunna motsvaras av tupeln \(p 
    = (7, 4, 9)\).
    Om vi låter nyckeln \(k\in \K\) vara \(2\) får vi att
    \begin{align}
      \nonumber
      c = e_2(p) &= (e_2(7), e_2(4), e_2(9)) \\
      \nonumber
        &= (9, 6, 11).
    \end{align}
    Om vi översätter tillbaka till bokstäver får vi att \(c\) motsvarar 
    strängen ''JGL''.
  \end{example}

  \begin{example}
    \label{ex:CaesarSkatten}
    Om vi krypterar ordet \emph{skatten} blir det \emph{UMCVVGP}.
  \end{example}
\end{frame}

\begin{frame}{\insertsubsectionhead}{Caesarchiffer}
  \begin{block}{Kryptanalys}
    \begin{itemize}
      \item Kan enkelt testa alla 29 nycklarna för hand.

      \item Kan titta efter upprepningar:
        \begin{itemize}
          \item Upprepade bokstäver är sannolikt konsonanter.
          \item Upprepade bokstavskombinationer är sannolikt vanliga ord.
        \end{itemize}

      \item Trots enkelheten att knäcka detta försökte terrorister seriöst att 
        använda systemet så sent som 2011 \cite{Register2011bjr}.

    \end{itemize}
  \end{block}
\end{frame}

\begin{frame}{\insertsubsectionhead}
  \begin{definition}[Substitutionschiffer]\label{def:substitutionCipher}
    Låt \(A\) vara vårt alfabet och låt \(\P = \C = A\).
    Vidare låt \(\K\) bestå av alla möjliga permutationer av \(A\).
    För varje permutation \(\pi\in \K\) definierar vi att
    \begin{align}
      \nonumber
      e_\pi(p) &= \pi(p), \text{ och } \\
      \nonumber
      d_\pi(c) &= \pi^{-1}(c),
    \end{align}
    där \(\pi^{-1}\) är den inverterade permutationen \(\pi\), \(p\in \P\) är 
    en klartextbokstav och \(c = e_\pi(p)\in \C\) är motsvarande 
    kryptotextbokstav.
  \end{definition}
\end{frame}

\begin{frame}{\insertsubsectionhead}
  \begin{table}
    \centering\small
    \begin{tabular}{c|ccccccccccccccc}
      \toprule
      \(\alpha\) & a & b & c & d & e & f & g & h & i & j & k & l & m & n & o \\
      \(\pi(\alpha)\) 
      & C & M & Q & F & Z & Ö & I & J & P & L & D & N & O & K & D \\
      \midrule
      \(\alpha\) & p & q & r & s & t & u & v & w & x & y & z & å & ä & ö \\
      \(\pi(\alpha)\) 
      & R & S & T & Å & V & Y & X & W & G & U & Ä & H & A & B \\
      \bottomrule
    \end{tabular}
    \caption{Nyckel för att kryptera med ett substitutionschiffer.
      Gemener används som klartextalfabete och versaler som kryptoalfabete.}
    \label{tbl:Substitutionschiffer}
  \end{table}
\end{frame}

\begin{frame}{\insertsubsectionhead}
  \begin{example}
    Vi låter \(A\) vara det svenska alfabetet.
    Nyckeln \(\pi\in \K\) ges av föregående tabell.
    Då får vi att \(e_\pi(h) = J, e_\pi(e) = Z, e_\pi(j) = L\).
  \end{example}

  \begin{example}
    \label{ex:SubstitutionSkatten}
    Om vi krypterar ordet \emph{skatten} blir det \emph{ÅDCVVZK}.
  \end{example}
\end{frame}

\begin{frame}{\insertsubsectionhead}
  \begin{table}
    \centering\tiny
    \begin{tabular}{r|cccccccccc}
      \toprule
      \(\alpha\) & a & b & c & d & e & f & g & h & i & j  \\
      \(\Pr(\x = \alpha)\) & 0.063  & 0.000 & 0.000 & 0.031 & 0.156 & 0.000 
      & 0.031 & 0.094 & 0.064 & 0.000 \\
      \midrule
      \(\alpha\) & k & l & m & n & o & p & q & r & s & t \\
      \(\Pr(\x = \alpha)\) & 0.000 & 0.063 & 0.000 & 0.094 & 0.031 & 0.000 
      & 0.000 & 0.031 & 0.156 & 0.125 \\
      \midrule
      \(\alpha\) & u & v & w & x & y & z & å & ä & ö \\
      \(\Pr(\x = \alpha)\) & 0.000 & 0.000 & 0.031 & 0.031 & 0.000 & 0.000 
      & 0.000 & 0.000 & 0.000 \\
      \bottomrule
    \end{tabular}
    \caption{Tabell av sannolikhetsfördelningen för den stokastiska variabeln 
    \(\x\) som antar bokstäver i meningen ''anenglishtexthasnoswedishletters'', 
    angiven med tre decimalers noggrannhet.}
    \label{tbl:SannolikhetstabellKlartext}
  \end{table}
\end{frame}

\begin{frame}{\insertsubsectionhead}{Sannolikhetsfördelningar}
  \begin{itemize}
    \item Likformig sannolikhetsfördelning (uniform distribution).
    \item Oberoende.
  \end{itemize}
\end{frame}

\begin{frame}{\insertsubsectionhead}
  \begin{table}
    \centering\tiny
    \begin{tabular}{r|cccccccccccc}
      \toprule
      \(\alpha\) & A & B & C & D & E & F & G & H & I & J \\
      \(\Pr(\y = \alpha)\) & 0.000 & 0.000 & 0.063  & 0.000 & 0.000 & 0.031 
      & 0.156 & 0.000 & 0.031 & 0.094 \\
      \midrule
      \(\alpha\) & K & L & M & N & O & P & Q & R & S & T \\
      \(\Pr(\y = \alpha)\) & 0.064 & 0.000 & 0.000 & 0.063 & 0.000 & 0.094 
      & 0.031 & 0.000 & 0.000 & 0.031 \\
      \midrule
      \(\alpha\) & U & V & W & X & Y & Z & Å & Ä & Ö \\
      \(\Pr(\y = \alpha)\) & 0.156 & 0.125 & 0.000 & 0.000 &  0.031 & 0.031 
      & 0.000 & 0.000 & 0.000 \\
      \bottomrule
    \end{tabular}
    \caption{Tabell av sannolikhetsfördelningen för den stokastiska variabeln 
    \(\y\) som antar bokstäver i meningen ''CPGPINKUJVGZVJCUPQUYGFKUJNGVVGTU'', 
    angiven med tre decimalers noggrannhet.}
    \label{tbl:SannolikhetstabellKryptotext}
  \end{table}
\end{frame}

\begin{frame}{\insertsubsectionhead}{Vigenèrechiffer}
  \begin{definition}[Vigenèrechiffer]
    Låt \(n\) vara ett positivt heltal.
    Definiera att \(\P = \C = \K = (\Z_{29})^n\).
    För alla nycklar \(k = (k_1,\ldots,k_n)\in \K\), klartexter \(p = (p_1, 
    \ldots, p_n)\in \P\) och kryptotexter \(c = (c_1,\ldots,c_n)\in \C\) 
    definierar vi att
    \begin{align}
      \nonumber
      e_k(p) &= (p_1 + k_1, \ldots, p_n + k_n), \text{ och } \\
      \nonumber
      d_k(c) &= (c_1 - k_1, \ldots, c_n - k_n),
    \end{align}
    där alla operationer utförs i \(\Z_{29}\).
  \end{definition}
\end{frame}

\begin{frame}{\insertsubsectionhead}{Vigenèrechiffer}
  \begin{table}
    \centering
    \begin{tabular}{l|cccccccccc}
      \toprule
      Klartext & a & b & c & d & e & f & g & h & i & j \\
      A        & A & B & C & D & E & F & G & H & I & J \\
      B        & B & C & D & E & F & G & H & I & J & K \\
      C        & C & D & E & F & G & H & I & J & K & L \\
      \midrule
      Klartext & k & l & m & n & o & p & q & r & s & t \\
      A        & K & L & M & N & O & P & Q & R & S & T \\
      B        & L & M & N & O & P & Q & R & S & T & U \\
      C        & M & N & O & P & Q & R & S & T & U & V \\
      \midrule
      Klartext & u & v & w & x & y & z & å & ä & ö \\
      A        & U & V & W & X & Y & Z & Å & Ä & Ö \\
      B        & V & W & X & Y & Z & Å & Ä & Ö & A \\
      C        & W & X & Y & Z & Å & Ä & Ö & A & B \\
      \bottomrule
    \end{tabular}
    \caption{Vigenèrechiffer med nyckeln \emph{ABC}.}
    \label{tbl:Vigenerechiffer}
  \end{table}
\end{frame}

\begin{frame}[fragile]{\insertsubsectionhead}{Vigenèrechiffer}
  \begin{example}
    \label{ex:VigenereSkatten}
    Om vi vill kryptera order \emph{skatten} ska bokstäverna i nyckeln användas 
    enligt
    \begin{verbatim}
      skatten
      ABCABCA
    \end{verbatim}
    och vi får alltså \emph{SLCTUGN} genom att använda de olika Caesarchiffren 
    i föregående tabell.
  \end{example}
\end{frame}

\begin{frame}[fragile]{\insertsubsectionhead}{Vigenèrechiffer}
  \begin{example}
    \label{ex:VigenereMedUpprepning}
    Ett Vigenèrechiffer med nyckeln \emph{ABCD} används för att kryptera texten 
    \emph{cryptoisshortforcryptography}.
    \begin{verbatim}
      Nyckel:     ABCDABCDABCDABCDABCDABCDABCD
      Klartext:   cryptoisshortforcryptography
      Kryptotext: CSASTPKVSIQUTGQUCSASTPIUAQJB
    \end{verbatim}
    Avståndet mellan den upprepade texten \emph{CSASTP} är 16, från första 
    tecken till första tecken.
    De möjliga nyckellängderna är alltså \(16, 8, 4, 2\) eller \(1\).
  \end{example}
\end{frame}

\begin{frame}[fragile]{\insertsubsectionhead}{Vigenèrechiffer}
  \begin{example}
    \label{ex:VigenereKolumner}
    Ett Vigenèrechiffer med nyckeln \emph{ABCD} används för att kryptera texten 
    \emph{cryptoisshortforcryptography}.
    \begin{verbatim}
      Nyckel:     ABCD              ABCD
      Klartext:   cryp  Kryptotext: CSAS
                  tois              TPKV
                  shor              SIQU
                  tfor              TGQU
                  cryp              CSAS
                  togr              TPIU
                  aphy              AQJB
    \end{verbatim}
  \end{example}
\end{frame}

\subsection{Permutationschiffer}

\begin{frame}{\insertsubsectionhead}
  \begin{definition}[Permutationschiffer]\label{def:permutationCipher}
    Låt \(n\) vara ett positivt heltal och \(A\) ett alfabet.
    Låt också \(\P = \C = A^n\) och låt \(\K\) vara alla möjliga permutationer 
    av mängden \(\{1,\ldots,n\}\).
    För en permutation \(\pi\in \K\) definierar vi
    \begin{align}
      \nonumber
      e_\pi(p_1,\ldots,p_n) = (p_{\pi(1)},\ldots,p_{\pi(n)}),
    \end{align}
    för alla klartexter \(p = (p_1,\ldots,p_n)\in \P\), och
    \begin{align}
      \nonumber
      d_\pi(c_1,\ldots,c_n) = (c_{\pi^{-1}(1)},\ldots,c_{\pi^{-1}(n)}),
    \end{align}
    för alla kryptotexter \(c = (c_1,\ldots,c_n)\in \C\) och där \(\pi^{-1}\) 
    är den inverterade permutationen \(\pi\).
  \end{definition}
\end{frame}

\begin{frame}{\insertsubsectionhead}
  \begin{table}
    \centering\small
    \begin{tabular}{r|rrrrrrrrrrrrrr}
      \toprule
      \(i\)       & 1 & 2 & 3 & 4 & 5 & 6 & 7 & 8 & 9 & 10 & 11 & 12 & 13 & 14 
      \\
      \(\pi(i)\)  & 1 & 8 & 2 & 9 & 3 & 10 & 4 & 11 & 5 & 12 & 6 & 13 & 7 & 14 
      \\
      \bottomrule
    \end{tabular}
    \caption{Definitionen av permutationen \(\pi\).}
    \label{tbl:pi}
  \end{table}

  \begin{example}\label{ex:permutationEnDagIJuni}
    Låt \(n = 14\).
    Permutationen \(\pi\in \K\) definieras enligt tabellen ovan.
    För att kryptera använder vi \(e_\pi\in \E\).
    Om vi låter \(p = (p_1, \ldots, p_n)\) vara vår klartext ''en dag i juni'', 
    får vi att \(c = e_\pi(p) = (p_1, p_8, p_2, p_9, \ldots, p_7, p_{14})\) och 
    således att \(c\) är vår kryptotext ''EIN\_\_JDUANGI\_.''.

    Vi avkrypterar på samma sätt med hjälp av \(\pi^{-1}\).
  \end{example}
\end{frame}

%\begin{frame}{\insertsubsectionhead}
%  \begin{definition}[Hillchiffret]
%  \end{definition}
%\end{frame}

\subsection{Perfekt sekretess}

\begin{frame}{\insertsubsectionhead}
  \begin{definition}\label{def:perfectSecrecy}
    Ett kryptosystem \((\P, \C, \K, \E, \D)\) sägs ha \emph{perfekt sekretess} 
    om \(\Pr(\p = p\mid \c = c) = \Pr(\p = p)\) för alla \(p\in \P\) och \(c\in 
    \C\).
    Det vill säga, sannolikheten a posteriori att en klartext är \(p\) om 
    kryptotexten är \(c\) är densamma som sannolikheten a priori att klartexten 
    är \(p\).
  \end{definition}
\end{frame}

\begin{frame}{\insertsubsectionhead}
  \begin{theorem}[Shannons sats]\label{thm:perfectSecrecy}
    Antag att \((\P, \C, \K, \E, \D)\) är ett kryptosystem sådant att \(|\K| 
    = |\C| = |\P|\).
    Detta kryptosystem tillhandahåller perfekt sekretess om och endast om
    varje nyckel \(k\in \K\) används med lika sannolikhet \(1/|\K|\) och det 
    för varje klartext \(p\in \P\) och kryptotext \(c\in \C\) finns en unik 
    nyckel \(k\in \K\) sådan att \(e_k(p) = c\).
  \end{theorem}
\end{frame}

\begin{frame}{\insertsubsectionhead}
  \begin{definition}[One-time Pad]
    Låt \(n\) vara ett positivt heltal.
    Definiera att \(\P = \C = \K = (\Z_2)^n\).
    För alla nycklar \(k = (k_1,\ldots,k_n)\in \K\), klartexter \(p = (p_1, 
    \ldots, p_n)\in \P\) och kryptotexter \(c = (c_1,\ldots,c_n)\in \C\) 
    definierar vi att
    \begin{align}
      \nonumber
      e_k(p) &= (p_1 + k_1, \ldots, p_n + k_n),
    \end{align}
    där alla operationer utförs i \(\Z_2\), och därefter definierar vi att 
    \(d_k = e_k\).

    Nyckeln \(k\in \K\) måste väljas slumpmässigt och får aldrig återanvändas.
  \end{definition}
\end{frame}

\begin{frame}{\insertsubsectionhead}
  \begin{itemize}
    \item Perfekt sekretess är krångligt att uppnå.
    \item Använder ''beräkningsmässigt säker'' istället.
  \end{itemize}
\end{frame}


\section{Modern symmetrisk kryptering}

\subsection{Data Encryption Standard (DES)}

\begin{frame}{\insertsubsectionhead}
  \begin{figure}
    \includegraphics[height=0.7\textheight]{feistel.eps}
    \caption{Feistelstruktur.
    Bild: \cite{Stallings2011can}.}
  \end{figure}
\end{frame}

\begin{frame}{\insertsubsectionhead}
  \begin{figure}
    \includegraphics[height=0.7\textheight]{des-round.eps}
    \caption{En runda i DES.
    Bild: \cite{Stallings2011can}.}
  \end{figure}
\end{frame}

\begin{frame}{\insertsubsectionhead}
  \begin{figure}
    \includegraphics[width=0.5\textwidth]{3des.eps}
    \caption{DES tillämpad i 3DES.
    Bild: \cite{Stallings2011can}.}
  \end{figure}
\end{frame}

\subsection{Advanced Encryption Standard (AES)}

\begin{frame}{\insertsubsectionhead}
  \begin{figure}
    \includegraphics[height=0.7\textheight]{aes.eps}
    \caption{AES översikt.
    Bild: \cite{Stallings2011can}.}
  \end{figure}
\end{frame}

\begin{frame}{\insertsubsectionhead}
  \begin{figure}
    \includegraphics[height=0.7\textheight]{aes-round.eps}
    \caption{En runda i AES.
    Bild: \cite{Stallings2011can}.}
  \end{figure}
\end{frame}

\section{Pseudoslumptal och strömchiffer}

\subsection{Pseudoslumptal}

\begin{frame}{\insertsubsectionhead}
  \begin{itemize}
    \item Pseudorandom number generator.
    \item True random number generator.
    \item Pseudorandom function.
  \end{itemize}
\end{frame}

\subsection{Strömchiffer}

\begin{frame}{\insertsubsectionhead}
  \begin{itemize}
    \item Pseudorandom number generator som utgår från nyckeln.
  \end{itemize}
\end{frame}


\section{Block Modes of Operation}

\subsection{Introduktion}

\begin{frame}{\insertsubsectionhead}
  \begin{itemize}
    \item Ett blockchiffer i standardutförande är inte särskilt säkert om vi 
      vill kryptera mer än ett block med samma nyckel.

    \item För att åtgärda detta använder vi olika 
      ''\foreignlanguage{english}{modes of operation}'' för blockchiffer.

  \end{itemize}
\end{frame}

\begin{frame}{\insertsubsectionhead}
  \begin{itemize}
    \item Det mode of operation som vi hittills använt utan att benämna det som 
      ett sådant är ''\foreignlanguage{english}{electronic code-book mode}'' 
      (ECB).

    \item Detta går som vi nämnt tidigare ut på att vi delar upp meddelandet 
      enligt blockstorleken och krypterar del för del.

  \end{itemize}
\end{frame}

\subsection{Några andra modes of operation}

\begin{frame}{\insertsubsectionhead}
  \begin{figure}
    \includegraphics[height=0.7\textheight]{cbc.eps}
    \caption{Cipher block chaining (CBC) mode.
    Bild: \cite{Stallings2011can}.}
  \end{figure}
\end{frame}

\begin{frame}{\insertsubsectionhead}
  \begin{figure}
    \includegraphics[height=0.7\textheight]{cfb.eps}
    \caption{Cipher feedback (CFB) mode.
    Bild: \cite{Stallings2011can}.}
  \end{figure}
\end{frame}

\begin{frame}{\insertsubsectionhead}
  \begin{figure}
    \includegraphics[height=0.7\textheight]{ctr.eps}
    \caption{Counter (CTR) mode.
    Bild: \cite{Stallings2011can}.}
  \end{figure}
\end{frame}


%%%%%%%%%%%%%%%%%%%%%%

\begin{frame}{Referenser}
  \small
  \printbibliography
\end{frame}

\end{document}
