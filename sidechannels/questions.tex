\question\label{q:sidechannels}
% examgen: sidechannels:E
\begin{parts}
  \part[3] What is a side-channel attack?

  \begin{solution}
    A side channel is an unintended channel emitting information which is due 
    to physical implementation flaws and not theoretical weaknesses or forcing 
    attempts.
  \end{solution}

  \part[1] Is social engineering a side channel?

  \begin{solution}
    No, social engineering would qualify as forcing; it is not due to the 
    hardware implementation, it is due to the human.
  \end{solution}
\end{parts}


\question[4]\label{q:sidechannels}
% examgen: sidechannels:E
Give an example of a side-channel attack and motivate why it is a side channel.

\begin{solution}
  A side channel is an unintended channel emitting information which is due 
  to physical implementation flaws and not theoretical weaknesses or forcing 
  attempts.

  (2 points) Extracting the secret key from a device by measuring energy 
  consumption or electromagnetic emissions while the device performs 
  computations using the secret key.

  (1 point) This is a side channel since it relies on a weakness in the 
  hardware implementation.
  (1 point) It is further an active attack since we might need the device to 
  perform operations on certain ciphertexts (or plaintexts).
\end{solution}


\question\label{q:sidechannels}
% examgen: sidechannels:C
Does knowledge about the hardware give any advantage? As a designer, as an 
attacker?


\question[2]\label{q:sidechannels}
% examgen: sidechannels:E
Describe an attack scenario where a side-channel is of central interest.

\begin{solution}
  The adversary is interested in learning classified information.
  They set up a device which records electromagnetic emissions to reconstruct 
  the image on a screen, thus when a target works with the classified data on 
  the computer the adversary sees the same image.
  This is a passive attack.
\end{solution}


\question[3]\label{q:sidechannels}
% examgen: sidechannels:E
Give an example of a passive side-channel attack.

\begin{solution}
  The adversary is interested in learning classified information.
  They set up a device which records electromagnetic emissions to reconstruct 
  the image on a screen, thus when a target works with the classified data on 
  the computer the adversary sees the same image.
  This is a passive attack since we only need to record.
\end{solution}


\question[3]\label{q:sidechannels}
% examgen: sidechannels:E
Given an example of an active side-channel attack.

\begin{solution}
  Extracting the secret key from a device by measuring energy consumption or 
  electromagnetic emissions while the device performs computations using the 
  secret key.
  It is an active attack since we might need the device to perform operations 
  on certain ciphertexts (or plaintexts).
\end{solution}


\question[3]\label{q:sidechannels}
% examgen: sidechannels:E
What is a covert channel?

\begin{solution}
  A covert channel is a mechanism that was not designed for communication but 
  which can nonetheless be abused to allow information to flow in a way which 
  is not allowed in the security policy.
  The problem commonly arises when two clearance levels shares resources.
\end{solution}


\question[4]\label{q:sidechannels}
% examgen: sidechannels:E
Describe a scenario where a covert channel is used.

\begin{solution}
  A server is anonymous (e.g.\ a Tor hidden service), i.e.\ you may access the 
  server but not know its location.
  Part of the server's service is giving the time.
  It has been shown that the variations in the system clock depend on the 
  ambient temperature.
  This means that by studying how the time on the server varies over day and 
  night and over the seasons, we can eventually figure out the ambient 
  temperature.
  From the ambient temperature we can later deduce the geographical location of 
  the server.
\end{solution}


\question\label{q:sidechannels}
% examgen: sidechannels:E:C
\begin{parts}
  \part[4] Give an example of a covert channel and
  \begin{solution}
    A server is anonymous (e.g.\ a Tor hidden service), i.e.\ you may access 
    the server but not know its location.
    Part of the server's service is giving the time.
    It has been shown that the variations in the system clock depend on the 
    ambient temperature.
    This means that by studying how the time on the server varies over day and 
    night and over the seasons, we can eventually figure out the ambient 
    temperature.
    From the ambient temperature we can later deduce the geographical location 
    of the server.
  \end{solution}

  \part[3] how we can prevent (or at least limit) it.
  \begin{solution}
    We can lower the resolution in the time-stamps the server gives, e.g.\ by 
    not giving seconds.
    This lowers the bandwidth of the covert channel, perhaps so that the attack 
    is infeasible.
    We could also sync the servers clock more often, e.g.\ by using the Network 
    Time Protocol.
    However, the only way to prevent it is by not revealing the time of the 
    server's system clock.
  \end{solution}
\end{parts}


\question\label{q:sidechannels}
% examgen: sidechannels:E:C
\begin{parts}
  \part[3] What is a covert channel?
  \begin{solution}
    A covert channel is a mechanism that was not designed for communication but 
    which can nonetheless be abused to allow information to flow in a way which 
    is not allowed in the security policy.
    The problem commonly arises when two clearance levels shares resources.
  \end{solution}

  \part[3] Describe a general approach to handling them.
  \begin{solution}
    The general approach is to limit the bandwidth of the channel.
    One way to achieve this is by adding random noise, e.g.\ adding dummy 
    traffic so that the adversary must distinguish the dummy traffic from the 
    real one.
    Another approach is to simply rate-limit the operation that causes the 
    problem, e.g.\ students are not allowed to go to the toilet too often or do 
    certain movements repeatedly.
  \end{solution}
\end{parts}
