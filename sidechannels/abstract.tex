When looking at secure systems it is easy to assume they are safe just because 
the secret keys are not directly reachable.
This is not always true.
Even if the key storage is unreachable, there is some information that can be 
extracted anyway.
For instance, the time an operation takes to perform reveals some information 
about the operands.
This is what is called side-channel information.

The \acp{ILO} are that you are able
\begin{itemize}
  \item to \emph{reflect} on the security of implementations, or lack thereof, 
    from the aspect of side-channels.
\end{itemize}

An overview of this area is provided in~\cite[Ch.~17]{Anderson2008sea}.
An interesting paper on this topic is 
\citetitle{genkin2013rsa}~\cite{genkin2013rsa} where the authors extract RSA 
keys using acoustic side-channels, \ie they analyse the sound emitted by the 
electrical circuitry to find the computations done and hence derive the RSA key 
used.
