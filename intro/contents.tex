\mode*

% Since this a solution template for a generic talk, very little can
% be said about how it should be structured. However, the talk length
% of between 15min and 45min and the theme suggest that you stick to
% the following rules:  

% - Exactly two or three sections (other than the summary).
% - At *most* three subsections per section.
% - Talk about 30s to 2min per frame. So there should be between about
%   15 and 30 frames, all told.


\section{Introduction}

\subsection{History}

\begin{frame}
  \begin{itemize}
    \item The word has its origin in greek~\footfullcite{OED2013cg}:
      \begin{description}
        \item[\ibygr{krupto's}] (\emph{kryptos}) meaning 
          hidden~\footfullcite{OED2013c}.
        \item[\ibygr{gra'fos}] (\emph{graphos}) meaning 
          writing~\footfullcite{OED2013g}.
      \end{description}

      \pause{}

    \item The area has been around for ages.

      \pause{}

    \item We should not confuse it with \emph{steganography}.
    \item Steganography concerns hiding a message's \emph{existence}.
    \item Cryptography concerns hiding a message's \emph{contents}.

  \end{itemize}
\end{frame}

\begin{frame}
  \begin{itemize}
    \item Then it was an art, now it's a science.

      \pause{}

    \item People used \enquote{clever} constructions.
    \item These were thought to be secure: \enquote{How can anyone figure this 
        out?}

      \pause{}

    \item Well, it turns out that there are always a lot of people with a lot 
      of time and motivation \dots
  \end{itemize}
\end{frame}

\subsection{Kerckhoff's Principle}

\begin{frame}
  \begin{block}{A quote\footfullcite{KerckhoffsPrinciple}}
    \begin{displayquote}\relax
      [A cryptosystem] should not require secrecy, and it should not be 
      a problem
      if it falls into the enemy hands;
    \end{displayquote}
  \end{block}

  \pause{}

  \begin{block}{Kerckhoff's Principle}
    \begin{itemize}
      \item No security-by-obscurity
      \item The key should be the only secret
    \end{itemize}
  \end{block}
\end{frame}

\begin{frame}
  \begin{remark}
    \begin{itemize}
      \item This doesn't mean we must tell the adversary what we're using.
      \item But we shouldn't loose any security if we do.
    \end{itemize}
  \end{remark}
\end{frame}

\subsection{Outline}

\begin{frame}
  \begin{description}
    \item[Shared-key]
      Stems from the classical crypto where a key is shared between two users.

      \pause

    \item[Public-key]
      This is more modern crypto, from 1970s.
      Each user has a public and a private key.

      \pause

    \item[Counter-intuitive]
      More modern, from 1980s and onwards.
      How to do computations on secret inputs, prove knowledge without revealing 
      of what.
  \end{description}
\end{frame}


%%%%%%%%%%%%%%%%%%%%%%

\begin{frame}[allowframebreaks]
  \printbibliography
\end{frame}

