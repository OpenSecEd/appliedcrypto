\mode*

% Since this a solution template for a generic talk, very little can
% be said about how it should be structured. However, the talk length
% of between 15min and 45min and the theme suggest that you stick to
% the following rules:  

% - Exactly two or three sections (other than the summary).
% - At *most* three subsections per section.
% - Talk about 30s to 2min per frame. So there should be between about
%   15 and 30 frames, all told.


\section{Shared-key cryptography}

\subsection{Ciphers}

\begin{frame}
  \begin{idea}
    \begin{itemize}
      \item Alice and Bob share a (small) common secret.

        \pause{}

      \item Alice takes a message, combines it with the secret, sends it to 
        Bob.

        \pause{}

      \item If Eve captures the whatever Alice sent, she shouldn't learn 
        anything about the message.

        \pause{}

      \item Bob combines what he received with the secret and gets the message.
    \end{itemize}
  \end{idea}
\end{frame}

\begin{frame}
  \begin{block}{Block-cipher encryption}
    \begin{description}
      \item[Input] A fixed-sized \emph{key} \(\Key{}\), a fixed-sized block of 
        \emph{plaintext} \(p\).
      \item[Output] A fixed-sized block of \emph{ciphertext} \(c\).
      \item[Notation] \(\Enc[\Key{}]{p} = c\)
    \end{description}
  \end{block}

  \pause{}

  \begin{block}{Block-cipher decryption}
    \begin{description}
      \item[Input] A fixed-sized \emph{key} \(\Key{}\), a fixed-sized block of 
        \emph{ciphertext} \(c\).
      \item[Output] A fixed-sized block of \emph{plaintext} \(p\).
      \item[Notation] \(\Dec[\Key{}]{c} = p\)
    \end{description}
  \end{block}
\end{frame}

\begin{frame}
  \begin{definition}[Crypto 
    system\footfullcite{Stinson2006cta}]\label{CryptoSystem}
    \begin{itemize}
        \item A \emph{crypto system} is a tuple \((\M, \C, \K, \E, \D)\), 
          where:
          \begin{itemize}
            \item \(\M\) is a finite set of \emph{plaintexts} or messages,
            \item \(\C\) is a finite set of \emph{ciphertexts},
            \item \(\K\) is the \emph{keyspace}, a finite set of keys.
            \item \(\E\) and \(\D\) are the sets of encryption and decryption
              rules, respectively.
          \end{itemize}

          \pause{}

        \item For every \(k\in \K\) there is a \(\Enc[k]{}\in \E\) and 
          \(\Dec[k]{}\in \D\) such that
          \begin{itemize}
            \item \(\Enc[k]{}\colon \M\to \C\) and \(\Dec[k]{}\colon \C\to 
                \M\) are functions and
            \item \(\Dec[k]{\Enc[k]{m}} = m\) for all plaintexts \(m\in 
                \M\).
          \end{itemize}
      \end{itemize}
  \end{definition}
\end{frame}

\begin{frame}
  \begin{definition}[Shift Cipher]\label{ShiftCipher}
    \begin{itemize}
      \item Let \(\M = \C = \K = \Z_{29}\)
      \item For each \(k\in \K\) we define
        \begin{align}
          \nonumber
          \Enc[k]{m} &= (m + k) \bmod 29, m\in \M, \text{\ och } \\
          \nonumber
          \Dec[k]{c} &= (c - k) \bmod 29, c\in \C.
        \end{align}
    \end{itemize}
  \end{definition}

  \pause{}

  \begin{example}
    \begin{itemize}
      \item \(\Enc[3]{7} = 7+3 \bmod 29 = 10\)\hfill h\(\to\)J
      \item \(\Enc[3]{4} = 4+3 \bmod 29 = 7\)\hfill e\(\to\)G
      \item \(\Enc[3]{9} = 9+3 \bmod 29 = 12\)\hfill j\(\to\)L
    \end{itemize}
  \end{example}
\end{frame}

\begin{frame}
  \begin{remark}
    \begin{itemize}
      \item The shift cipher is a classical cipher --- also know as the Caesar 
        Cipher.
      \item It's easily broken \emph{by hand}!
      \item It's used here for illustrative purposes.
    \end{itemize}
  \end{remark}
\end{frame}

\begin{frame}
  \begin{exercise}
    \begin{itemize}
      \item What do we have to do to set this up between two parties, say Alice 
        and Bob?
      \item What problems do we have to solve?
    \end{itemize}
  \end{exercise}
\end{frame}

\subsection{Security}

\begin{frame}
  \begin{definition}[Perfect secrecy~\footfullcite{ShannonSecrecy}]
    \begin{itemize}
      \item Cryptosystem \((\M, \C, \K, \E, \D)\).
      \item Stochastic variables \(M, C\).
      \item \emph{Perfect secrecy} if and only if \[\Pr(M = m\mid C = c) 
          = \Pr(M = m)\] for all \(m\in \M\) and \(c\in \C\).
    \end{itemize}
  \end{definition}

  \pause{}

  \begin{remark}
    Equivalent to \(H(M\mid C) = H(M)\), i.e.\ ciphertext does not reveal 
    anything about plaintext.
  \end{remark}
\end{frame}

\begin{frame}
  \begin{theorem}[Shannon's theorem]
    \begin{itemize}
      \item Assume cryptosystem \((\M, \C, \K, \E, \D)\) such that \(|\K| 
          = |\C| = |\M|\).

        \pause{}

      \item This provides perfect secrecy if and only if
        \begin{enumerate}
          \item every key \(k\in \K\) is used with equal probability 
            \(1/|\K|\),
          \item for every plaintext \(m\in \M\) and \(c\in \C\) there is 
            a unique key such that \(\Enc[k]{m} = c\).
        \end{enumerate}
    \end{itemize}
  \end{theorem}
\end{frame}

\begin{frame}
  \begin{example}[One-time Pad]
    \begin{itemize}
      \item Let \(n\) be a positive integer.
      \item Let \(\M = \C = \K = (\Z_2)^n\).

        \pause{}

      \item For each key \(k = (k_1, \ldots, k_n)\in \K\), plaintexts \(m 
          = (m_1, \ldots, m_n)\in \M\) and ciphertexts \(c = (c_1, \ldots, 
          c_n)\in \C\) we define
        \[\Enc[k]{m} = (m_1 + k_1, \ldots, m_n + k_n)\]

        \pause{}

      \item We also define \(\DecOp = \EncOp\).

        \pause{}

      \item \(k\in \K\) must be chosen uniformly randomly for each encryption.
    \end{itemize}
  \end{example}
\end{frame}

\begin{frame}
  \begin{definition}[\Acl{PRP}, \acs{PRP}\footfullcite{KatzLindell-v1}]
    \begin{itemize}
      \item Let \(F\colon \{0,1\}^s\times \{0, 1\}^n\to \{0,1\}^n\).

        \pause{}

      \item \(F\) is \iac{PRP} if
        \begin{enumerate}
          \item for any \(k\in \{0, 1\}^s\), \(F\) is a bijection;

            \pause{}

          \item for any \(k\in \{0, 1\}^s\), we can \enquote{efficiently} 
            evaluate \(F_k(x)\);

            \pause{}

          \item for all \enquote{efficient} distinguishers~\(D\),
            \[\left|\Pr[D^{F_k}(1^n) = 1] - \Pr[D^{f_n}(1^n) = 1]\right| 
              < \epsilon(s)\] if we choose \(k\in \{0,1\}^s\) and the random 
            permutation \(f_n\) uniformly at random.
        \end{enumerate}
    \end{itemize}
  \end{definition}
\end{frame}


%%%%%%%%%%%%%%%%%%%%%%

\begin{frame}[allowframebreaks]
  \printbibliography{}
\end{frame}

