Shared-key cryptography is the traditional approach to cryptography and the 
aspect of cryptography that comes as intuitive for most people.
We instroduce definitions of security, theoretical models, how they relate to 
practical implementations and provable security.

The \acp{ILO} are that you should be able
\begin{itemize}
  \item to \emph{understand} what properties can be achieved with shared-key 
    cryptography and their definitions of security.
\end{itemize}

The basics are covered by
Chapter 5 in Anderson's \citetitle{Anderson2008sea}~\cite{Anderson2008sea} and
Chapter 14 in Gollmann's \citetitle{Gollmann2011cs}~\cite{Gollmann2011cs}.
%We will also cover some topics from 
%\citetitle{KatzLindell-v1}~\cite{KatzLindell-v1} and 
%\citetitle{GoldreichFOC-1}~\cite{GoldreichFOC-1}.
For the remaining topics, we refer to the \citetitle{EOCS}~\cite{EOCS} (and 
cited papers and books).
