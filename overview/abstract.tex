Cryptography has a central role in security.
To fully understand how many security mechanisms can be implemented we need 
cryptography.
For this reason, we also need higher-level knowledge about what can be achieved 
with cryptography to not limit our thoughts about possible solutions.
This learning session is intended to give a high-level overview of 
cryptography: \ac{SKE}, \ac{PKE}, digital signatures, \ac{ZKP} and \ac{MPC}.
In particular, the \acp{ILO} are that you should be able to
\begin{itemize}
  \item \emph{understand} what properties can be achieved with cryptography.
  \item \emph{analyse} a situation and \emph{suggest} what cryptographic 
    properties are desirable.
\end{itemize}

The basics are covered by
Chapter 5 in Anderson's \citetitle{Anderson2008sea}~\cite{Anderson2008sea} and
Chapter 14 in Gollmann's \citetitle{Gollmann2011cs}~\cite{Gollmann2011cs}.
(To practice your understanding of these mechanisms it is recommended to do 
exercises 14.2, 14.3 and 14.7 in~\cite{Gollmann2011cs}.)
%We will also cover some topics from 
%\citetitle{KatzLindell-v1}~\cite{KatzLindell-v1} and 
%\citetitle{GoldreichFOC-1}~\cite{GoldreichFOC-1}.
For the remaining topics, however, we refer to the \citetitle{EOCS}~\cite{EOCS} 
(and cited papers and books).
