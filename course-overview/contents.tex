\mode*

\section{The goal}

\begin{frame}[fragile]
  \begin{center}
    \Large
    Primitives
    \qquad
    \qquad
    Constructions
  \end{center}
\end{frame}

\begin{frame}[fragile]
  \begin{columns}
    \begin{column}{0.5\textwidth}
      \begin{block}{Primitives}
        \begin{itemize}
          \item Block ciphers
          \item Stream ciphers
          \item Hash functions
          \item Message authentication codes
          \item Digital signatures
          \item Public key encryption
          \item Zero-knowledge proofs
          \item Secure multi-party computation
        \end{itemize}
      \end{block}
    \end{column}
    \begin{column}{0.5\textwidth}
      \begin{block}{Constructions}
        \begin{itemize}
          \item Key exchange protocols
          \item Identification and authentication protocols
          \item Complex constructions from primitives
        \end{itemize}
      \end{block}
    \end{column}
  \end{columns}
\end{frame}

\subsection{Primitives}

\begin{frame}[fragile]
  \begin{example}[Primitives: OTP]
    \begin{itemize}
      \item The One-Time Pad is perfectly secure.
      \item It is a stream cipher.
      \item It's also malleable---predictable bit flips!
    \end{itemize}
  \end{example}
\end{frame}

\begin{frame}[fragile]
  \begin{example}[Primitives: RSA and ElGamal]
    \begin{itemize}
      \item RSA and ElGamal are public key encryption schemes.
      \item They secure the connection to our internet bank.
      \item Is it a good idea to send
        \begin{center}
          Enc(\texttt{``Transfer 1000 SEK to account 1234''}, k)
        \end{center}
        to the bank?
    \end{itemize}
  \end{example}
\end{frame}

\subsection{Constructions}

\begin{frame}[fragile]
  \begin{example}[Construction: BankID]
    \begin{itemize}
      \item BankID is a construction based on public key encryption.
      \item It is used to authenticate users.
      \item It is also used to sign transactions.
      \item How should this be constructed to minimize attacks?
    \end{itemize}
  \end{example}
\end{frame}

\begin{frame}[fragile]
  \begin{example}[Construction: Tor]
    \begin{itemize}
      \item Tor is a construction based on onion routing.
      \item It is used to anonymize internet traffic.
      \item It uses crypto a lot.
      \item How should this be constructed to minimize attacks?
    \end{itemize}
  \end{example}
\end{frame}

\begin{frame}[fragile]
  \begin{example}[Example: The Signal Protocol]
    \begin{itemize}
      \item The Signal Protocol is used for secure messaging.
      \item It provides end-to-end encryption for messages.
      \item Key features include perfect forward secrecy and deniability.
      \item Widely used in applications like WhatsApp and Facebook Messenger.
      \item How to do group messaging securely?
    \end{itemize}
  \end{example}
\end{frame}

\begin{frame}[fragile]
  \begin{example}[Example: Remote Attestation Protocols]
    \begin{itemize}
      \item Remote Attestation validates the integrity of a device.
      \item It leverages cryptographic proofs to verify software states.
      \item Commonly used in trusted computing environments.
      \item Helps in ensuring the device has not been tampered with.
      \item How can we use this and how can we do it correctly?
    \end{itemize}
  \end{example}
\end{frame}

\begin{frame}[fragile]
  \begin{example}[Example: Keyless Entry to Cars]
    \begin{itemize}
      \item Keyless entry systems use wireless communication.
      \item Protocols allow unlocking and starting the vehicle without a 
        physical key.
      \item Vulnerable to relay attacks if not properly secured.
    \end{itemize}
  \end{example}
\end{frame}

\subsection{Intended learning outcomes}

\begin{frame}[fragile]
  \begin{block}{Intended learning outcomes}
    After passing the course, the student should be able to:
    \begin{itemize}
      \item use basic terminology in 
        computer security and cryptography correctly
      \item describe cryptographic concepts and explain their security 
        properties
      \item find and use documentation of cryptographic libraries and standards
      \item identify and categorise threats against a cryptographic IT-system 
        at a conceptual level, suggest appropriate countermeasures and present 
        the reasoning to others
    \end{itemize}
  \end{block}
\end{frame}

\begin{frame}
  \begin{block}{Purpose}
    \dots in order to
    \begin{itemize}
      \item as citizen and engineer be able to discuss applied cryptography in 
        general, and risks of using/developing cryptography in particular
      \item in professional life and/or research and development project be 
        able to evaluate challenges in software development related to 
        cryptography.
    \end{itemize}
  \end{block}
\end{frame}

\begin{frame}
  \begin{block}{In other words ...}
    \begin{itemize}
      \item<+> You should be able to write your own version of BankID.
      \item<+> You should be able to make well-founded contributions to 
        discussions about things like ChatControl.
    \end{itemize}
  \end{block}
\end{frame}


\section{The content}

\subsection{Lectures}

\begin{frame}[fragile]
  \begin{block}{Lecture format}
    \begin{itemize}
      \item Most on campus
      \item Some flipped (watch videos before class)
      \item Some online (participate through Zoom)
    \end{itemize}
  \end{block}

  \begin{figure}
    \includegraphics[width=0.8\textwidth]{figs/lecture-mode.png}
    \caption{Screenshot from Canvas showing the lecture mode.}
  \end{figure}
\end{frame}

\begin{frame}[fragile]
  \begin{block}{Lecture content}
    \begin{itemize}
      \item Crypto primitives
      \item Practical applications
      \item Issues and higher level problems
    \end{itemize}
  \end{block}
\end{frame}

\begin{frame}[fragile]
  \begin{block}{Lecture content (cont.)}
    \begin{itemize}
      \item The lectures are complemented by reading material.
      \item \alert<2>{However, not all of them are published yet.}
    \end{itemize}
  \end{block}

  \begin{figure}
    \includegraphics[width=0.8\textwidth]{figs/lecture-mode.png}
    \caption{Screenshot from Canvas showing the lecture mode.}
  \end{figure}
\end{frame}

\subsection{Assignments}

\begin{frame}[fragile]
  \begin{block}{LADOK modules}
    \begin{itemize}
      \item LAB1
      \item INL1
    \end{itemize}
  \end{block}
\end{frame}

\begin{frame}[fragile]
  \begin{block}{LAB1, mandatory}
    \begin{itemize}
      \item Cryptanalysis of Ciphertexts
      \item Implement AES (Kattis Problem)
      \item AES presentation
      \item MANDATORY Seminar (31/1): usability (Sonja)  ON CAMPUS
      \item MANDATORY Seminar (pick 6/2): Impact considerations around crypto systems (Sonja)  ON CAMPUS
      \item MANDATORY Design Considerations (after the impact considerations seminar)
      \item MANDATORY Lab (20-21/2): Introduction to ProVerif  (Karl and Jesper)  ON CAMPUS
    \end{itemize}
  \end{block}
\end{frame}

\begin{frame}[fragile]
  \begin{block}{LAB1, optional}
    \begin{itemize}
      \item Optional: Cryptopals (C, B, A)
      \item Optional: Side channels (C, B, A)
      \item Optional: Secure multi-party computation (C, B, A)
    \end{itemize}
  \end{block}

  \begin{remark}[Higher grades]
    \begin{itemize}
      \item To get a higher grade, you need to do some of the optional 
        assignments.
    \end{itemize}
  \end{remark}
\end{frame}

\begin{frame}[fragile]
  \begin{block}{INL1}
    \begin{itemize}
      \item	INL1Quiz Cryptographic Concepts 2024
      \item	INL1Written
      \item	INL1Oral
    \end{itemize}
  \end{block}
\end{frame}

\begin{frame}[fragile]
  \begin{block}{Assignment format}
    \begin{itemize}
      \item Most can be done at any time.
      \item Some have a specified lab session.
        \begin{itemize}
          \item MANDATORY Seminar (31/1): usability (Sonja)  ON CAMPUS
          \item MANDATORY Seminar (pick 6/2): Impact considerations around 
            crypto systems (Sonja)  ON CAMPUS
          \item MANDATORY Lab (20-21/2): Introduction to ProVerif  (Karl and 
            Jesper)  ON CAMPUS
        \end{itemize}
      \item All assignments are individual.
    \end{itemize}
  \end{block}

  \begin{remark}[LabWeek]
    \begin{itemize}
      \item If you miss, you can catch up in LabWeek in June.
    \end{itemize}
  \end{remark}
\end{frame}

\subsection{Structure}

\begin{frame}[fragile]
  \begin{center}
    \huge
    Canvas
  \end{center}
\end{frame}


\section{What you should be able to do in the end}

\begin{frame}[fragile]
  \begin{block}{What you should be able to to do}
    \begin{itemize}
      \item Let's look at INL1Written
    \end{itemize}
  \end{block}
\end{frame}
