\mode*

\section{Digital Postbox}

Traditionally we send paper mail by addressing a letter to a physical address 
(including the recipient's name).
We write the address on an envelope and send it.
We want a digital transformation.
You should design a digital postbox and delivery service.

\mode<presentation>{%
\begin{frame}
  \begin{idea}[Digital mail]
    \begin{itemize}
      \item We send paper mail by writing an address on an envelope.
      \item We want to have a digital version.
    \end{itemize}
  \end{idea}

  \begin{remark}
    \begin{itemize}
      \item We have a few already: Kivra, eBoks, Digimail, Min myndighetspost.
      \item Managed by DIGG\footnote{%
          See \url{https://www.digg.se/digital-post}.
        }
    \end{itemize}
  \end{remark}
\end{frame}
}

We have the following requirements on the service:
\begin{frame}
\begin{description}
  \item[Registration]
    A user can register for a post box.
    As this post service will replace physical mail, it must be tied to a real 
    identity.

    \pause

  \item[Sending mail]
    A sender should be able to send mail to a recipient.
    This mail should not be readable by any intermediary.

  \item[Reading mail]
    The owner can read the mail.
    Only the owner should be able to do this --- no one else, not even service 
    staff!
    Recipient must be able to verify the authenticity of mail.

    \pause

  \item[Anonymous mail] (Extra feature)
    A sender can send a letter to a recipient, but neither the recipient nor 
    the service will learn the identity of the sender.
    The service will not learn the recipient either.
\end{description}
\end{frame}

Solving this assignment will touch upon almost every topic in the course.
Make sure to base your designs on the theory of the course, add references 
(that will help you).
\enquote{This feels secure} is not a convincing argument.
Likewise, \enquote{all connections should use TLS} will not cut it either; why 
do you want TLS, what properties do you need and which of those will TLS 
provide and why?

\mode<presentation>{%
  \begin{frame}
    \begin{exercise}[Design the digital mail service]
      \begin{description}
        \item[Registration]
          A user can register for a post box.
          This post service will replace physical mail.

        \item[Sending mail]
          A sender should be able to send mail to a recipient.
          Mail should not be readable by any intermediary.

        \item[Reading mail]
          The owner can read all their mail.
          Service staff should not be able to read mail!
          Recipient must be able to verify the authenticity of mail.
      \end{description}
    \end{exercise}

    \pause

    \begin{block}{Organization}
      \begin{itemize}
        \item Groups of around three--four people.
        \item Work for 45 minutes, take 15 min break (60 min incl break).
        \item Group presentations and discussions.
      \end{itemize}
    \end{block}
  \end{frame}

  \begin{frame}
    \begin{solution}
      \begin{itemize}
        \item Presentation from groups.
      \end{itemize}
    \end{solution}

    \begin{question}
      \begin{itemize}
        \item Limitations?
        \item Recovery from failure?
      \end{itemize}
    \end{question}
  \end{frame}

  \begin{frame}
    \begin{exercise}[Adapt for anonymity]
      \begin{description}
        \item[Anonymous mail]
          A sender can send a letter to a recipient, but neither the recipient 
          nor the service will learn the identity of the sender.
          The service will not learn the recipient either.
      \end{description}
    \end{exercise}

    \pause

    \begin{block}{Organization}
      \begin{itemize}
        \item Groups of around three--four people.
        \item Work for 20 minutes.
      \end{itemize}
    \end{block}
  \end{frame}
}

