\mode*

\section{Digital Postbox}

Traditionally we send paper mail by addressing a letter to a physical address 
(including the recipient's name).
We write the address on an envelope and send it.
We want a digital transformation.
You should design a digital postbox and delivery service.

\mode<presentation>{%
\begin{frame}
  \begin{idea}[Digital mail]
    \begin{itemize}
      \item We send paper mail by writing an address on an envelope.
      \item We want to have a digital version.
    \end{itemize}
  \end{idea}

  \begin{remark}
    \begin{itemize}
      \item We have a few already: Kivra, eBoks, Digimail, Min myndighetspost.
      \item Managed by DIGG\footnote{%
          See \url{https://www.digg.se/digital-post}.
        }
    \end{itemize}
  \end{remark}
\end{frame}
}

\begin{frame}
\begin{exercise}
  What functionality do we need? (High level)
\end{exercise}

\begin{onlyenv}<2>
  \begin{solution}
    We'd need at least the following:
    \begin{itemize}
      \item Registration
      \item Sending mail
      \item Reading mail
    \end{itemize}
  \end{solution}
\end{onlyenv}
\end{frame}

\begin{frame}
\begin{exercise}
  What are the requirements on the service?
\end{exercise}
\end{frame}

We have the following requirements on the service:
\begin{frame}
\begin{description}
  \item[Registration]
    A user can register for a post box.
    As this post service will replace physical mail, it must be tied to a real 
    identity with a physical address.

    \pause

  \item[Sending mail]
    A sender should be able to send mail to a recipient.
    This mail should not be readable by any intermediary.

  \item[Reading mail]
    The owner can read the mail.
    Only the owner should be able to do this --- no one else, not even service 
    staff!
    Recipient must be able to verify the authenticity of mail.
%
%  \item[Anonymous mail] (Extra feature)
%    A sender can send a letter to a recipient, but neither the recipient nor 
%    the service will learn the identity of the sender.
%    The service will not learn the recipient either.
\end{description}
\end{frame}

\begin{frame}
  \begin{remark}
    DIGG\footnote{Myndigheten för digital förvaltning} sets the official 
    requirements, see
    \url{https://www.digg.se/digitala-tjanster/digital-post/digital-post-for-dig-som-leverantor/allmanna-villkor/allmanna-villkor-fr.o.m.-16-januari-2023/bilaga-1-krav-pa-sakerhet-for-brevladeoperatorer}.
  \end{remark}
\end{frame}

Solving this assignment will touch upon almost every topic in the course.
Make sure to base your designs on the theory of the course, add references 
(that will help you).
\enquote{This feels secure} is not a convincing argument.
Likewise, \enquote{all connections should use TLS} will not cut it either; why 
do you want TLS, what properties do you need and which of those will TLS 
provide and why?

\mode<presentation>{%
  \begin{frame}
    \begin{exercise}[Design the digital mail service]
      \begin{description}
        \item[Registration]
          Alice can register for a post box.

        \item[Sending mail]
          Alice should be able to send mail to a recipient Bob.
          Mail should not be readable by any intermediary.

        \item[Reading mail]
          Bob can read all his received mail.
          Service staff should not be able to read mail!
          Bob must be able to verify the authenticity of mail, \eg that it's 
          from Alice.
      \end{description}
    \end{exercise}

    \pause

    \begin{block}{Organization}
      \begin{itemize}
        \item Groups of around three--four people.
        \item Pick one of the three tasks functionalities.
        \item Work for 15 minutes.
        \item Group presentations and discussions.
      \end{itemize}
    \end{block}
  \end{frame}

  \begin{frame}
    \begin{solution}
      \begin{itemize}
        \item Presentation from groups.
        \item Comments from everyone else.
      \end{itemize}
    \end{solution}

    \begin{question}
      \begin{itemize}
        \item Obstacles?
        \item Limitations?
        \item Recovery from failure?
        \item In whom do we trust?
      \end{itemize}
    \end{question}
  \end{frame}
}

Notes from the session:
\begin{itemize}
  \item Same as Signal protocol.
  \item Just tie in BankID to get real identities.
  \item The problem of end-to-end encryption: lost keys doesn't work. But it 
    still works with the existing digital mail services.
\end{itemize}

\begin{frame}
  \begin{exercise}
    What metadata are revealed to whom?
    How does this compare to the physical mail service?
  \end{exercise}
\end{frame}

\mode<presentation>{%
  \begin{frame}
    \begin{exercise}[Adapt for anonymity]
      \begin{description}
        \item[Weak sender anonymity]
          Alice can send a letter to Bob, but the service will not learn the 
          identity of the sender.

        \item[Sender anonymity]
          Alice can send a letter to Bob, but neither Bob nor the service will 
          learn the identity of the sender.

        \item[Sender--receiver anonymity]
          Same, but the service will not learn that Bob is the recipient 
          either.
      \end{description}
    \end{exercise}

    \pause

    \begin{block}{Organization}
      \begin{itemize}
        \item Groups of around three--four people.
        \item Work for 20 minutes.
        \item Group presentations and discussions.
      \end{itemize}
    \end{block}
  \end{frame}

  \begin{frame}
    \begin{solution}
      \begin{itemize}
        \item Presentation from groups.
        \item Comments from everyone else.
      \end{itemize}
    \end{solution}

    \begin{question}
      \begin{itemize}
        \item Obstacles?
        \item Limitations?
        \item Recovery from failure?
        \item In whom do we trust?
      \end{itemize}
    \end{question}
  \end{frame}
}

