\questions
% tags: crypto:zkp
% tags: E
In a zero-knowledge proof-of-knowledge
\begin{choices}
  \choice the verifier learns nothing.

  \CorrectChoice the verifier learns one bit, but cannot prove it to anyone 
  else.

  \choice the verifier learns the prover's secret.
\end{choices}


\question
% tags: crypto:smc
% tags: E
In secure multiparty-computation
\begin{checkboxes}
  \choice no-one learns anything.
  \CorrectChoice everyone learns the output.
  \choice no-one learns anything about the inputs.
  \CorrectChoice everyone learns about the inputs what can be inferred from the 
  output.
\end{checkboxes}
(Select all that apply.)


\question[3]
% tags: crypto:zkp
% tags: A:C:E
Alice has a private key~\(x\) and corresponding public key~\(g^x \mod p\) (for 
some prime~\(p\)).
(This could be a key in the ElGamal encryption/signature scheme.)
Eve knows \(g^x\) is Alice's public key.
Alice wants to prove she's Alice to Eve by using her public--private key-pair.

Why should Alice convince Eve with a zero-knowledge proof-of-knowledge?
Or, why should Alice \emph{not} use a digital signature?

\begin{solution}
  If Alice uses a digital signature, then Eve has a signed message by Alice.
  Eve can misuse this message.
  Eve can forward the signature to Bob and he will be able to verify it's 
  Alice's signature.
  (Eve might actually be able to trick Bob that she's Alice.)

  If Alice uses a zero-knowledge proof-of-knowledge, Eve learns only that 
  Alice's knows the private key corresponding to the public key --- but nothing 
  else.
  Eve cannot prove this to Bob later on.
  Actually, she has nothing to convince Bob that she has interacted with Alice.
\end{solution}
